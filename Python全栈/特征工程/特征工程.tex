\documentclass[11pt]{article}

    \usepackage[breakable]{tcolorbox}
    \usepackage{parskip} % Stop auto-indenting (to mimic markdown behaviour)
    
    \usepackage{fontspec, xunicode, xltxtra}
    \setmainfont{Microsoft YaHei}
    \usepackage{ctex}

    \usepackage{iftex}
    \ifPDFTeX


    	\usepackage[T1]{fontenc}
    	\usepackage{mathpazo}
    \else
    	\usepackage{fontspec}
    \fi

    % Basic figure setup, for now with no caption control since it's done
    % automatically by Pandoc (which extracts ![](path) syntax from Markdown).
    \usepackage{graphicx}
    % Maintain compatibility with old templates. Remove in nbconvert 6.0
    \let\Oldincludegraphics\includegraphics
    % Ensure that by default, figures have no caption (until we provide a
    % proper Figure object with a Caption API and a way to capture that
    % in the conversion process - todo).
    \usepackage{caption}
    \DeclareCaptionFormat{nocaption}{}
    \captionsetup{format=nocaption,aboveskip=0pt,belowskip=0pt}

    \usepackage[Export]{adjustbox} % Used to constrain images to a maximum size
    \adjustboxset{max size={0.9\linewidth}{0.9\paperheight}}
    \usepackage{float}
    \floatplacement{figure}{H} % forces figures to be placed at the correct location
    \usepackage{xcolor} % Allow colors to be defined
    \usepackage{enumerate} % Needed for markdown enumerations to work
    \usepackage{geometry} % Used to adjust the document margins
    \usepackage{amsmath} % Equations
    \usepackage{amssymb} % Equations
    \usepackage{textcomp} % defines textquotesingle
    % Hack from http://tex.stackexchange.com/a/47451/13684:
    \AtBeginDocument{%
        \def\PYZsq{\textquotesingle}% Upright quotes in Pygmentized code
    }
    \usepackage{upquote} % Upright quotes for verbatim code
    \usepackage{eurosym} % defines \euro
    \usepackage[mathletters]{ucs} % Extended unicode (utf-8) support
    \usepackage{fancyvrb} % verbatim replacement that allows latex
    \usepackage{grffile} % extends the file name processing of package graphics 
                         % to support a larger range
    \makeatletter % fix for grffile with XeLaTeX
    \def\Gread@@xetex#1{%
      \IfFileExists{"\Gin@base".bb}%
      {\Gread@eps{\Gin@base.bb}}%
      {\Gread@@xetex@aux#1}%
    }
    \makeatother

    % The hyperref package gives us a pdf with properly built
    % internal navigation ('pdf bookmarks' for the table of contents,
    % internal cross-reference links, web links for URLs, etc.)
    \usepackage{hyperref}
    % The default LaTeX title has an obnoxious amount of whitespace. By default,
    % titling removes some of it. It also provides customization options.
    \usepackage{titling}
    \usepackage{longtable} % longtable support required by pandoc >1.10
    \usepackage{booktabs}  % table support for pandoc > 1.12.2
    \usepackage[inline]{enumitem} % IRkernel/repr support (it uses the enumerate* environment)
    \usepackage[normalem]{ulem} % ulem is needed to support strikethroughs (\sout)
                                % normalem makes italics be italics, not underlines
    \usepackage{mathrsfs}
    

    
    % Colors for the hyperref package
    \definecolor{urlcolor}{rgb}{0,.145,.698}
    \definecolor{linkcolor}{rgb}{.71,0.21,0.01}
    \definecolor{citecolor}{rgb}{.12,.54,.11}

    % ANSI colors
    \definecolor{ansi-black}{HTML}{3E424D}
    \definecolor{ansi-black-intense}{HTML}{282C36}
    \definecolor{ansi-red}{HTML}{E75C58}
    \definecolor{ansi-red-intense}{HTML}{B22B31}
    \definecolor{ansi-green}{HTML}{00A250}
    \definecolor{ansi-green-intense}{HTML}{007427}
    \definecolor{ansi-yellow}{HTML}{DDB62B}
    \definecolor{ansi-yellow-intense}{HTML}{B27D12}
    \definecolor{ansi-blue}{HTML}{208FFB}
    \definecolor{ansi-blue-intense}{HTML}{0065CA}
    \definecolor{ansi-magenta}{HTML}{D160C4}
    \definecolor{ansi-magenta-intense}{HTML}{A03196}
    \definecolor{ansi-cyan}{HTML}{60C6C8}
    \definecolor{ansi-cyan-intense}{HTML}{258F8F}
    \definecolor{ansi-white}{HTML}{C5C1B4}
    \definecolor{ansi-white-intense}{HTML}{A1A6B2}
    \definecolor{ansi-default-inverse-fg}{HTML}{FFFFFF}
    \definecolor{ansi-default-inverse-bg}{HTML}{000000}

    % commands and environments needed by pandoc snippets
    % extracted from the output of `pandoc -s`
    \providecommand{\tightlist}{%
      \setlength{\itemsep}{0pt}\setlength{\parskip}{0pt}}
    \DefineVerbatimEnvironment{Highlighting}{Verbatim}{commandchars=\\\{\}}
    % Add ',fontsize=\small' for more characters per line
    \newenvironment{Shaded}{}{}
    \newcommand{\KeywordTok}[1]{\textcolor[rgb]{0.00,0.44,0.13}{\textbf{{#1}}}}
    \newcommand{\DataTypeTok}[1]{\textcolor[rgb]{0.56,0.13,0.00}{{#1}}}
    \newcommand{\DecValTok}[1]{\textcolor[rgb]{0.25,0.63,0.44}{{#1}}}
    \newcommand{\BaseNTok}[1]{\textcolor[rgb]{0.25,0.63,0.44}{{#1}}}
    \newcommand{\FloatTok}[1]{\textcolor[rgb]{0.25,0.63,0.44}{{#1}}}
    \newcommand{\CharTok}[1]{\textcolor[rgb]{0.25,0.44,0.63}{{#1}}}
    \newcommand{\StringTok}[1]{\textcolor[rgb]{0.25,0.44,0.63}{{#1}}}
    \newcommand{\CommentTok}[1]{\textcolor[rgb]{0.38,0.63,0.69}{\textit{{#1}}}}
    \newcommand{\OtherTok}[1]{\textcolor[rgb]{0.00,0.44,0.13}{{#1}}}
    \newcommand{\AlertTok}[1]{\textcolor[rgb]{1.00,0.00,0.00}{\textbf{{#1}}}}
    \newcommand{\FunctionTok}[1]{\textcolor[rgb]{0.02,0.16,0.49}{{#1}}}
    \newcommand{\RegionMarkerTok}[1]{{#1}}
    \newcommand{\ErrorTok}[1]{\textcolor[rgb]{1.00,0.00,0.00}{\textbf{{#1}}}}
    \newcommand{\NormalTok}[1]{{#1}}
    
    % Additional commands for more recent versions of Pandoc
    \newcommand{\ConstantTok}[1]{\textcolor[rgb]{0.53,0.00,0.00}{{#1}}}
    \newcommand{\SpecialCharTok}[1]{\textcolor[rgb]{0.25,0.44,0.63}{{#1}}}
    \newcommand{\VerbatimStringTok}[1]{\textcolor[rgb]{0.25,0.44,0.63}{{#1}}}
    \newcommand{\SpecialStringTok}[1]{\textcolor[rgb]{0.73,0.40,0.53}{{#1}}}
    \newcommand{\ImportTok}[1]{{#1}}
    \newcommand{\DocumentationTok}[1]{\textcolor[rgb]{0.73,0.13,0.13}{\textit{{#1}}}}
    \newcommand{\AnnotationTok}[1]{\textcolor[rgb]{0.38,0.63,0.69}{\textbf{\textit{{#1}}}}}
    \newcommand{\CommentVarTok}[1]{\textcolor[rgb]{0.38,0.63,0.69}{\textbf{\textit{{#1}}}}}
    \newcommand{\VariableTok}[1]{\textcolor[rgb]{0.10,0.09,0.49}{{#1}}}
    \newcommand{\ControlFlowTok}[1]{\textcolor[rgb]{0.00,0.44,0.13}{\textbf{{#1}}}}
    \newcommand{\OperatorTok}[1]{\textcolor[rgb]{0.40,0.40,0.40}{{#1}}}
    \newcommand{\BuiltInTok}[1]{{#1}}
    \newcommand{\ExtensionTok}[1]{{#1}}
    \newcommand{\PreprocessorTok}[1]{\textcolor[rgb]{0.74,0.48,0.00}{{#1}}}
    \newcommand{\AttributeTok}[1]{\textcolor[rgb]{0.49,0.56,0.16}{{#1}}}
    \newcommand{\InformationTok}[1]{\textcolor[rgb]{0.38,0.63,0.69}{\textbf{\textit{{#1}}}}}
    \newcommand{\WarningTok}[1]{\textcolor[rgb]{0.38,0.63,0.69}{\textbf{\textit{{#1}}}}}
    
    
    % Define a nice break command that doesn't care if a line doesn't already
    % exist.
    \def\br{\hspace*{\fill} \\* }
    % Math Jax compatibility definitions
    \def\gt{>}
    \def\lt{<}
    \let\Oldtex\TeX
    \let\Oldlatex\LaTeX
    \renewcommand{\TeX}{\textrm{\Oldtex}}
    \renewcommand{\LaTeX}{\textrm{\Oldlatex}}
    % Document parameters
    % Document title
    \title{特征工程}
    
    
    
    
    
% Pygments definitions
\makeatletter
\def\PY@reset{\let\PY@it=\relax \let\PY@bf=\relax%
    \let\PY@ul=\relax \let\PY@tc=\relax%
    \let\PY@bc=\relax \let\PY@ff=\relax}
\def\PY@tok#1{\csname PY@tok@#1\endcsname}
\def\PY@toks#1+{\ifx\relax#1\empty\else%
    \PY@tok{#1}\expandafter\PY@toks\fi}
\def\PY@do#1{\PY@bc{\PY@tc{\PY@ul{%
    \PY@it{\PY@bf{\PY@ff{#1}}}}}}}
\def\PY#1#2{\PY@reset\PY@toks#1+\relax+\PY@do{#2}}

\expandafter\def\csname PY@tok@w\endcsname{\def\PY@tc##1{\textcolor[rgb]{0.73,0.73,0.73}{##1}}}
\expandafter\def\csname PY@tok@c\endcsname{\let\PY@it=\textit\def\PY@tc##1{\textcolor[rgb]{0.25,0.50,0.50}{##1}}}
\expandafter\def\csname PY@tok@cp\endcsname{\def\PY@tc##1{\textcolor[rgb]{0.74,0.48,0.00}{##1}}}
\expandafter\def\csname PY@tok@k\endcsname{\let\PY@bf=\textbf\def\PY@tc##1{\textcolor[rgb]{0.00,0.50,0.00}{##1}}}
\expandafter\def\csname PY@tok@kp\endcsname{\def\PY@tc##1{\textcolor[rgb]{0.00,0.50,0.00}{##1}}}
\expandafter\def\csname PY@tok@kt\endcsname{\def\PY@tc##1{\textcolor[rgb]{0.69,0.00,0.25}{##1}}}
\expandafter\def\csname PY@tok@o\endcsname{\def\PY@tc##1{\textcolor[rgb]{0.40,0.40,0.40}{##1}}}
\expandafter\def\csname PY@tok@ow\endcsname{\let\PY@bf=\textbf\def\PY@tc##1{\textcolor[rgb]{0.67,0.13,1.00}{##1}}}
\expandafter\def\csname PY@tok@nb\endcsname{\def\PY@tc##1{\textcolor[rgb]{0.00,0.50,0.00}{##1}}}
\expandafter\def\csname PY@tok@nf\endcsname{\def\PY@tc##1{\textcolor[rgb]{0.00,0.00,1.00}{##1}}}
\expandafter\def\csname PY@tok@nc\endcsname{\let\PY@bf=\textbf\def\PY@tc##1{\textcolor[rgb]{0.00,0.00,1.00}{##1}}}
\expandafter\def\csname PY@tok@nn\endcsname{\let\PY@bf=\textbf\def\PY@tc##1{\textcolor[rgb]{0.00,0.00,1.00}{##1}}}
\expandafter\def\csname PY@tok@ne\endcsname{\let\PY@bf=\textbf\def\PY@tc##1{\textcolor[rgb]{0.82,0.25,0.23}{##1}}}
\expandafter\def\csname PY@tok@nv\endcsname{\def\PY@tc##1{\textcolor[rgb]{0.10,0.09,0.49}{##1}}}
\expandafter\def\csname PY@tok@no\endcsname{\def\PY@tc##1{\textcolor[rgb]{0.53,0.00,0.00}{##1}}}
\expandafter\def\csname PY@tok@nl\endcsname{\def\PY@tc##1{\textcolor[rgb]{0.63,0.63,0.00}{##1}}}
\expandafter\def\csname PY@tok@ni\endcsname{\let\PY@bf=\textbf\def\PY@tc##1{\textcolor[rgb]{0.60,0.60,0.60}{##1}}}
\expandafter\def\csname PY@tok@na\endcsname{\def\PY@tc##1{\textcolor[rgb]{0.49,0.56,0.16}{##1}}}
\expandafter\def\csname PY@tok@nt\endcsname{\let\PY@bf=\textbf\def\PY@tc##1{\textcolor[rgb]{0.00,0.50,0.00}{##1}}}
\expandafter\def\csname PY@tok@nd\endcsname{\def\PY@tc##1{\textcolor[rgb]{0.67,0.13,1.00}{##1}}}
\expandafter\def\csname PY@tok@s\endcsname{\def\PY@tc##1{\textcolor[rgb]{0.73,0.13,0.13}{##1}}}
\expandafter\def\csname PY@tok@sd\endcsname{\let\PY@it=\textit\def\PY@tc##1{\textcolor[rgb]{0.73,0.13,0.13}{##1}}}
\expandafter\def\csname PY@tok@si\endcsname{\let\PY@bf=\textbf\def\PY@tc##1{\textcolor[rgb]{0.73,0.40,0.53}{##1}}}
\expandafter\def\csname PY@tok@se\endcsname{\let\PY@bf=\textbf\def\PY@tc##1{\textcolor[rgb]{0.73,0.40,0.13}{##1}}}
\expandafter\def\csname PY@tok@sr\endcsname{\def\PY@tc##1{\textcolor[rgb]{0.73,0.40,0.53}{##1}}}
\expandafter\def\csname PY@tok@ss\endcsname{\def\PY@tc##1{\textcolor[rgb]{0.10,0.09,0.49}{##1}}}
\expandafter\def\csname PY@tok@sx\endcsname{\def\PY@tc##1{\textcolor[rgb]{0.00,0.50,0.00}{##1}}}
\expandafter\def\csname PY@tok@m\endcsname{\def\PY@tc##1{\textcolor[rgb]{0.40,0.40,0.40}{##1}}}
\expandafter\def\csname PY@tok@gh\endcsname{\let\PY@bf=\textbf\def\PY@tc##1{\textcolor[rgb]{0.00,0.00,0.50}{##1}}}
\expandafter\def\csname PY@tok@gu\endcsname{\let\PY@bf=\textbf\def\PY@tc##1{\textcolor[rgb]{0.50,0.00,0.50}{##1}}}
\expandafter\def\csname PY@tok@gd\endcsname{\def\PY@tc##1{\textcolor[rgb]{0.63,0.00,0.00}{##1}}}
\expandafter\def\csname PY@tok@gi\endcsname{\def\PY@tc##1{\textcolor[rgb]{0.00,0.63,0.00}{##1}}}
\expandafter\def\csname PY@tok@gr\endcsname{\def\PY@tc##1{\textcolor[rgb]{1.00,0.00,0.00}{##1}}}
\expandafter\def\csname PY@tok@ge\endcsname{\let\PY@it=\textit}
\expandafter\def\csname PY@tok@gs\endcsname{\let\PY@bf=\textbf}
\expandafter\def\csname PY@tok@gp\endcsname{\let\PY@bf=\textbf\def\PY@tc##1{\textcolor[rgb]{0.00,0.00,0.50}{##1}}}
\expandafter\def\csname PY@tok@go\endcsname{\def\PY@tc##1{\textcolor[rgb]{0.53,0.53,0.53}{##1}}}
\expandafter\def\csname PY@tok@gt\endcsname{\def\PY@tc##1{\textcolor[rgb]{0.00,0.27,0.87}{##1}}}
\expandafter\def\csname PY@tok@err\endcsname{\def\PY@bc##1{\setlength{\fboxsep}{0pt}\fcolorbox[rgb]{1.00,0.00,0.00}{1,1,1}{\strut ##1}}}
\expandafter\def\csname PY@tok@kc\endcsname{\let\PY@bf=\textbf\def\PY@tc##1{\textcolor[rgb]{0.00,0.50,0.00}{##1}}}
\expandafter\def\csname PY@tok@kd\endcsname{\let\PY@bf=\textbf\def\PY@tc##1{\textcolor[rgb]{0.00,0.50,0.00}{##1}}}
\expandafter\def\csname PY@tok@kn\endcsname{\let\PY@bf=\textbf\def\PY@tc##1{\textcolor[rgb]{0.00,0.50,0.00}{##1}}}
\expandafter\def\csname PY@tok@kr\endcsname{\let\PY@bf=\textbf\def\PY@tc##1{\textcolor[rgb]{0.00,0.50,0.00}{##1}}}
\expandafter\def\csname PY@tok@bp\endcsname{\def\PY@tc##1{\textcolor[rgb]{0.00,0.50,0.00}{##1}}}
\expandafter\def\csname PY@tok@fm\endcsname{\def\PY@tc##1{\textcolor[rgb]{0.00,0.00,1.00}{##1}}}
\expandafter\def\csname PY@tok@vc\endcsname{\def\PY@tc##1{\textcolor[rgb]{0.10,0.09,0.49}{##1}}}
\expandafter\def\csname PY@tok@vg\endcsname{\def\PY@tc##1{\textcolor[rgb]{0.10,0.09,0.49}{##1}}}
\expandafter\def\csname PY@tok@vi\endcsname{\def\PY@tc##1{\textcolor[rgb]{0.10,0.09,0.49}{##1}}}
\expandafter\def\csname PY@tok@vm\endcsname{\def\PY@tc##1{\textcolor[rgb]{0.10,0.09,0.49}{##1}}}
\expandafter\def\csname PY@tok@sa\endcsname{\def\PY@tc##1{\textcolor[rgb]{0.73,0.13,0.13}{##1}}}
\expandafter\def\csname PY@tok@sb\endcsname{\def\PY@tc##1{\textcolor[rgb]{0.73,0.13,0.13}{##1}}}
\expandafter\def\csname PY@tok@sc\endcsname{\def\PY@tc##1{\textcolor[rgb]{0.73,0.13,0.13}{##1}}}
\expandafter\def\csname PY@tok@dl\endcsname{\def\PY@tc##1{\textcolor[rgb]{0.73,0.13,0.13}{##1}}}
\expandafter\def\csname PY@tok@s2\endcsname{\def\PY@tc##1{\textcolor[rgb]{0.73,0.13,0.13}{##1}}}
\expandafter\def\csname PY@tok@sh\endcsname{\def\PY@tc##1{\textcolor[rgb]{0.73,0.13,0.13}{##1}}}
\expandafter\def\csname PY@tok@s1\endcsname{\def\PY@tc##1{\textcolor[rgb]{0.73,0.13,0.13}{##1}}}
\expandafter\def\csname PY@tok@mb\endcsname{\def\PY@tc##1{\textcolor[rgb]{0.40,0.40,0.40}{##1}}}
\expandafter\def\csname PY@tok@mf\endcsname{\def\PY@tc##1{\textcolor[rgb]{0.40,0.40,0.40}{##1}}}
\expandafter\def\csname PY@tok@mh\endcsname{\def\PY@tc##1{\textcolor[rgb]{0.40,0.40,0.40}{##1}}}
\expandafter\def\csname PY@tok@mi\endcsname{\def\PY@tc##1{\textcolor[rgb]{0.40,0.40,0.40}{##1}}}
\expandafter\def\csname PY@tok@il\endcsname{\def\PY@tc##1{\textcolor[rgb]{0.40,0.40,0.40}{##1}}}
\expandafter\def\csname PY@tok@mo\endcsname{\def\PY@tc##1{\textcolor[rgb]{0.40,0.40,0.40}{##1}}}
\expandafter\def\csname PY@tok@ch\endcsname{\let\PY@it=\textit\def\PY@tc##1{\textcolor[rgb]{0.25,0.50,0.50}{##1}}}
\expandafter\def\csname PY@tok@cm\endcsname{\let\PY@it=\textit\def\PY@tc##1{\textcolor[rgb]{0.25,0.50,0.50}{##1}}}
\expandafter\def\csname PY@tok@cpf\endcsname{\let\PY@it=\textit\def\PY@tc##1{\textcolor[rgb]{0.25,0.50,0.50}{##1}}}
\expandafter\def\csname PY@tok@c1\endcsname{\let\PY@it=\textit\def\PY@tc##1{\textcolor[rgb]{0.25,0.50,0.50}{##1}}}
\expandafter\def\csname PY@tok@cs\endcsname{\let\PY@it=\textit\def\PY@tc##1{\textcolor[rgb]{0.25,0.50,0.50}{##1}}}

\def\PYZbs{\char`\\}
\def\PYZus{\char`\_}
\def\PYZob{\char`\{}
\def\PYZcb{\char`\}}
\def\PYZca{\char`\^}
\def\PYZam{\char`\&}
\def\PYZlt{\char`\<}
\def\PYZgt{\char`\>}
\def\PYZsh{\char`\#}
\def\PYZpc{\char`\%}
\def\PYZdl{\char`\$}
\def\PYZhy{\char`\-}
\def\PYZsq{\char`\'}
\def\PYZdq{\char`\"}
\def\PYZti{\char`\~}
% for compatibility with earlier versions
\def\PYZat{@}
\def\PYZlb{[}
\def\PYZrb{]}
\makeatother


    % For linebreaks inside Verbatim environment from package fancyvrb. 
    \makeatletter
        \newbox\Wrappedcontinuationbox 
        \newbox\Wrappedvisiblespacebox 
        \newcommand*\Wrappedvisiblespace {\textcolor{red}{\textvisiblespace}} 
        \newcommand*\Wrappedcontinuationsymbol {\textcolor{red}{\llap{\tiny$\m@th\hookrightarrow$}}} 
        \newcommand*\Wrappedcontinuationindent {3ex } 
        \newcommand*\Wrappedafterbreak {\kern\Wrappedcontinuationindent\copy\Wrappedcontinuationbox} 
        % Take advantage of the already applied Pygments mark-up to insert 
        % potential linebreaks for TeX processing. 
        %        {, <, #, %, $, ' and ": go to next line. 
        %        _, }, ^, &, >, - and ~: stay at end of broken line. 
        % Use of \textquotesingle for straight quote. 
        \newcommand*\Wrappedbreaksatspecials {% 
            \def\PYGZus{\discretionary{\char`\_}{\Wrappedafterbreak}{\char`\_}}% 
            \def\PYGZob{\discretionary{}{\Wrappedafterbreak\char`\{}{\char`\{}}% 
            \def\PYGZcb{\discretionary{\char`\}}{\Wrappedafterbreak}{\char`\}}}% 
            \def\PYGZca{\discretionary{\char`\^}{\Wrappedafterbreak}{\char`\^}}% 
            \def\PYGZam{\discretionary{\char`\&}{\Wrappedafterbreak}{\char`\&}}% 
            \def\PYGZlt{\discretionary{}{\Wrappedafterbreak\char`\<}{\char`\<}}% 
            \def\PYGZgt{\discretionary{\char`\>}{\Wrappedafterbreak}{\char`\>}}% 
            \def\PYGZsh{\discretionary{}{\Wrappedafterbreak\char`\#}{\char`\#}}% 
            \def\PYGZpc{\discretionary{}{\Wrappedafterbreak\char`\%}{\char`\%}}% 
            \def\PYGZdl{\discretionary{}{\Wrappedafterbreak\char`\$}{\char`\$}}% 
            \def\PYGZhy{\discretionary{\char`\-}{\Wrappedafterbreak}{\char`\-}}% 
            \def\PYGZsq{\discretionary{}{\Wrappedafterbreak\textquotesingle}{\textquotesingle}}% 
            \def\PYGZdq{\discretionary{}{\Wrappedafterbreak\char`\"}{\char`\"}}% 
            \def\PYGZti{\discretionary{\char`\~}{\Wrappedafterbreak}{\char`\~}}% 
        } 
        % Some characters . , ; ? ! / are not pygmentized. 
        % This macro makes them "active" and they will insert potential linebreaks 
        \newcommand*\Wrappedbreaksatpunct {% 
            \lccode`\~`\.\lowercase{\def~}{\discretionary{\hbox{\char`\.}}{\Wrappedafterbreak}{\hbox{\char`\.}}}% 
            \lccode`\~`\,\lowercase{\def~}{\discretionary{\hbox{\char`\,}}{\Wrappedafterbreak}{\hbox{\char`\,}}}% 
            \lccode`\~`\;\lowercase{\def~}{\discretionary{\hbox{\char`\;}}{\Wrappedafterbreak}{\hbox{\char`\;}}}% 
            \lccode`\~`\:\lowercase{\def~}{\discretionary{\hbox{\char`\:}}{\Wrappedafterbreak}{\hbox{\char`\:}}}% 
            \lccode`\~`\?\lowercase{\def~}{\discretionary{\hbox{\char`\?}}{\Wrappedafterbreak}{\hbox{\char`\?}}}% 
            \lccode`\~`\!\lowercase{\def~}{\discretionary{\hbox{\char`\!}}{\Wrappedafterbreak}{\hbox{\char`\!}}}% 
            \lccode`\~`\/\lowercase{\def~}{\discretionary{\hbox{\char`\/}}{\Wrappedafterbreak}{\hbox{\char`\/}}}% 
            \catcode`\.\active
            \catcode`\,\active 
            \catcode`\;\active
            \catcode`\:\active
            \catcode`\?\active
            \catcode`\!\active
            \catcode`\/\active 
            \lccode`\~`\~ 	
        }
    \makeatother

    \let\OriginalVerbatim=\Verbatim
    \makeatletter
    \renewcommand{\Verbatim}[1][1]{%
        %\parskip\z@skip
        \sbox\Wrappedcontinuationbox {\Wrappedcontinuationsymbol}%
        \sbox\Wrappedvisiblespacebox {\FV@SetupFont\Wrappedvisiblespace}%
        \def\FancyVerbFormatLine ##1{\hsize\linewidth
            \vtop{\raggedright\hyphenpenalty\z@\exhyphenpenalty\z@
                \doublehyphendemerits\z@\finalhyphendemerits\z@
                \strut ##1\strut}%
        }%
        % If the linebreak is at a space, the latter will be displayed as visible
        % space at end of first line, and a continuation symbol starts next line.
        % Stretch/shrink are however usually zero for typewriter font.
        \def\FV@Space {%
            \nobreak\hskip\z@ plus\fontdimen3\font minus\fontdimen4\font
            \discretionary{\copy\Wrappedvisiblespacebox}{\Wrappedafterbreak}
            {\kern\fontdimen2\font}%
        }%
        
        % Allow breaks at special characters using \PYG... macros.
        \Wrappedbreaksatspecials
        % Breaks at punctuation characters . , ; ? ! and / need catcode=\active 	
        \OriginalVerbatim[#1,codes*=\Wrappedbreaksatpunct]%
    }
    \makeatother

    % Exact colors from NB
    \definecolor{incolor}{HTML}{303F9F}
    \definecolor{outcolor}{HTML}{D84315}
    \definecolor{cellborder}{HTML}{CFCFCF}
    \definecolor{cellbackground}{HTML}{F7F7F7}
    
    % prompt
    \makeatletter
    \newcommand{\boxspacing}{\kern\kvtcb@left@rule\kern\kvtcb@boxsep}
    \makeatother
    \newcommand{\prompt}[4]{
        \ttfamily\llap{{\color{#2}[#3]:\hspace{3pt}#4}}\vspace{-\baselineskip}
    }
    

    
    % Prevent overflowing lines due to hard-to-break entities
    \sloppy 
    % Setup hyperref package
    \hypersetup{
      breaklinks=true,  % so long urls are correctly broken across lines
      colorlinks=true,
      urlcolor=urlcolor,
      linkcolor=linkcolor,
      citecolor=citecolor,
      }
    % Slightly bigger margins than the latex defaults
    
    \geometry{verbose,tmargin=1in,bmargin=1in,lmargin=1in,rmargin=1in}
    
    

\begin{document}
    
    \maketitle
    
    

    
    \hypertarget{ux83b7ux53d6ux6570ux636e}{%
\subsection{1.获取数据}\label{ux83b7ux53d6ux6570ux636e}}

    \begin{tcolorbox}[breakable, size=fbox, boxrule=1pt, pad at break*=1mm,colback=cellbackground, colframe=cellborder]
\prompt{In}{incolor}{1}{\boxspacing}
\begin{Verbatim}[commandchars=\\\{\}]
\PY{k+kn}{import} \PY{n+nn}{os}
\PY{k+kn}{import} \PY{n+nn}{pandas} \PY{k}{as} \PY{n+nn}{pd}
\PY{k}{def} \PY{n+nf}{load\PYZus{}housing\PYZus{}data}\PY{p}{(}\PY{n}{housing\PYZus{}path} \PY{o}{=} \PY{l+s+s1}{\PYZsq{}}\PY{l+s+s1}{./}\PY{l+s+s1}{\PYZsq{}}\PY{p}{)}\PY{p}{:} \PY{c+c1}{\PYZsh{} 导入数据}
    \PY{n}{csv\PYZus{}path} \PY{o}{=} \PY{n}{os}\PY{o}{.}\PY{n}{path}\PY{o}{.}\PY{n}{join}\PY{p}{(}\PY{n}{housing\PYZus{}path}\PY{p}{,} \PY{l+s+s1}{\PYZsq{}}\PY{l+s+s1}{housing.csv}\PY{l+s+s1}{\PYZsq{}}\PY{p}{)}
    \PY{k}{return} \PY{n}{pd}\PY{o}{.}\PY{n}{read\PYZus{}csv}\PY{p}{(}\PY{n}{csv\PYZus{}path}\PY{p}{)}
\end{Verbatim}
\end{tcolorbox}

    \begin{tcolorbox}[breakable, size=fbox, boxrule=1pt, pad at break*=1mm,colback=cellbackground, colframe=cellborder]
\prompt{In}{incolor}{2}{\boxspacing}
\begin{Verbatim}[commandchars=\\\{\}]
\PY{n}{data\PYZus{}housing} \PY{o}{=} \PY{n}{load\PYZus{}housing\PYZus{}data}\PY{p}{(}\PY{p}{)}
\PY{n}{data\PYZus{}housing}\PY{o}{.}\PY{n}{head}\PY{p}{(}\PY{p}{)}
\end{Verbatim}
\end{tcolorbox}

            \begin{tcolorbox}[breakable, size=fbox, boxrule=.5pt, pad at break*=1mm, opacityfill=0]
\prompt{Out}{outcolor}{2}{\boxspacing}
\begin{Verbatim}[commandchars=\\\{\}]
   longitude  latitude  housing\_median\_age  total\_rooms  total\_bedrooms  \textbackslash{}
0    -122.23     37.88                41.0        880.0           129.0
1    -122.22     37.86                21.0       7099.0          1106.0
2    -122.24     37.85                52.0       1467.0           190.0
3    -122.25     37.85                52.0       1274.0           235.0
4    -122.25     37.85                52.0       1627.0           280.0

   population  households  median\_income  median\_house\_value ocean\_proximity
0       322.0       126.0         8.3252            452600.0        NEAR BAY
1      2401.0      1138.0         8.3014            358500.0        NEAR BAY
2       496.0       177.0         7.2574            352100.0        NEAR BAY
3       558.0       219.0         5.6431            341300.0        NEAR BAY
4       565.0       259.0         3.8462            342200.0        NEAR BAY
\end{Verbatim}
\end{tcolorbox}
        
    \begin{tcolorbox}[breakable, size=fbox, boxrule=1pt, pad at break*=1mm,colback=cellbackground, colframe=cellborder]
\prompt{In}{incolor}{3}{\boxspacing}
\begin{Verbatim}[commandchars=\\\{\}]
\PY{n}{data\PYZus{}housing}\PY{o}{.}\PY{n}{shape}
\end{Verbatim}
\end{tcolorbox}

            \begin{tcolorbox}[breakable, size=fbox, boxrule=.5pt, pad at break*=1mm, opacityfill=0]
\prompt{Out}{outcolor}{3}{\boxspacing}
\begin{Verbatim}[commandchars=\\\{\}]
(20640, 10)
\end{Verbatim}
\end{tcolorbox}
        
    \begin{tcolorbox}[breakable, size=fbox, boxrule=1pt, pad at break*=1mm,colback=cellbackground, colframe=cellborder]
\prompt{In}{incolor}{4}{\boxspacing}
\begin{Verbatim}[commandchars=\\\{\}]
\PY{n}{data\PYZus{}housing}\PY{o}{.}\PY{n}{info}\PY{p}{(}\PY{p}{)}  \PY{c+c1}{\PYZsh{} 数据集的简单描述,得到每个属性类型和非空值数据量}
\end{Verbatim}
\end{tcolorbox}

    \begin{Verbatim}[commandchars=\\\{\}]
<class 'pandas.core.frame.DataFrame'>
RangeIndex: 20640 entries, 0 to 20639
Data columns (total 10 columns):
longitude             20640 non-null float64
latitude              20640 non-null float64
housing\_median\_age    20640 non-null float64
total\_rooms           20640 non-null float64
total\_bedrooms        20433 non-null float64
population            20640 non-null float64
households            20640 non-null float64
median\_income         20640 non-null float64
median\_house\_value    20640 non-null float64
ocean\_proximity       20640 non-null object
dtypes: float64(9), object(1)
memory usage: 1.6+ MB
    \end{Verbatim}

    \begin{tcolorbox}[breakable, size=fbox, boxrule=1pt, pad at break*=1mm,colback=cellbackground, colframe=cellborder]
\prompt{In}{incolor}{5}{\boxspacing}
\begin{Verbatim}[commandchars=\\\{\}]
\PY{n}{data\PYZus{}housing}\PY{o}{.}\PY{n}{describe}\PY{p}{(}\PY{p}{)}  \PY{c+c1}{\PYZsh{} 显示数值属性摘要}
\end{Verbatim}
\end{tcolorbox}

            \begin{tcolorbox}[breakable, size=fbox, boxrule=.5pt, pad at break*=1mm, opacityfill=0]
\prompt{Out}{outcolor}{5}{\boxspacing}
\begin{Verbatim}[commandchars=\\\{\}]
          longitude      latitude  housing\_median\_age   total\_rooms  \textbackslash{}
count  20640.000000  20640.000000        20640.000000  20640.000000
mean    -119.569704     35.631861           28.639486   2635.763081
std        2.003532      2.135952           12.585558   2181.615252
min     -124.350000     32.540000            1.000000      2.000000
25\%     -121.800000     33.930000           18.000000   1447.750000
50\%     -118.490000     34.260000           29.000000   2127.000000
75\%     -118.010000     37.710000           37.000000   3148.000000
max     -114.310000     41.950000           52.000000  39320.000000

       total\_bedrooms    population    households  median\_income  \textbackslash{}
count    20433.000000  20640.000000  20640.000000   20640.000000
mean       537.870553   1425.476744    499.539680       3.870671
std        421.385070   1132.462122    382.329753       1.899822
min          1.000000      3.000000      1.000000       0.499900
25\%        296.000000    787.000000    280.000000       2.563400
50\%        435.000000   1166.000000    409.000000       3.534800
75\%        647.000000   1725.000000    605.000000       4.743250
max       6445.000000  35682.000000   6082.000000      15.000100

       median\_house\_value
count        20640.000000
mean        206855.816909
std         115395.615874
min          14999.000000
25\%         119600.000000
50\%         179700.000000
75\%         264725.000000
max         500001.000000
\end{Verbatim}
\end{tcolorbox}
        
    \begin{tcolorbox}[breakable, size=fbox, boxrule=1pt, pad at break*=1mm,colback=cellbackground, colframe=cellborder]
\prompt{In}{incolor}{6}{\boxspacing}
\begin{Verbatim}[commandchars=\\\{\}]
\PY{n}{data\PYZus{}housing}\PY{p}{[}\PY{l+s+s1}{\PYZsq{}}\PY{l+s+s1}{ocean\PYZus{}proximity}\PY{l+s+s1}{\PYZsq{}}\PY{p}{]}\PY{o}{.}\PY{n}{value\PYZus{}counts}\PY{p}{(}\PY{p}{)}  \PY{c+c1}{\PYZsh{} 查看多少种类, 5种分类,获得每种分类下有多少区域}
\end{Verbatim}
\end{tcolorbox}

            \begin{tcolorbox}[breakable, size=fbox, boxrule=.5pt, pad at break*=1mm, opacityfill=0]
\prompt{Out}{outcolor}{6}{\boxspacing}
\begin{Verbatim}[commandchars=\\\{\}]
<1H OCEAN     9136
INLAND        6551
NEAR OCEAN    2658
NEAR BAY      2290
ISLAND           5
Name: ocean\_proximity, dtype: int64
\end{Verbatim}
\end{tcolorbox}
        
    \begin{tcolorbox}[breakable, size=fbox, boxrule=1pt, pad at break*=1mm,colback=cellbackground, colframe=cellborder]
\prompt{In}{incolor}{7}{\boxspacing}
\begin{Verbatim}[commandchars=\\\{\}]
\PY{c+c1}{\PYZsh{} 在jupyter内绘制图}
\PY{o}{\PYZpc{}}\PY{k}{matplotlib} inline
\PY{k+kn}{import} \PY{n+nn}{matplotlib}\PY{n+nn}{.}\PY{n+nn}{pyplot} \PY{k}{as} \PY{n+nn}{plt}
\PY{n}{data\PYZus{}housing}\PY{o}{.}\PY{n}{hist}\PY{p}{(}\PY{n}{bins} \PY{o}{=} \PY{l+m+mi}{50}\PY{p}{,} \PY{n}{figsize} \PY{o}{=} \PY{p}{(}\PY{l+m+mi}{20}\PY{p}{,} \PY{l+m+mi}{15}\PY{p}{)}\PY{p}{)}
\PY{n}{plt}\PY{o}{.}\PY{n}{show}\PY{p}{(}\PY{p}{)}
\end{Verbatim}
\end{tcolorbox}

    \begin{center}
    \adjustimage{max size={0.9\linewidth}{0.9\paperheight}}{特征工程_files/特征工程_7_0.png}
    \end{center}
    { \hspace*{\fill} \\}
    
    \hypertarget{ux89c2ux5bdfux56feux8868ux603bux7ed3}{%
\subsubsection{观察图表总结:}\label{ux89c2ux5bdfux56feux8868ux603bux7ed3}}

\hypertarget{ux623fux9f84ux548cux623fux4ef7ux88abux8bbeux5b9aux4e86ux4e0aux9650}{%
\paragraph{房龄和房价被设定了上限}\label{ux623fux9f84ux548cux623fux4ef7ux88abux8bbeux5b9aux4e86ux4e0aux9650}}

\begin{enumerate}
\def\labelenumi{\arabic{enumi}.}
\tightlist
\item
  设置了上限的区域,重新收集标签值
\item
  把设置了上限的区域数据移除
\end{enumerate}

\hypertarget{ux91cdux5c3e-ux5934ux9ad8ux5c3eux957f}{%
\paragraph{重尾 头高尾长}\label{ux91cdux5c3e-ux5934ux9ad8ux5c3eux957f}}

\begin{itemize}
\tightlist
\item
  转换数据,把数据形状变成偏钟型分布(正态分布) \#\#\#\#\# 收入特征
\item
  数据提供的上游证实 数据单位 年薪 万美元,提前对特征进行了缩放
\end{itemize}

    \hypertarget{ux5206ux5272ux8badux7ec3ux96c6ux548cux6d4bux8bd5ux96c6}{%
\subsection{2.分割训练集和测试集}\label{ux5206ux5272ux8badux7ec3ux96c6ux548cux6d4bux8bd5ux96c6}}

\begin{itemize}
\tightlist
\item
  训练集用于模型训练 占整个数据集的80\%,测试集占20\%
\end{itemize}

    \begin{tcolorbox}[breakable, size=fbox, boxrule=1pt, pad at break*=1mm,colback=cellbackground, colframe=cellborder]
\prompt{In}{incolor}{8}{\boxspacing}
\begin{Verbatim}[commandchars=\\\{\}]
\PY{k+kn}{import} \PY{n+nn}{numpy} \PY{k}{as} \PY{n+nn}{np}
\PY{c+c1}{\PYZsh{} 对原来的数组进行重新洗牌,随机打乱原来的元素顺序}
\PY{c+c1}{\PYZsh{} data:传入数据    test\PYZus{}radio:数据集和测试集的比例}
\PY{k}{def} \PY{n+nf}{split\PYZus{}train\PYZus{}test}\PY{p}{(}\PY{n}{data}\PY{p}{,} \PY{n}{test\PYZus{}radio}\PY{p}{)}\PY{p}{:}
    \PY{c+c1}{\PYZsh{} 随机数种子}
    \PY{n}{np}\PY{o}{.}\PY{n}{random}\PY{o}{.}\PY{n}{seed}\PY{p}{(}\PY{l+m+mi}{2}\PY{p}{)}
    \PY{n}{indices} \PY{o}{=} \PY{n}{np}\PY{o}{.}\PY{n}{random}\PY{o}{.}\PY{n}{permutation}\PY{p}{(}\PY{n+nb}{len}\PY{p}{(}\PY{n}{data}\PY{p}{)}\PY{p}{)}  \PY{c+c1}{\PYZsh{} 随机打乱整个数据集的索引}
    \PY{n}{test\PYZus{}set\PYZus{}size} \PY{o}{=} \PY{n+nb}{int}\PY{p}{(}\PY{n+nb}{len}\PY{p}{(}\PY{n}{data}\PY{p}{)} \PY{o}{*} \PY{n}{test\PYZus{}radio}\PY{p}{)}  \PY{c+c1}{\PYZsh{} 测试集和数据集分割点的索引}
    \PY{n}{test\PYZus{}indices} \PY{o}{=} \PY{n}{indices}\PY{p}{[} \PY{p}{:} \PY{n}{test\PYZus{}set\PYZus{}size}\PY{p}{]}  \PY{c+c1}{\PYZsh{} 测试集索引}
    \PY{n}{train\PYZus{}indices} \PY{o}{=} \PY{n}{indices}\PY{p}{[}\PY{n}{test\PYZus{}set\PYZus{}size} \PY{p}{:} \PY{p}{]}  \PY{c+c1}{\PYZsh{} 训练集索引}
    \PY{k}{return} \PY{n}{data}\PY{o}{.}\PY{n}{iloc}\PY{p}{[}\PY{n}{train\PYZus{}indices}\PY{p}{]}\PY{p}{,} \PY{n}{data}\PY{o}{.}\PY{n}{iloc}\PY{p}{[}\PY{n}{test\PYZus{}indices}\PY{p}{]}
\end{Verbatim}
\end{tcolorbox}

    \begin{tcolorbox}[breakable, size=fbox, boxrule=1pt, pad at break*=1mm,colback=cellbackground, colframe=cellborder]
\prompt{In}{incolor}{9}{\boxspacing}
\begin{Verbatim}[commandchars=\\\{\}]
\PY{n}{train\PYZus{}set}\PY{p}{,} \PY{n}{test\PYZus{}set} \PY{o}{=} \PY{n}{split\PYZus{}train\PYZus{}test}\PY{p}{(}\PY{n}{data\PYZus{}housing}\PY{p}{,} \PY{l+m+mf}{0.2}\PY{p}{)}  \PY{c+c1}{\PYZsh{} 传入数据和比例}
\PY{n+nb}{print}\PY{p}{(}\PY{n+nb}{len}\PY{p}{(}\PY{n}{train\PYZus{}set}\PY{p}{)}\PY{p}{)}
\PY{n+nb}{print}\PY{p}{(}\PY{n+nb}{len}\PY{p}{(}\PY{n}{test\PYZus{}set}\PY{p}{)}\PY{p}{)}
\end{Verbatim}
\end{tcolorbox}

    \begin{Verbatim}[commandchars=\\\{\}]
16512
4128
    \end{Verbatim}

    \hypertarget{ux6807ux8bc6ux6570ux636e}{%
\subsection{3.标识数据}\label{ux6807ux8bc6ux6570ux636e}}

\begin{itemize}
\tightlist
\item
  为了避免新增数据的时候,同一条数据被反复划分,出现时而在测试集时而在训练集的情况,需要对数据进行标识
\item
  使用行索引进行标识
\end{itemize}

    \begin{tcolorbox}[breakable, size=fbox, boxrule=1pt, pad at break*=1mm,colback=cellbackground, colframe=cellborder]
\prompt{In}{incolor}{10}{\boxspacing}
\begin{Verbatim}[commandchars=\\\{\}]
\PY{k+kn}{import} \PY{n+nn}{hashlib}
\PY{c+c1}{\PYZsh{} 唯一hash标识的判断,返回True\PYZbs{}False,对于测试集数据返回True,训练集返回False}
\PY{k}{def} \PY{n+nf}{test\PYZus{}set\PYZus{}check}\PY{p}{(}\PY{n}{identifier}\PY{p}{,} \PY{n}{test\PYZus{}radio}\PY{p}{,} \PY{n+nb}{hash} \PY{o}{=} \PY{n}{hashlib}\PY{o}{.}\PY{n}{md5}\PY{p}{)}\PY{p}{:}
    \PY{c+c1}{\PYZsh{} 将identifier类型转为np.int64}
    \PY{c+c1}{\PYZsh{} digest()[\PYZhy{}1]取hash加密后最后一个字节}
    \PY{c+c1}{\PYZsh{} 返回摘要,作为二进制数据的字符串}
    \PY{k}{return} \PY{n+nb}{hash}\PY{p}{(}\PY{n}{np}\PY{o}{.}\PY{n}{int64}\PY{p}{(}\PY{n}{identifier}\PY{p}{)}\PY{p}{)}\PY{o}{.}\PY{n}{digest}\PY{p}{(}\PY{p}{)}\PY{p}{[}\PY{o}{\PYZhy{}}\PY{l+m+mi}{1}\PY{p}{]} \PY{o}{\PYZlt{}} \PY{l+m+mi}{256} \PY{o}{*} \PY{n}{test\PYZus{}radio}
\end{Verbatim}
\end{tcolorbox}

    \begin{tcolorbox}[breakable, size=fbox, boxrule=1pt, pad at break*=1mm,colback=cellbackground, colframe=cellborder]
\prompt{In}{incolor}{11}{\boxspacing}
\begin{Verbatim}[commandchars=\\\{\}]
\PY{c+c1}{\PYZsh{} id\PYZus{}column:data中作为标识的一列}
\PY{k}{def} \PY{n+nf}{split\PYZus{}train\PYZus{}test\PYZus{}by\PYZus{}id}\PY{p}{(}\PY{n}{data}\PY{p}{,} \PY{n}{test\PYZus{}radio}\PY{p}{,} \PY{n}{id\PYZus{}column}\PY{p}{)}\PY{p}{:}
    \PY{n}{ids} \PY{o}{=} \PY{n}{data}\PY{p}{[}\PY{n}{id\PYZus{}column}\PY{p}{]}
    \PY{c+c1}{\PYZsh{} 判断数据是否在测试集中,apply对每个数据都会应用test\PYZus{}set\PYZus{}check检查数据}
    \PY{n}{isin\PYZus{}test\PYZus{}set} \PY{o}{=} \PY{n}{ids}\PY{o}{.}\PY{n}{apply}\PY{p}{(}\PY{k}{lambda} \PY{n}{id\PYZus{}}\PY{p}{:}\PY{n}{test\PYZus{}set\PYZus{}check}\PY{p}{(}\PY{n}{id\PYZus{}}\PY{p}{,} \PY{n}{test\PYZus{}radio}\PY{p}{)}\PY{p}{)}
    \PY{c+c1}{\PYZsh{} \PYZhy{}负号代表不在测试集中}
    \PY{k}{return} \PY{n}{data}\PY{o}{.}\PY{n}{loc}\PY{p}{[}\PY{o}{\PYZhy{}}\PY{n}{isin\PYZus{}test\PYZus{}set}\PY{p}{]}\PY{p}{,} \PY{n}{data}\PY{o}{.}\PY{n}{loc}\PY{p}{[}\PY{n}{isin\PYZus{}test\PYZus{}set}\PY{p}{]}
\end{Verbatim}
\end{tcolorbox}

    \begin{tcolorbox}[breakable, size=fbox, boxrule=1pt, pad at break*=1mm,colback=cellbackground, colframe=cellborder]
\prompt{In}{incolor}{12}{\boxspacing}
\begin{Verbatim}[commandchars=\\\{\}]
\PY{n}{data\PYZus{}housing\PYZus{}with\PYZus{}id} \PY{o}{=} \PY{n}{data\PYZus{}housing}\PY{o}{.}\PY{n}{reset\PYZus{}index}\PY{p}{(}\PY{p}{)}  \PY{c+c1}{\PYZsh{} 使用行索引作为标识符 ID}
\PY{n}{data\PYZus{}housing\PYZus{}with\PYZus{}id}\PY{o}{.}\PY{n}{head}\PY{p}{(}\PY{p}{)}
\end{Verbatim}
\end{tcolorbox}

            \begin{tcolorbox}[breakable, size=fbox, boxrule=.5pt, pad at break*=1mm, opacityfill=0]
\prompt{Out}{outcolor}{12}{\boxspacing}
\begin{Verbatim}[commandchars=\\\{\}]
   index  longitude  latitude  housing\_median\_age  total\_rooms  \textbackslash{}
0      0    -122.23     37.88                41.0        880.0
1      1    -122.22     37.86                21.0       7099.0
2      2    -122.24     37.85                52.0       1467.0
3      3    -122.25     37.85                52.0       1274.0
4      4    -122.25     37.85                52.0       1627.0

   total\_bedrooms  population  households  median\_income  median\_house\_value  \textbackslash{}
0           129.0       322.0       126.0         8.3252            452600.0
1          1106.0      2401.0      1138.0         8.3014            358500.0
2           190.0       496.0       177.0         7.2574            352100.0
3           235.0       558.0       219.0         5.6431            341300.0
4           280.0       565.0       259.0         3.8462            342200.0

  ocean\_proximity
0        NEAR BAY
1        NEAR BAY
2        NEAR BAY
3        NEAR BAY
4        NEAR BAY
\end{Verbatim}
\end{tcolorbox}
        
    \begin{tcolorbox}[breakable, size=fbox, boxrule=1pt, pad at break*=1mm,colback=cellbackground, colframe=cellborder]
\prompt{In}{incolor}{13}{\boxspacing}
\begin{Verbatim}[commandchars=\\\{\}]
\PY{c+c1}{\PYZsh{} 分割数据}
\PY{n}{train\PYZus{}set}\PY{p}{,} \PY{n}{test\PYZus{}set} \PY{o}{=} \PY{n}{split\PYZus{}train\PYZus{}test\PYZus{}by\PYZus{}id}\PY{p}{(}\PY{n}{data\PYZus{}housing\PYZus{}with\PYZus{}id}\PY{p}{,} \PY{l+m+mf}{0.2}\PY{p}{,} \PY{l+s+s1}{\PYZsq{}}\PY{l+s+s1}{index}\PY{l+s+s1}{\PYZsq{}}\PY{p}{)}
\end{Verbatim}
\end{tcolorbox}

    \begin{tcolorbox}[breakable, size=fbox, boxrule=1pt, pad at break*=1mm,colback=cellbackground, colframe=cellborder]
\prompt{In}{incolor}{14}{\boxspacing}
\begin{Verbatim}[commandchars=\\\{\}]
\PY{n}{train\PYZus{}set}\PY{o}{.}\PY{n}{head}\PY{p}{(}\PY{p}{)}
\end{Verbatim}
\end{tcolorbox}

            \begin{tcolorbox}[breakable, size=fbox, boxrule=.5pt, pad at break*=1mm, opacityfill=0]
\prompt{Out}{outcolor}{14}{\boxspacing}
\begin{Verbatim}[commandchars=\\\{\}]
   index  longitude  latitude  housing\_median\_age  total\_rooms  \textbackslash{}
0      0    -122.23     37.88                41.0        880.0
1      1    -122.22     37.86                21.0       7099.0
2      2    -122.24     37.85                52.0       1467.0
3      3    -122.25     37.85                52.0       1274.0
6      6    -122.25     37.84                52.0       2535.0

   total\_bedrooms  population  households  median\_income  median\_house\_value  \textbackslash{}
0           129.0       322.0       126.0         8.3252            452600.0
1          1106.0      2401.0      1138.0         8.3014            358500.0
2           190.0       496.0       177.0         7.2574            352100.0
3           235.0       558.0       219.0         5.6431            341300.0
6           489.0      1094.0       514.0         3.6591            299200.0

  ocean\_proximity
0        NEAR BAY
1        NEAR BAY
2        NEAR BAY
3        NEAR BAY
6        NEAR BAY
\end{Verbatim}
\end{tcolorbox}
        
    \begin{tcolorbox}[breakable, size=fbox, boxrule=1pt, pad at break*=1mm,colback=cellbackground, colframe=cellborder]
\prompt{In}{incolor}{15}{\boxspacing}
\begin{Verbatim}[commandchars=\\\{\}]
\PY{n}{train\PYZus{}set}\PY{o}{.}\PY{n}{tail}\PY{p}{(}\PY{p}{)}
\end{Verbatim}
\end{tcolorbox}

            \begin{tcolorbox}[breakable, size=fbox, boxrule=.5pt, pad at break*=1mm, opacityfill=0]
\prompt{Out}{outcolor}{15}{\boxspacing}
\begin{Verbatim}[commandchars=\\\{\}]
       index  longitude  latitude  housing\_median\_age  total\_rooms  \textbackslash{}
20634  20634    -121.56     39.27                28.0       2332.0
20635  20635    -121.09     39.48                25.0       1665.0
20636  20636    -121.21     39.49                18.0        697.0
20638  20638    -121.32     39.43                18.0       1860.0
20639  20639    -121.24     39.37                16.0       2785.0

       total\_bedrooms  population  households  median\_income  \textbackslash{}
20634           395.0      1041.0       344.0         3.7125
20635           374.0       845.0       330.0         1.5603
20636           150.0       356.0       114.0         2.5568
20638           409.0       741.0       349.0         1.8672
20639           616.0      1387.0       530.0         2.3886

       median\_house\_value ocean\_proximity
20634            116800.0          INLAND
20635             78100.0          INLAND
20636             77100.0          INLAND
20638             84700.0          INLAND
20639             89400.0          INLAND
\end{Verbatim}
\end{tcolorbox}
        
    \hypertarget{ux521bux5efaux7279ux5f81ux6570ux636eux5217}{%
\subsection{4.创建特征数据列}\label{ux521bux5efaux7279ux5f81ux6570ux636eux5217}}

\begin{itemize}
\tightlist
\item
  基于行索引作为标识,不能删除和中间插入数据,只能末尾插入,否则行索引会变
\item
  寻找稳定特征来创建唯一标识符

  \begin{enumerate}
  \def\labelenumi{\arabic{enumi}.}
  \tightlist
  \item
    创建一组特征数据
  \end{enumerate}
\end{itemize}

    \begin{tcolorbox}[breakable, size=fbox, boxrule=1pt, pad at break*=1mm,colback=cellbackground, colframe=cellborder]
\prompt{In}{incolor}{16}{\boxspacing}
\begin{Verbatim}[commandchars=\\\{\}]
\PY{c+c1}{\PYZsh{} 使用经纬度创建一组特征数据 \PYZsq{}id\PYZsq{}}
\PY{n}{data\PYZus{}housing\PYZus{}with\PYZus{}id}\PY{p}{[}\PY{l+s+s1}{\PYZsq{}}\PY{l+s+s1}{id}\PY{l+s+s1}{\PYZsq{}}\PY{p}{]} \PY{o}{=} \PY{n}{data\PYZus{}housing}\PY{p}{[}\PY{l+s+s1}{\PYZsq{}}\PY{l+s+s1}{longitude}\PY{l+s+s1}{\PYZsq{}}\PY{p}{]} \PY{o}{*} \PY{l+m+mi}{1000} \PY{o}{+} \PY{n}{data\PYZus{}housing}\PY{p}{[}\PY{l+s+s1}{\PYZsq{}}\PY{l+s+s1}{latitude}\PY{l+s+s1}{\PYZsq{}}\PY{p}{]}
\PY{n}{data\PYZus{}housing\PYZus{}with\PYZus{}id}\PY{o}{.}\PY{n}{head}\PY{p}{(}\PY{p}{)}
\end{Verbatim}
\end{tcolorbox}

            \begin{tcolorbox}[breakable, size=fbox, boxrule=.5pt, pad at break*=1mm, opacityfill=0]
\prompt{Out}{outcolor}{16}{\boxspacing}
\begin{Verbatim}[commandchars=\\\{\}]
   index  longitude  latitude  housing\_median\_age  total\_rooms  \textbackslash{}
0      0    -122.23     37.88                41.0        880.0
1      1    -122.22     37.86                21.0       7099.0
2      2    -122.24     37.85                52.0       1467.0
3      3    -122.25     37.85                52.0       1274.0
4      4    -122.25     37.85                52.0       1627.0

   total\_bedrooms  population  households  median\_income  median\_house\_value  \textbackslash{}
0           129.0       322.0       126.0         8.3252            452600.0
1          1106.0      2401.0      1138.0         8.3014            358500.0
2           190.0       496.0       177.0         7.2574            352100.0
3           235.0       558.0       219.0         5.6431            341300.0
4           280.0       565.0       259.0         3.8462            342200.0

  ocean\_proximity         id
0        NEAR BAY -122192.12
1        NEAR BAY -122182.14
2        NEAR BAY -122202.15
3        NEAR BAY -122212.15
4        NEAR BAY -122212.15
\end{Verbatim}
\end{tcolorbox}
        
    \begin{tcolorbox}[breakable, size=fbox, boxrule=1pt, pad at break*=1mm,colback=cellbackground, colframe=cellborder]
\prompt{In}{incolor}{17}{\boxspacing}
\begin{Verbatim}[commandchars=\\\{\}]
\PY{c+c1}{\PYZsh{} 通过特征id标识,分割数据}
\PY{n}{train\PYZus{}set}\PY{p}{,} \PY{n}{test\PYZus{}set} \PY{o}{=} \PY{n}{split\PYZus{}train\PYZus{}test\PYZus{}by\PYZus{}id}\PY{p}{(}\PY{n}{data\PYZus{}housing\PYZus{}with\PYZus{}id}\PY{p}{,} \PY{l+m+mf}{0.2}\PY{p}{,} \PY{l+s+s1}{\PYZsq{}}\PY{l+s+s1}{id}\PY{l+s+s1}{\PYZsq{}}\PY{p}{)}
\end{Verbatim}
\end{tcolorbox}

    \begin{tcolorbox}[breakable, size=fbox, boxrule=1pt, pad at break*=1mm,colback=cellbackground, colframe=cellborder]
\prompt{In}{incolor}{18}{\boxspacing}
\begin{Verbatim}[commandchars=\\\{\}]
\PY{n}{train\PYZus{}set}\PY{o}{.}\PY{n}{head}\PY{p}{(}\PY{p}{)}
\end{Verbatim}
\end{tcolorbox}

            \begin{tcolorbox}[breakable, size=fbox, boxrule=.5pt, pad at break*=1mm, opacityfill=0]
\prompt{Out}{outcolor}{18}{\boxspacing}
\begin{Verbatim}[commandchars=\\\{\}]
   index  longitude  latitude  housing\_median\_age  total\_rooms  \textbackslash{}
0      0    -122.23     37.88                41.0        880.0
1      1    -122.22     37.86                21.0       7099.0
2      2    -122.24     37.85                52.0       1467.0
3      3    -122.25     37.85                52.0       1274.0
4      4    -122.25     37.85                52.0       1627.0

   total\_bedrooms  population  households  median\_income  median\_house\_value  \textbackslash{}
0           129.0       322.0       126.0         8.3252            452600.0
1          1106.0      2401.0      1138.0         8.3014            358500.0
2           190.0       496.0       177.0         7.2574            352100.0
3           235.0       558.0       219.0         5.6431            341300.0
4           280.0       565.0       259.0         3.8462            342200.0

  ocean\_proximity         id
0        NEAR BAY -122192.12
1        NEAR BAY -122182.14
2        NEAR BAY -122202.15
3        NEAR BAY -122212.15
4        NEAR BAY -122212.15
\end{Verbatim}
\end{tcolorbox}
        
    \begin{tcolorbox}[breakable, size=fbox, boxrule=1pt, pad at break*=1mm,colback=cellbackground, colframe=cellborder]
\prompt{In}{incolor}{19}{\boxspacing}
\begin{Verbatim}[commandchars=\\\{\}]
\PY{n}{train\PYZus{}set}\PY{o}{.}\PY{n}{tail}\PY{p}{(}\PY{l+m+mi}{100}\PY{p}{)}
\end{Verbatim}
\end{tcolorbox}

            \begin{tcolorbox}[breakable, size=fbox, boxrule=.5pt, pad at break*=1mm, opacityfill=0]
\prompt{Out}{outcolor}{19}{\boxspacing}
\begin{Verbatim}[commandchars=\\\{\}]
       index  longitude  latitude  housing\_median\_age  total\_rooms  \textbackslash{}
20519  20519    -121.53     38.58                33.0       4988.0
20520  20520    -121.53     38.58                35.0       1316.0
20521  20521    -121.53     38.57                34.0       3395.0
20522  20522    -121.54     38.54                36.0       1672.0
20525  20525    -121.56     38.44                43.0       1485.0
{\ldots}      {\ldots}        {\ldots}       {\ldots}                 {\ldots}          {\ldots}
20634  20634    -121.56     39.27                28.0       2332.0
20635  20635    -121.09     39.48                25.0       1665.0
20637  20637    -121.22     39.43                17.0       2254.0
20638  20638    -121.32     39.43                18.0       1860.0
20639  20639    -121.24     39.37                16.0       2785.0

       total\_bedrooms  population  households  median\_income  \textbackslash{}
20519          1169.0      2414.0      1075.0         1.9728
20520           321.0       732.0       336.0         2.1213
20521           592.0      1518.0       627.0         4.0833
20522           302.0       969.0       337.0         3.0536
20525           270.0       653.0       251.0         3.0000
{\ldots}               {\ldots}         {\ldots}         {\ldots}            {\ldots}
20634           395.0      1041.0       344.0         3.7125
20635           374.0       845.0       330.0         1.5603
20637           485.0      1007.0       433.0         1.7000
20638           409.0       741.0       349.0         1.8672
20639           616.0      1387.0       530.0         2.3886

       median\_house\_value ocean\_proximity         id
20519             76400.0          INLAND -121491.42
20520             79200.0          INLAND -121491.42
20521            118500.0          INLAND -121491.43
20522             73100.0          INLAND -121501.46
20525            141700.0          INLAND -121521.56
{\ldots}                   {\ldots}             {\ldots}        {\ldots}
20634            116800.0          INLAND -121520.73
20635             78100.0          INLAND -121050.52
20637             92300.0          INLAND -121180.57
20638             84700.0          INLAND -121280.57
20639             89400.0          INLAND -121200.63

[100 rows x 12 columns]
\end{Verbatim}
\end{tcolorbox}
        
    \hypertarget{ux901aux8fc7sklearnux5206ux5272ux6570ux636e}{%
\subsection{5.通过sklearn分割数据}\label{ux901aux8fc7sklearnux5206ux5272ux6570ux636e}}

\begin{itemize}
\tightlist
\item
  sklearn只适用于数据不变的数据集
\end{itemize}

    \begin{tcolorbox}[breakable, size=fbox, boxrule=1pt, pad at break*=1mm,colback=cellbackground, colframe=cellborder]
\prompt{In}{incolor}{20}{\boxspacing}
\begin{Verbatim}[commandchars=\\\{\}]
\PY{k+kn}{from} \PY{n+nn}{sklearn}\PY{n+nn}{.}\PY{n+nn}{model\PYZus{}selection} \PY{k+kn}{import} \PY{n}{train\PYZus{}test\PYZus{}split}
\PY{c+c1}{\PYZsh{} test\PYZus{}size:分割比例  random\PYZus{}state:相当于随机种子号}
\PY{n}{train\PYZus{}set}\PY{p}{,} \PY{n}{test\PYZus{}set} \PY{o}{=} \PY{n}{train\PYZus{}test\PYZus{}split}\PY{p}{(}\PY{n}{data\PYZus{}housing}\PY{p}{,} \PY{n}{test\PYZus{}size} \PY{o}{=} \PY{l+m+mf}{0.2}\PY{p}{,} \PY{n}{random\PYZus{}state} \PY{o}{=} \PY{l+m+mi}{42}\PY{p}{)}
\PY{n}{train\PYZus{}set}\PY{o}{.}\PY{n}{head}\PY{p}{(}\PY{p}{)}
\end{Verbatim}
\end{tcolorbox}

            \begin{tcolorbox}[breakable, size=fbox, boxrule=.5pt, pad at break*=1mm, opacityfill=0]
\prompt{Out}{outcolor}{20}{\boxspacing}
\begin{Verbatim}[commandchars=\\\{\}]
       longitude  latitude  housing\_median\_age  total\_rooms  total\_bedrooms  \textbackslash{}
14196    -117.03     32.71                33.0       3126.0           627.0
8267     -118.16     33.77                49.0       3382.0           787.0
17445    -120.48     34.66                 4.0       1897.0           331.0
14265    -117.11     32.69                36.0       1421.0           367.0
2271     -119.80     36.78                43.0       2382.0           431.0

       population  households  median\_income  median\_house\_value  \textbackslash{}
14196      2300.0       623.0         3.2596            103000.0
8267       1314.0       756.0         3.8125            382100.0
17445       915.0       336.0         4.1563            172600.0
14265      1418.0       355.0         1.9425             93400.0
2271        874.0       380.0         3.5542             96500.0

      ocean\_proximity
14196      NEAR OCEAN
8267       NEAR OCEAN
17445      NEAR OCEAN
14265      NEAR OCEAN
2271           INLAND
\end{Verbatim}
\end{tcolorbox}
        
    \hypertarget{ux4eceux6570ux636eux53efux89c6ux5316ux4e2dux63a2ux7d22ux6570ux636e}{%
\subsection{从数据可视化中探索数据}\label{ux4eceux6570ux636eux53efux89c6ux5316ux4e2dux63a2ux7d22ux6570ux636e}}

\#\#\# 分层采样 1. 创建一个收入类别属性 2. 每一层数据量适中 3.
收入中位数除以1.5,限制收入类别数量,使用ceil取整,得到离散类别,将大于5的列合并为类别5

    \begin{tcolorbox}[breakable, size=fbox, boxrule=1pt, pad at break*=1mm,colback=cellbackground, colframe=cellborder]
\prompt{In}{incolor}{21}{\boxspacing}
\begin{Verbatim}[commandchars=\\\{\}]
\PY{n}{data\PYZus{}housing}\PY{p}{[}\PY{l+s+s1}{\PYZsq{}}\PY{l+s+s1}{income\PYZus{}cat}\PY{l+s+s1}{\PYZsq{}}\PY{p}{]} \PY{o}{=} \PY{n}{np}\PY{o}{.}\PY{n}{ceil}\PY{p}{(}\PY{n}{data\PYZus{}housing}\PY{p}{[}\PY{l+s+s1}{\PYZsq{}}\PY{l+s+s1}{median\PYZus{}income}\PY{l+s+s1}{\PYZsq{}}\PY{p}{]} \PY{o}{/} \PY{l+m+mf}{1.5}\PY{p}{)}
\PY{n}{data\PYZus{}housing}\PY{p}{[}\PY{l+s+s1}{\PYZsq{}}\PY{l+s+s1}{income\PYZus{}cat}\PY{l+s+s1}{\PYZsq{}}\PY{p}{]}\PY{o}{.}\PY{n}{head}\PY{p}{(}\PY{l+m+mi}{20}\PY{p}{)}
\end{Verbatim}
\end{tcolorbox}

            \begin{tcolorbox}[breakable, size=fbox, boxrule=.5pt, pad at break*=1mm, opacityfill=0]
\prompt{Out}{outcolor}{21}{\boxspacing}
\begin{Verbatim}[commandchars=\\\{\}]
0     6.0
1     6.0
2     5.0
3     4.0
4     3.0
5     3.0
6     3.0
7     3.0
8     2.0
9     3.0
10    3.0
11    3.0
12    3.0
13    2.0
14    2.0
15    2.0
16    2.0
17    2.0
18    2.0
19    2.0
Name: income\_cat, dtype: float64
\end{Verbatim}
\end{tcolorbox}
        
    \begin{tcolorbox}[breakable, size=fbox, boxrule=1pt, pad at break*=1mm,colback=cellbackground, colframe=cellborder]
\prompt{In}{incolor}{22}{\boxspacing}
\begin{Verbatim}[commandchars=\\\{\}]
\PY{c+c1}{\PYZsh{} 1.使用where将大于5的值替换成5}
\PY{n}{data\PYZus{}housing}\PY{p}{[}\PY{l+s+s1}{\PYZsq{}}\PY{l+s+s1}{income\PYZus{}cat}\PY{l+s+s1}{\PYZsq{}}\PY{p}{]}\PY{o}{.}\PY{n}{where}\PY{p}{(}\PY{n}{data\PYZus{}housing}\PY{p}{[}\PY{l+s+s1}{\PYZsq{}}\PY{l+s+s1}{income\PYZus{}cat}\PY{l+s+s1}{\PYZsq{}}\PY{p}{]} \PY{o}{\PYZlt{}} \PY{l+m+mi}{5}\PY{p}{,} \PY{l+m+mf}{5.0}\PY{p}{,} \PY{n}{inplace} \PY{o}{=} \PY{k+kc}{True}\PY{p}{)}
\PY{n}{data\PYZus{}housing}\PY{p}{[}\PY{l+s+s1}{\PYZsq{}}\PY{l+s+s1}{income\PYZus{}cat}\PY{l+s+s1}{\PYZsq{}}\PY{p}{]}\PY{o}{.}\PY{n}{head}\PY{p}{(}\PY{l+m+mi}{20}\PY{p}{)}
\end{Verbatim}
\end{tcolorbox}

            \begin{tcolorbox}[breakable, size=fbox, boxrule=.5pt, pad at break*=1mm, opacityfill=0]
\prompt{Out}{outcolor}{22}{\boxspacing}
\begin{Verbatim}[commandchars=\\\{\}]
0     5.0
1     5.0
2     5.0
3     4.0
4     3.0
5     3.0
6     3.0
7     3.0
8     2.0
9     3.0
10    3.0
11    3.0
12    3.0
13    2.0
14    2.0
15    2.0
16    2.0
17    2.0
18    2.0
19    2.0
Name: income\_cat, dtype: float64
\end{Verbatim}
\end{tcolorbox}
        
    \begin{tcolorbox}[breakable, size=fbox, boxrule=1pt, pad at break*=1mm,colback=cellbackground, colframe=cellborder]
\prompt{In}{incolor}{23}{\boxspacing}
\begin{Verbatim}[commandchars=\\\{\}]
\PY{c+c1}{\PYZsh{} 2.把连续值转换成类别标签}
\PY{n}{data\PYZus{}housing}\PY{p}{[}\PY{l+s+s1}{\PYZsq{}}\PY{l+s+s1}{income\PYZus{}cat}\PY{l+s+s1}{\PYZsq{}}\PY{p}{]} \PY{o}{=} \PY{n}{pd}\PY{o}{.}\PY{n}{cut}\PY{p}{(}\PY{n}{data\PYZus{}housing}\PY{p}{[}\PY{l+s+s1}{\PYZsq{}}\PY{l+s+s1}{median\PYZus{}income}\PY{l+s+s1}{\PYZsq{}}\PY{p}{]}\PY{p}{,} \PY{n}{bins} \PY{o}{=} \PY{p}{[}\PY{l+m+mf}{0.0}\PY{p}{,} \PY{l+m+mf}{1.5}\PY{p}{,} \PY{l+m+mf}{3.0}\PY{p}{,} \PY{l+m+mf}{4.5}\PY{p}{,} \PY{l+m+mf}{6.0}\PY{p}{,} \PY{n}{np}\PY{o}{.}\PY{n}{inf}\PY{p}{]}\PY{p}{,} \PY{n}{labels} \PY{o}{=} \PY{p}{[}\PY{l+m+mi}{1}\PY{p}{,} \PY{l+m+mi}{2}\PY{p}{,} \PY{l+m+mi}{3}\PY{p}{,} \PY{l+m+mi}{4}\PY{p}{,} \PY{l+m+mi}{5}\PY{p}{]}\PY{p}{)}
\PY{n}{data\PYZus{}housing}\PY{p}{[}\PY{l+s+s1}{\PYZsq{}}\PY{l+s+s1}{income\PYZus{}cat}\PY{l+s+s1}{\PYZsq{}}\PY{p}{]}\PY{o}{.}\PY{n}{head}\PY{p}{(}\PY{l+m+mi}{20}\PY{p}{)}
\end{Verbatim}
\end{tcolorbox}

            \begin{tcolorbox}[breakable, size=fbox, boxrule=.5pt, pad at break*=1mm, opacityfill=0]
\prompt{Out}{outcolor}{23}{\boxspacing}
\begin{Verbatim}[commandchars=\\\{\}]
0     5
1     5
2     5
3     4
4     3
5     3
6     3
7     3
8     2
9     3
10    3
11    3
12    3
13    2
14    2
15    2
16    2
17    2
18    2
19    2
Name: income\_cat, dtype: category
Categories (5, int64): [1 < 2 < 3 < 4 < 5]
\end{Verbatim}
\end{tcolorbox}
        
    \begin{tcolorbox}[breakable, size=fbox, boxrule=1pt, pad at break*=1mm,colback=cellbackground, colframe=cellborder]
\prompt{In}{incolor}{24}{\boxspacing}
\begin{Verbatim}[commandchars=\\\{\}]
\PY{c+c1}{\PYZsh{} 统计不同类别有多少}
\PY{n}{data\PYZus{}housing}\PY{p}{[}\PY{l+s+s1}{\PYZsq{}}\PY{l+s+s1}{income\PYZus{}cat}\PY{l+s+s1}{\PYZsq{}}\PY{p}{]}\PY{o}{.}\PY{n}{hist}\PY{p}{(}\PY{p}{)}
\end{Verbatim}
\end{tcolorbox}

            \begin{tcolorbox}[breakable, size=fbox, boxrule=.5pt, pad at break*=1mm, opacityfill=0]
\prompt{Out}{outcolor}{24}{\boxspacing}
\begin{Verbatim}[commandchars=\\\{\}]
<matplotlib.axes.\_subplots.AxesSubplot at 0x23bff0f9ac8>
\end{Verbatim}
\end{tcolorbox}
        
    \begin{center}
    \adjustimage{max size={0.9\linewidth}{0.9\paperheight}}{特征工程_files/特征工程_30_1.png}
    \end{center}
    { \hspace*{\fill} \\}
    
    \hypertarget{sklearnux5206ux5c42}{%
\subsection{sklearn分层}\label{sklearnux5206ux5c42}}

    \begin{tcolorbox}[breakable, size=fbox, boxrule=1pt, pad at break*=1mm,colback=cellbackground, colframe=cellborder]
\prompt{In}{incolor}{25}{\boxspacing}
\begin{Verbatim}[commandchars=\\\{\}]
\PY{c+c1}{\PYZsh{} sklearn分层}
\PY{k+kn}{from} \PY{n+nn}{sklearn}\PY{n+nn}{.}\PY{n+nn}{model\PYZus{}selection} \PY{k+kn}{import} \PY{n}{StratifiedShuffleSplit}
\end{Verbatim}
\end{tcolorbox}

    \begin{tcolorbox}[breakable, size=fbox, boxrule=1pt, pad at break*=1mm,colback=cellbackground, colframe=cellborder]
\prompt{In}{incolor}{26}{\boxspacing}
\begin{Verbatim}[commandchars=\\\{\}]
\PY{c+c1}{\PYZsh{} 创建一个split对象。n\PYZus{}splits=1将数据分1份2:8,默认n\PYZus{}splits=10将数据分10份2:8}
\PY{n}{split} \PY{o}{=} \PY{n}{StratifiedShuffleSplit}\PY{p}{(}\PY{n}{n\PYZus{}splits}\PY{o}{=}\PY{l+m+mi}{2}\PY{p}{,} \PY{n}{test\PYZus{}size} \PY{o}{=} \PY{l+m+mf}{0.2}\PY{p}{,} \PY{n}{random\PYZus{}state} \PY{o}{=} \PY{l+m+mi}{42}\PY{p}{)}
\end{Verbatim}
\end{tcolorbox}

    \begin{tcolorbox}[breakable, size=fbox, boxrule=1pt, pad at break*=1mm,colback=cellbackground, colframe=cellborder]
\prompt{In}{incolor}{27}{\boxspacing}
\begin{Verbatim}[commandchars=\\\{\}]
\PY{c+c1}{\PYZsh{} split(data, column) 以column分层, n\PYZus{}splits为n,循环进行n次}
\PY{k}{for} \PY{n}{train\PYZus{}index}\PY{p}{,} \PY{n}{test\PYZus{}index} \PY{o+ow}{in} \PY{n}{split}\PY{o}{.}\PY{n}{split}\PY{p}{(}\PY{n}{data\PYZus{}housing}\PY{p}{,} \PY{n}{data\PYZus{}housing}\PY{p}{[}\PY{l+s+s1}{\PYZsq{}}\PY{l+s+s1}{income\PYZus{}cat}\PY{l+s+s1}{\PYZsq{}}\PY{p}{]}\PY{p}{)}\PY{p}{:}
    \PY{n}{strat\PYZus{}train\PYZus{}set} \PY{o}{=} \PY{n}{data\PYZus{}housing}\PY{o}{.}\PY{n}{loc}\PY{p}{[}\PY{n}{train\PYZus{}index}\PY{p}{]}
    \PY{n}{strat\PYZus{}test\PYZus{}set} \PY{o}{=} \PY{n}{data\PYZus{}housing}\PY{o}{.}\PY{n}{loc}\PY{p}{[}\PY{n}{test\PYZus{}index}\PY{p}{]}
    \PY{n+nb}{print}\PY{p}{(}\PY{n}{strat\PYZus{}train\PYZus{}set}\PY{o}{.}\PY{n}{shape}\PY{p}{,} \PY{l+s+s2}{\PYZdq{}}\PY{l+s+s2}{ }\PY{l+s+s2}{\PYZdq{}}\PY{p}{,} \PY{n}{strat\PYZus{}test\PYZus{}set}\PY{o}{.}\PY{n}{shape}\PY{p}{)}
\end{Verbatim}
\end{tcolorbox}

    \begin{Verbatim}[commandchars=\\\{\}]
(16512, 11)   (4128, 11)
(16512, 11)   (4128, 11)
    \end{Verbatim}

    \begin{tcolorbox}[breakable, size=fbox, boxrule=1pt, pad at break*=1mm,colback=cellbackground, colframe=cellborder]
\prompt{In}{incolor}{28}{\boxspacing}
\begin{Verbatim}[commandchars=\\\{\}]
\PY{n}{strat\PYZus{}train\PYZus{}set}\PY{o}{.}\PY{n}{shape}
\end{Verbatim}
\end{tcolorbox}

            \begin{tcolorbox}[breakable, size=fbox, boxrule=.5pt, pad at break*=1mm, opacityfill=0]
\prompt{Out}{outcolor}{28}{\boxspacing}
\begin{Verbatim}[commandchars=\\\{\}]
(16512, 11)
\end{Verbatim}
\end{tcolorbox}
        
    \begin{tcolorbox}[breakable, size=fbox, boxrule=1pt, pad at break*=1mm,colback=cellbackground, colframe=cellborder]
\prompt{In}{incolor}{29}{\boxspacing}
\begin{Verbatim}[commandchars=\\\{\}]
\PY{n}{strat\PYZus{}train\PYZus{}set}\PY{o}{.}\PY{n}{head}\PY{p}{(}\PY{l+m+mi}{20}\PY{p}{)}
\end{Verbatim}
\end{tcolorbox}

            \begin{tcolorbox}[breakable, size=fbox, boxrule=.5pt, pad at break*=1mm, opacityfill=0]
\prompt{Out}{outcolor}{29}{\boxspacing}
\begin{Verbatim}[commandchars=\\\{\}]
       longitude  latitude  housing\_median\_age  total\_rooms  total\_bedrooms  \textbackslash{}
5288     -118.47     34.05                27.0       4401.0          1033.0
12865    -121.34     38.69                17.0       1968.0           364.0
9174     -118.52     34.39                21.0       5477.0          1275.0
17247    -119.70     34.43                52.0       1364.0           460.0
14138    -117.07     32.74                37.0       1042.0           205.0
5722     -118.23     34.18                43.0       1708.0           280.0
18559    -122.06     36.96                52.0         65.0            17.0
18488    -121.59     36.97                16.0        865.0           123.0
9796     -121.88     36.49                28.0       2830.0           458.0
19705    -121.66     39.09                27.0       2098.0           372.0
19422    -121.04     37.70                52.0        349.0            59.0
10767    -117.90     33.63                28.0       2370.0           352.0
5227     -118.25     33.93                42.0        819.0           233.0
20336    -118.96     34.23                14.0      15207.0          2924.0
12284    -116.81     33.90                17.0       2009.0           469.0
19497    -121.00     37.66                43.0       2039.0           331.0
1828     -122.31     37.92                12.0       1895.0           600.0
4443     -118.21     34.09                37.0       1822.0           498.0
19911    -119.31     36.32                44.0       2032.0           308.0
3357     -120.65     40.42                39.0       3240.0           652.0

       population  households  median\_income  median\_house\_value  \textbackslash{}
5288       1725.0       962.0         4.1750            500001.0
12865       996.0       331.0         3.7031            114300.0
9174       3384.0      1222.0         3.6625            228100.0
17247       804.0       400.0         2.3750            293800.0
14138       589.0       208.0         2.6629            116900.0
5722        768.0       276.0         6.2070            457400.0
18559        24.0        10.0         4.5000            258300.0
18488       403.0       130.0         5.7396            308700.0
9796        898.0       370.0         5.8142            500001.0
19705      1090.0       333.0         4.4500             96200.0
19422       121.0        40.0         3.3036            197500.0
10767       832.0       347.0         7.1148            500001.0
5227        899.0       228.0         1.1346             85400.0
20336      6301.0      2829.0         3.9699            217000.0
12284       820.0       381.0         1.3286             81800.0
19497       875.0       342.0         3.9844            152000.0
1828        983.0       519.0         2.5000            195800.0
4443       1961.0       506.0         1.9881            159200.0
19911       791.0       336.0         4.0298            109000.0
3357       1467.0       621.0         2.1875             64300.0

      ocean\_proximity income\_cat
5288        <1H OCEAN          3
12865          INLAND          3
9174        <1H OCEAN          3
17247       <1H OCEAN          2
14138      NEAR OCEAN          2
5722        <1H OCEAN          5
18559      NEAR OCEAN          3
18488          INLAND          4
9796       NEAR OCEAN          4
19705          INLAND          3
19422          INLAND          3
10767       <1H OCEAN          5
5227        <1H OCEAN          1
20336       <1H OCEAN          3
12284          INLAND          1
19497          INLAND          3
1828         NEAR BAY          2
4443        <1H OCEAN          2
19911          INLAND          3
3357           INLAND          2
\end{Verbatim}
\end{tcolorbox}
        
    \begin{tcolorbox}[breakable, size=fbox, boxrule=1pt, pad at break*=1mm,colback=cellbackground, colframe=cellborder]
\prompt{In}{incolor}{30}{\boxspacing}
\begin{Verbatim}[commandchars=\\\{\}]
\PY{c+c1}{\PYZsh{} 查看测试集的分层}
\PY{n}{strat\PYZus{}train\PYZus{}set}\PY{p}{[}\PY{l+s+s1}{\PYZsq{}}\PY{l+s+s1}{income\PYZus{}cat}\PY{l+s+s1}{\PYZsq{}}\PY{p}{]}\PY{o}{.}\PY{n}{hist}\PY{p}{(}\PY{p}{)}
\end{Verbatim}
\end{tcolorbox}

            \begin{tcolorbox}[breakable, size=fbox, boxrule=.5pt, pad at break*=1mm, opacityfill=0]
\prompt{Out}{outcolor}{30}{\boxspacing}
\begin{Verbatim}[commandchars=\\\{\}]
<matplotlib.axes.\_subplots.AxesSubplot at 0x23bff443288>
\end{Verbatim}
\end{tcolorbox}
        
    \begin{center}
    \adjustimage{max size={0.9\linewidth}{0.9\paperheight}}{特征工程_files/特征工程_37_1.png}
    \end{center}
    { \hspace*{\fill} \\}
    
    \begin{tcolorbox}[breakable, size=fbox, boxrule=1pt, pad at break*=1mm,colback=cellbackground, colframe=cellborder]
\prompt{In}{incolor}{31}{\boxspacing}
\begin{Verbatim}[commandchars=\\\{\}]
\PY{c+c1}{\PYZsh{} 查看百分比}
\PY{n}{strat\PYZus{}train\PYZus{}set}\PY{p}{[}\PY{l+s+s1}{\PYZsq{}}\PY{l+s+s1}{income\PYZus{}cat}\PY{l+s+s1}{\PYZsq{}}\PY{p}{]}\PY{o}{.}\PY{n}{value\PYZus{}counts}\PY{p}{(}\PY{p}{)}\PY{o}{/}\PY{n+nb}{len}\PY{p}{(}\PY{n}{strat\PYZus{}train\PYZus{}set}\PY{p}{)} \PY{c+c1}{\PYZsh{} 分层抽样测试集合}
\end{Verbatim}
\end{tcolorbox}

            \begin{tcolorbox}[breakable, size=fbox, boxrule=.5pt, pad at break*=1mm, opacityfill=0]
\prompt{Out}{outcolor}{31}{\boxspacing}
\begin{Verbatim}[commandchars=\\\{\}]
3    0.350594
2    0.318859
4    0.176296
5    0.114402
1    0.039850
Name: income\_cat, dtype: float64
\end{Verbatim}
\end{tcolorbox}
        
    \begin{tcolorbox}[breakable, size=fbox, boxrule=1pt, pad at break*=1mm,colback=cellbackground, colframe=cellborder]
\prompt{In}{incolor}{32}{\boxspacing}
\begin{Verbatim}[commandchars=\\\{\}]
\PY{n}{data\PYZus{}housing}\PY{p}{[}\PY{l+s+s1}{\PYZsq{}}\PY{l+s+s1}{income\PYZus{}cat}\PY{l+s+s1}{\PYZsq{}}\PY{p}{]}\PY{o}{.}\PY{n}{value\PYZus{}counts}\PY{p}{(}\PY{p}{)} \PY{o}{/} \PY{n+nb}{len}\PY{p}{(}\PY{n}{data\PYZus{}housing}\PY{p}{)}  \PY{c+c1}{\PYZsh{} 与完整数据集合对比}
\end{Verbatim}
\end{tcolorbox}

            \begin{tcolorbox}[breakable, size=fbox, boxrule=.5pt, pad at break*=1mm, opacityfill=0]
\prompt{Out}{outcolor}{32}{\boxspacing}
\begin{Verbatim}[commandchars=\\\{\}]
3    0.350581
2    0.318847
4    0.176308
5    0.114438
1    0.039826
Name: income\_cat, dtype: float64
\end{Verbatim}
\end{tcolorbox}
        
    \begin{tcolorbox}[breakable, size=fbox, boxrule=1pt, pad at break*=1mm,colback=cellbackground, colframe=cellborder]
\prompt{In}{incolor}{33}{\boxspacing}
\begin{Verbatim}[commandchars=\\\{\}]
\PY{k}{def} \PY{n+nf}{income\PYZus{}cat\PYZus{}proportions}\PY{p}{(}\PY{n}{data}\PY{p}{)}\PY{p}{:}
    \PY{k}{return} \PY{n}{data}\PY{p}{[}\PY{l+s+s1}{\PYZsq{}}\PY{l+s+s1}{income\PYZus{}cat}\PY{l+s+s1}{\PYZsq{}}\PY{p}{]}\PY{o}{.}\PY{n}{value\PYZus{}counts}\PY{p}{(}\PY{p}{)} \PY{o}{/} \PY{n+nb}{len}\PY{p}{(}\PY{n}{data}\PY{p}{)}

\PY{n}{train\PYZus{}set}\PY{p}{,} \PY{n}{test\PYZus{}set} \PY{o}{=} \PY{n}{train\PYZus{}test\PYZus{}split}\PY{p}{(}\PY{n}{data\PYZus{}housing}\PY{p}{,} \PY{n}{test\PYZus{}size} \PY{o}{=} \PY{l+m+mf}{0.2}\PY{p}{,} \PY{n}{random\PYZus{}state}\PY{o}{=}\PY{l+m+mi}{42}\PY{p}{)}

\PY{n}{compare\PYZus{}props} \PY{o}{=} \PY{n}{pd}\PY{o}{.}\PY{n}{DataFrame}\PY{p}{(}\PY{p}{\PYZob{}}
    \PY{l+s+s2}{\PYZdq{}}\PY{l+s+s2}{全部数据}\PY{l+s+s2}{\PYZdq{}}\PY{p}{:} \PY{n}{income\PYZus{}cat\PYZus{}proportions}\PY{p}{(}\PY{n}{data\PYZus{}housing}\PY{p}{)}\PY{p}{,}
    \PY{l+s+s2}{\PYZdq{}}\PY{l+s+s2}{分层抽样}\PY{l+s+s2}{\PYZdq{}}\PY{p}{:} \PY{n}{income\PYZus{}cat\PYZus{}proportions}\PY{p}{(}\PY{n}{strat\PYZus{}test\PYZus{}set}\PY{p}{)}\PY{p}{,}
    \PY{l+s+s2}{\PYZdq{}}\PY{l+s+s2}{随机抽样}\PY{l+s+s2}{\PYZdq{}}\PY{p}{:} \PY{n}{income\PYZus{}cat\PYZus{}proportions}\PY{p}{(}\PY{n}{test\PYZus{}set}\PY{p}{)}\PY{p}{,}
\PY{p}{\PYZcb{}}\PY{p}{)}\PY{o}{.}\PY{n}{sort\PYZus{}index}\PY{p}{(}\PY{p}{)}
\PY{n}{compare\PYZus{}props}\PY{p}{[}\PY{l+s+s2}{\PYZdq{}}\PY{l+s+s2}{随机. }\PY{l+s+si}{\PYZpc{}e}\PY{l+s+s2}{rror}\PY{l+s+s2}{\PYZdq{}}\PY{p}{]} \PY{o}{=} \PY{l+m+mi}{100} \PY{o}{*} \PY{n}{compare\PYZus{}props}\PY{p}{[}\PY{l+s+s2}{\PYZdq{}}\PY{l+s+s2}{随机抽样}\PY{l+s+s2}{\PYZdq{}}\PY{p}{]} \PY{o}{/} \PY{n}{compare\PYZus{}props}\PY{p}{[}\PY{l+s+s2}{\PYZdq{}}\PY{l+s+s2}{全部数据}\PY{l+s+s2}{\PYZdq{}}\PY{p}{]} \PY{o}{\PYZhy{}} \PY{l+m+mi}{100}
\PY{n}{compare\PYZus{}props}\PY{p}{[}\PY{l+s+s2}{\PYZdq{}}\PY{l+s+s2}{分层. }\PY{l+s+si}{\PYZpc{}e}\PY{l+s+s2}{rror}\PY{l+s+s2}{\PYZdq{}}\PY{p}{]} \PY{o}{=} \PY{l+m+mi}{100} \PY{o}{*} \PY{n}{compare\PYZus{}props}\PY{p}{[}\PY{l+s+s2}{\PYZdq{}}\PY{l+s+s2}{分层抽样}\PY{l+s+s2}{\PYZdq{}}\PY{p}{]} \PY{o}{/} \PY{n}{compare\PYZus{}props}\PY{p}{[}\PY{l+s+s2}{\PYZdq{}}\PY{l+s+s2}{全部数据}\PY{l+s+s2}{\PYZdq{}}\PY{p}{]} \PY{o}{\PYZhy{}} \PY{l+m+mi}{100}
\end{Verbatim}
\end{tcolorbox}

    \begin{tcolorbox}[breakable, size=fbox, boxrule=1pt, pad at break*=1mm,colback=cellbackground, colframe=cellborder]
\prompt{In}{incolor}{34}{\boxspacing}
\begin{Verbatim}[commandchars=\\\{\}]
\PY{n}{compare\PYZus{}props}
\end{Verbatim}
\end{tcolorbox}

            \begin{tcolorbox}[breakable, size=fbox, boxrule=.5pt, pad at break*=1mm, opacityfill=0]
\prompt{Out}{outcolor}{34}{\boxspacing}
\begin{Verbatim}[commandchars=\\\{\}]
       全部数据      分层抽样      随机抽样  随机. \%error  分层. \%error
1  0.039826  0.039729  0.040213    0.973236   -0.243309
2  0.318847  0.318798  0.324370    1.732260   -0.015195
3  0.350581  0.350533  0.358527    2.266446   -0.013820
4  0.176308  0.176357  0.167393   -5.056334    0.027480
5  0.114438  0.114583  0.109496   -4.318374    0.127011
\end{Verbatim}
\end{tcolorbox}
        
    \hypertarget{ux5c06ux5730ux7406ux6570ux636eux53efux89c6ux5316}{%
\subsection{将地理数据可视化}\label{ux5c06ux5730ux7406ux6570ux636eux53efux89c6ux5316}}

    \begin{tcolorbox}[breakable, size=fbox, boxrule=1pt, pad at break*=1mm,colback=cellbackground, colframe=cellborder]
\prompt{In}{incolor}{35}{\boxspacing}
\begin{Verbatim}[commandchars=\\\{\}]
\PY{n}{data\PYZus{}housing}\PY{o}{.}\PY{n}{plot}\PY{p}{(}\PY{n}{kind}\PY{o}{=}\PY{l+s+s1}{\PYZsq{}}\PY{l+s+s1}{scatter}\PY{l+s+s1}{\PYZsq{}}\PY{p}{,} \PY{n}{x}\PY{o}{=}\PY{l+s+s1}{\PYZsq{}}\PY{l+s+s1}{longitude}\PY{l+s+s1}{\PYZsq{}}\PY{p}{,} \PY{n}{y}\PY{o}{=}\PY{l+s+s1}{\PYZsq{}}\PY{l+s+s1}{latitude}\PY{l+s+s1}{\PYZsq{}}\PY{p}{)}
\end{Verbatim}
\end{tcolorbox}

            \begin{tcolorbox}[breakable, size=fbox, boxrule=.5pt, pad at break*=1mm, opacityfill=0]
\prompt{Out}{outcolor}{35}{\boxspacing}
\begin{Verbatim}[commandchars=\\\{\}]
<matplotlib.axes.\_subplots.AxesSubplot at 0x23bff0f6248>
\end{Verbatim}
\end{tcolorbox}
        
    \begin{center}
    \adjustimage{max size={0.9\linewidth}{0.9\paperheight}}{特征工程_files/特征工程_43_1.png}
    \end{center}
    { \hspace*{\fill} \\}
    
    \begin{tcolorbox}[breakable, size=fbox, boxrule=1pt, pad at break*=1mm,colback=cellbackground, colframe=cellborder]
\prompt{In}{incolor}{36}{\boxspacing}
\begin{Verbatim}[commandchars=\\\{\}]
\PY{n}{data\PYZus{}housing}\PY{o}{.}\PY{n}{plot}\PY{p}{(}\PY{n}{kind}\PY{o}{=}\PY{l+s+s1}{\PYZsq{}}\PY{l+s+s1}{scatter}\PY{l+s+s1}{\PYZsq{}}\PY{p}{,} \PY{n}{x}\PY{o}{=}\PY{l+s+s1}{\PYZsq{}}\PY{l+s+s1}{longitude}\PY{l+s+s1}{\PYZsq{}}\PY{p}{,} \PY{n}{y}\PY{o}{=}\PY{l+s+s1}{\PYZsq{}}\PY{l+s+s1}{latitude}\PY{l+s+s1}{\PYZsq{}}\PY{p}{,}\PY{n}{alpha}\PY{o}{=}\PY{l+m+mf}{0.1}\PY{p}{)}
\end{Verbatim}
\end{tcolorbox}

            \begin{tcolorbox}[breakable, size=fbox, boxrule=.5pt, pad at break*=1mm, opacityfill=0]
\prompt{Out}{outcolor}{36}{\boxspacing}
\begin{Verbatim}[commandchars=\\\{\}]
<matplotlib.axes.\_subplots.AxesSubplot at 0x23bfed5a608>
\end{Verbatim}
\end{tcolorbox}
        
    \begin{center}
    \adjustimage{max size={0.9\linewidth}{0.9\paperheight}}{特征工程_files/特征工程_44_1.png}
    \end{center}
    { \hspace*{\fill} \\}
    
    \begin{tcolorbox}[breakable, size=fbox, boxrule=1pt, pad at break*=1mm,colback=cellbackground, colframe=cellborder]
\prompt{In}{incolor}{37}{\boxspacing}
\begin{Verbatim}[commandchars=\\\{\}]
\PY{c+c1}{\PYZsh{} alpha为透明度}
\PY{c+c1}{\PYZsh{} s为size,根据\PYZdq{}population\PYZdq{}设置大小}
\PY{c+c1}{\PYZsh{} cmap为颜色图,根据c=\PYZdq{}median\PYZus{}house\PYZus{}value\PYZdq{}设置颜色渐变,jet为渐变类型}
\PY{n}{data\PYZus{}housing}\PY{o}{.}\PY{n}{plot}\PY{p}{(}\PY{n}{kind}\PY{o}{=}\PY{l+s+s1}{\PYZsq{}}\PY{l+s+s1}{scatter}\PY{l+s+s1}{\PYZsq{}}\PY{p}{,} \PY{n}{x}\PY{o}{=}\PY{l+s+s1}{\PYZsq{}}\PY{l+s+s1}{longitude}\PY{l+s+s1}{\PYZsq{}}\PY{p}{,} \PY{n}{y}\PY{o}{=}\PY{l+s+s1}{\PYZsq{}}\PY{l+s+s1}{latitude}\PY{l+s+s1}{\PYZsq{}}\PY{p}{,}\PY{n}{alpha}\PY{o}{=}\PY{l+m+mf}{0.4}\PY{p}{,}
                 \PY{n}{s}\PY{o}{=}\PY{n}{data\PYZus{}housing}\PY{p}{[}\PY{l+s+s1}{\PYZsq{}}\PY{l+s+s1}{population}\PY{l+s+s1}{\PYZsq{}}\PY{p}{]}\PY{o}{/}\PY{l+m+mi}{100}\PY{p}{,} \PY{n}{label}\PY{o}{=}\PY{l+s+s2}{\PYZdq{}}\PY{l+s+s2}{population}\PY{l+s+s2}{\PYZdq{}}\PY{p}{,} \PY{n}{figsize}\PY{o}{=}\PY{p}{(}\PY{l+m+mi}{10}\PY{p}{,} \PY{l+m+mi}{7}\PY{p}{)}\PY{p}{,}
                  \PY{n}{c}\PY{o}{=}\PY{l+s+s2}{\PYZdq{}}\PY{l+s+s2}{median\PYZus{}house\PYZus{}value}\PY{l+s+s2}{\PYZdq{}}\PY{p}{,} \PY{n}{cmap}\PY{o}{=}\PY{n}{plt}\PY{o}{.}\PY{n}{get\PYZus{}cmap}\PY{p}{(}\PY{l+s+s1}{\PYZsq{}}\PY{l+s+s1}{jet}\PY{l+s+s1}{\PYZsq{}}\PY{p}{)}\PY{p}{,} \PY{n}{colorbar}\PY{o}{=}\PY{k+kc}{True}\PY{p}{,}
                  \PY{n}{sharex}\PY{o}{=}\PY{k+kc}{False}
                 \PY{p}{)}
\end{Verbatim}
\end{tcolorbox}

            \begin{tcolorbox}[breakable, size=fbox, boxrule=.5pt, pad at break*=1mm, opacityfill=0]
\prompt{Out}{outcolor}{37}{\boxspacing}
\begin{Verbatim}[commandchars=\\\{\}]
<matplotlib.axes.\_subplots.AxesSubplot at 0x23bff182788>
\end{Verbatim}
\end{tcolorbox}
        
    \begin{center}
    \adjustimage{max size={0.9\linewidth}{0.9\paperheight}}{特征工程_files/特征工程_45_1.png}
    \end{center}
    { \hspace*{\fill} \\}
    
    \begin{tcolorbox}[breakable, size=fbox, boxrule=1pt, pad at break*=1mm,colback=cellbackground, colframe=cellborder]
\prompt{In}{incolor}{38}{\boxspacing}
\begin{Verbatim}[commandchars=\\\{\}]
\PY{k+kn}{import} \PY{n+nn}{matplotlib}\PY{n+nn}{.}\PY{n+nn}{image} \PY{k}{as} \PY{n+nn}{mpimg}

\PY{c+c1}{\PYZsh{}\PYZsh{}\PYZsh{} 1.绘制散点图}
\PY{c+c1}{\PYZsh{} alpha为透明度}
\PY{c+c1}{\PYZsh{} s为size,根据\PYZdq{}population\PYZdq{}设置大小}
\PY{c+c1}{\PYZsh{} cmap为颜色图,根据c=\PYZdq{}median\PYZus{}house\PYZus{}value\PYZdq{}设置颜色渐变,jet为渐变类型}
\PY{c+c1}{\PYZsh{} colorbar=True为显示颜色条}
\PY{n}{data\PYZus{}housing}\PY{o}{.}\PY{n}{plot}\PY{p}{(}\PY{n}{kind}\PY{o}{=}\PY{l+s+s1}{\PYZsq{}}\PY{l+s+s1}{scatter}\PY{l+s+s1}{\PYZsq{}}\PY{p}{,} \PY{n}{x}\PY{o}{=}\PY{l+s+s1}{\PYZsq{}}\PY{l+s+s1}{longitude}\PY{l+s+s1}{\PYZsq{}}\PY{p}{,} \PY{n}{y}\PY{o}{=}\PY{l+s+s1}{\PYZsq{}}\PY{l+s+s1}{latitude}\PY{l+s+s1}{\PYZsq{}}\PY{p}{,}\PY{n}{alpha}\PY{o}{=}\PY{l+m+mf}{0.4}\PY{p}{,}
                 \PY{n}{s}\PY{o}{=}\PY{n}{data\PYZus{}housing}\PY{p}{[}\PY{l+s+s1}{\PYZsq{}}\PY{l+s+s1}{population}\PY{l+s+s1}{\PYZsq{}}\PY{p}{]}\PY{o}{/}\PY{l+m+mi}{100}\PY{p}{,} \PY{n}{label}\PY{o}{=}\PY{l+s+s2}{\PYZdq{}}\PY{l+s+s2}{population}\PY{l+s+s2}{\PYZdq{}}\PY{p}{,} \PY{n}{figsize}\PY{o}{=}\PY{p}{(}\PY{l+m+mi}{10}\PY{p}{,} \PY{l+m+mi}{7}\PY{p}{)}\PY{p}{,}
                  \PY{n}{c}\PY{o}{=}\PY{l+s+s2}{\PYZdq{}}\PY{l+s+s2}{median\PYZus{}house\PYZus{}value}\PY{l+s+s2}{\PYZdq{}}\PY{p}{,} \PY{n}{cmap}\PY{o}{=}\PY{n}{plt}\PY{o}{.}\PY{n}{get\PYZus{}cmap}\PY{p}{(}\PY{l+s+s1}{\PYZsq{}}\PY{l+s+s1}{jet}\PY{l+s+s1}{\PYZsq{}}\PY{p}{)}\PY{p}{,} \PY{n}{colorbar}\PY{o}{=}\PY{k+kc}{False}\PY{p}{,}
                  \PY{n}{sharex}\PY{o}{=}\PY{k+kc}{False}
                 \PY{p}{)}

\PY{c+c1}{\PYZsh{}\PYZsh{}\PYZsh{} 2.添加地理背景图}
\PY{n}{california\PYZus{}img} \PY{o}{=} \PY{n}{mpimg}\PY{o}{.}\PY{n}{imread}\PY{p}{(}\PY{l+s+s2}{\PYZdq{}}\PY{l+s+s2}{./california.png}\PY{l+s+s2}{\PYZdq{}}\PY{p}{)}
\PY{c+c1}{\PYZsh{} extent为(X起点,X终点,Y起点,Y终点)}
\PY{n}{plt}\PY{o}{.}\PY{n}{imshow}\PY{p}{(}\PY{n}{california\PYZus{}img}\PY{p}{,} \PY{n}{extent}\PY{o}{=}\PY{p}{[}\PY{o}{\PYZhy{}}\PY{l+m+mf}{124.55}\PY{p}{,} \PY{o}{\PYZhy{}}\PY{l+m+mf}{113.80}\PY{p}{,} \PY{l+m+mf}{32.45}\PY{p}{,} \PY{l+m+mf}{42.05}\PY{p}{]}\PY{p}{,} \PY{n}{alpha}\PY{o}{=}\PY{l+m+mf}{0.5}\PY{p}{,}
          \PY{n}{cmap}\PY{o}{=}\PY{n}{plt}\PY{o}{.}\PY{n}{get\PYZus{}cmap}\PY{p}{(}\PY{l+s+s2}{\PYZdq{}}\PY{l+s+s2}{jet}\PY{l+s+s2}{\PYZdq{}}\PY{p}{)}\PY{p}{)}

\PY{c+c1}{\PYZsh{}\PYZsh{}\PYZsh{} 3.构建价格渐变条}
\PY{n}{prices} \PY{o}{=} \PY{n}{data\PYZus{}housing}\PY{p}{[}\PY{l+s+s2}{\PYZdq{}}\PY{l+s+s2}{median\PYZus{}house\PYZus{}value}\PY{l+s+s2}{\PYZdq{}}\PY{p}{]}  \PY{c+c1}{\PYZsh{} 使用价格中位数列}
\PY{n}{tick\PYZus{}values} \PY{o}{=} \PY{n}{np}\PY{o}{.}\PY{n}{linspace}\PY{p}{(}\PY{n}{prices}\PY{o}{.}\PY{n}{min}\PY{p}{(}\PY{p}{)}\PY{p}{,} \PY{n}{prices}\PY{o}{.}\PY{n}{max}\PY{p}{(}\PY{p}{)}\PY{p}{,} \PY{l+m+mi}{11}\PY{p}{)}  \PY{c+c1}{\PYZsh{} 将价格分为一个等差列表}
\PY{n}{cbar} \PY{o}{=} \PY{n}{plt}\PY{o}{.}\PY{n}{colorbar}\PY{p}{(}\PY{p}{)}
\PY{c+c1}{\PYZsh{} 使用等差列表构建一个label}
\PY{n}{cbar}\PY{o}{.}\PY{n}{ax}\PY{o}{.}\PY{n}{set\PYZus{}yticklabels}\PY{p}{(}\PY{p}{[}\PY{l+s+s2}{\PYZdq{}}\PY{l+s+s2}{\PYZdl{}}\PY{l+s+si}{\PYZpc{}d}\PY{l+s+s2}{k}\PY{l+s+s2}{\PYZdq{}}\PY{o}{\PYZpc{}}\PY{p}{(}\PY{n+nb}{round}\PY{p}{(}\PY{n}{v}\PY{o}{/}\PY{l+m+mi}{1000}\PY{p}{)}\PY{p}{)} \PY{k}{for} \PY{n}{v} \PY{o+ow}{in} \PY{n}{tick\PYZus{}values}\PY{p}{]}\PY{p}{,} \PY{n}{fontsize}\PY{o}{=}\PY{l+m+mi}{14}\PY{p}{)}  \PY{c+c1}{\PYZsh{} round()返回四舍五入的小数}
\PY{n}{cbar}\PY{o}{.}\PY{n}{set\PYZus{}label}\PY{p}{(}\PY{l+s+s1}{\PYZsq{}}\PY{l+s+s1}{Median House Value}\PY{l+s+s1}{\PYZsq{}}\PY{p}{,} \PY{n}{fontsize}\PY{o}{=}\PY{l+m+mi}{16}\PY{p}{)}

\PY{c+c1}{\PYZsh{}\PYZsh{}\PYZsh{} 4.显示图片}
\PY{n}{plt}\PY{o}{.}\PY{n}{ylabel}\PY{p}{(}\PY{l+s+s2}{\PYZdq{}}\PY{l+s+s2}{Latitude}\PY{l+s+s2}{\PYZdq{}}\PY{p}{,} \PY{n}{fontsize}\PY{o}{=}\PY{l+m+mi}{14}\PY{p}{)}
\PY{n}{plt}\PY{o}{.}\PY{n}{xlabel}\PY{p}{(}\PY{l+s+s2}{\PYZdq{}}\PY{l+s+s2}{Longitude}\PY{l+s+s2}{\PYZdq{}}\PY{p}{,} \PY{n}{fontsize}\PY{o}{=}\PY{l+m+mi}{14}\PY{p}{)}
\PY{n}{plt}\PY{o}{.}\PY{n}{legend}\PY{p}{(}\PY{n}{fontsize}\PY{o}{=}\PY{l+m+mi}{16}\PY{p}{)}
\PY{n}{plt}\PY{o}{.}\PY{n}{show}\PY{p}{(}\PY{p}{)}
\end{Verbatim}
\end{tcolorbox}

    \begin{center}
    \adjustimage{max size={0.9\linewidth}{0.9\paperheight}}{特征工程_files/特征工程_46_0.png}
    \end{center}
    { \hspace*{\fill} \\}
    
    \hypertarget{ux5bfbux627eux7279ux5f81ux76f8ux5173ux6027}{%
\subsection{寻找特征相关性}\label{ux5bfbux627eux7279ux5f81ux76f8ux5173ux6027}}

    \hypertarget{ux76aeux5c14ux900aux76f8ux5173ux7cfbux6570}{%
\paragraph{皮尔逊相关系数}\label{ux76aeux5c14ux900aux76f8ux5173ux7cfbux6570}}

    \begin{tcolorbox}[breakable, size=fbox, boxrule=1pt, pad at break*=1mm,colback=cellbackground, colframe=cellborder]
\prompt{In}{incolor}{39}{\boxspacing}
\begin{Verbatim}[commandchars=\\\{\}]
\PY{c+c1}{\PYZsh{} corr() 计算出每对特征之间的相关系数,成为皮尔逊相关系数}
\PY{c+c1}{\PYZsh{} 皮尔逊相关系数:\PYZhy{}1 \PYZti{} 0 \PYZti{} 1    负相关\PYZti{}不相关\PYZti{}正相关}
\PY{n}{corr\PYZus{}matrix} \PY{o}{=} \PY{n}{data\PYZus{}housing}\PY{o}{.}\PY{n}{corr}\PY{p}{(}\PY{p}{)}
\PY{n}{corr\PYZus{}matrix}
\end{Verbatim}
\end{tcolorbox}

            \begin{tcolorbox}[breakable, size=fbox, boxrule=.5pt, pad at break*=1mm, opacityfill=0]
\prompt{Out}{outcolor}{39}{\boxspacing}
\begin{Verbatim}[commandchars=\\\{\}]
                    longitude  latitude  housing\_median\_age  total\_rooms  \textbackslash{}
longitude            1.000000 -0.924664           -0.108197     0.044568
latitude            -0.924664  1.000000            0.011173    -0.036100
housing\_median\_age  -0.108197  0.011173            1.000000    -0.361262
total\_rooms          0.044568 -0.036100           -0.361262     1.000000
total\_bedrooms       0.069608 -0.066983           -0.320451     0.930380
population           0.099773 -0.108785           -0.296244     0.857126
households           0.055310 -0.071035           -0.302916     0.918484
median\_income       -0.015176 -0.079809           -0.119034     0.198050
median\_house\_value  -0.045967 -0.144160            0.105623     0.134153

                    total\_bedrooms  population  households  median\_income  \textbackslash{}
longitude                 0.069608    0.099773    0.055310      -0.015176
latitude                 -0.066983   -0.108785   -0.071035      -0.079809
housing\_median\_age       -0.320451   -0.296244   -0.302916      -0.119034
total\_rooms               0.930380    0.857126    0.918484       0.198050
total\_bedrooms            1.000000    0.877747    0.979728      -0.007723
population                0.877747    1.000000    0.907222       0.004834
households                0.979728    0.907222    1.000000       0.013033
median\_income            -0.007723    0.004834    0.013033       1.000000
median\_house\_value        0.049686   -0.024650    0.065843       0.688075

                    median\_house\_value
longitude                    -0.045967
latitude                     -0.144160
housing\_median\_age            0.105623
total\_rooms                   0.134153
total\_bedrooms                0.049686
population                   -0.024650
households                    0.065843
median\_income                 0.688075
median\_house\_value            1.000000
\end{Verbatim}
\end{tcolorbox}
        
    \begin{tcolorbox}[breakable, size=fbox, boxrule=1pt, pad at break*=1mm,colback=cellbackground, colframe=cellborder]
\prompt{In}{incolor}{40}{\boxspacing}
\begin{Verbatim}[commandchars=\\\{\}]
\PY{n}{corr\PYZus{}matrix}\PY{p}{[}\PY{l+s+s1}{\PYZsq{}}\PY{l+s+s1}{median\PYZus{}house\PYZus{}value}\PY{l+s+s1}{\PYZsq{}}\PY{p}{]}\PY{o}{.}\PY{n}{sort\PYZus{}values}\PY{p}{(}\PY{n}{ascending} \PY{o}{=} \PY{k+kc}{False}\PY{p}{)}
\end{Verbatim}
\end{tcolorbox}

            \begin{tcolorbox}[breakable, size=fbox, boxrule=.5pt, pad at break*=1mm, opacityfill=0]
\prompt{Out}{outcolor}{40}{\boxspacing}
\begin{Verbatim}[commandchars=\\\{\}]
median\_house\_value    1.000000
median\_income         0.688075
total\_rooms           0.134153
housing\_median\_age    0.105623
households            0.065843
total\_bedrooms        0.049686
population           -0.024650
longitude            -0.045967
latitude             -0.144160
Name: median\_house\_value, dtype: float64
\end{Verbatim}
\end{tcolorbox}
        
    \hypertarget{ux7ed8ux5236ux7279ux5f81ux76f8ux5173ux6027}{%
\paragraph{绘制特征相关性}\label{ux7ed8ux5236ux7279ux5f81ux76f8ux5173ux6027}}

    \begin{tcolorbox}[breakable, size=fbox, boxrule=1pt, pad at break*=1mm,colback=cellbackground, colframe=cellborder]
\prompt{In}{incolor}{41}{\boxspacing}
\begin{Verbatim}[commandchars=\\\{\}]
\PY{c+c1}{\PYZsh{} 使用pandas scatter\PYZus{}matrix 函数,可以绘制特征之间的相关性}
\PY{k+kn}{from} \PY{n+nn}{pandas}\PY{n+nn}{.}\PY{n+nn}{plotting} \PY{k+kn}{import} \PY{n}{scatter\PYZus{}matrix}

\PY{n}{attributes} \PY{o}{=} \PY{p}{[}\PY{l+s+s1}{\PYZsq{}}\PY{l+s+s1}{median\PYZus{}house\PYZus{}value}\PY{l+s+s1}{\PYZsq{}}\PY{p}{,} \PY{l+s+s1}{\PYZsq{}}\PY{l+s+s1}{median\PYZus{}income}\PY{l+s+s1}{\PYZsq{}} \PY{p}{,} \PY{l+s+s1}{\PYZsq{}}\PY{l+s+s1}{total\PYZus{}rooms}\PY{l+s+s1}{\PYZsq{}}\PY{p}{,} \PY{l+s+s1}{\PYZsq{}}\PY{l+s+s1}{housing\PYZus{}median\PYZus{}age}\PY{l+s+s1}{\PYZsq{}}\PY{p}{]}
\PY{n}{scatter\PYZus{}matrix}\PY{p}{(}\PY{n}{data\PYZus{}housing}\PY{p}{[}\PY{n}{attributes}\PY{p}{]}\PY{p}{,} \PY{n}{figsize} \PY{o}{=} \PY{p}{(}\PY{l+m+mi}{12}\PY{p}{,} \PY{l+m+mi}{8}\PY{p}{)}\PY{p}{)}
\end{Verbatim}
\end{tcolorbox}

            \begin{tcolorbox}[breakable, size=fbox, boxrule=.5pt, pad at break*=1mm, opacityfill=0]
\prompt{Out}{outcolor}{41}{\boxspacing}
\begin{Verbatim}[commandchars=\\\{\}]
array([[<matplotlib.axes.\_subplots.AxesSubplot object at 0x0000023BFF355848>,
        <matplotlib.axes.\_subplots.AxesSubplot object at 0x0000023BFF11D948>,
        <matplotlib.axes.\_subplots.AxesSubplot object at 0x0000023BFEFA41C8>,
        <matplotlib.axes.\_subplots.AxesSubplot object at 0x0000023BFF032B88>],
       [<matplotlib.axes.\_subplots.AxesSubplot object at 0x0000023BFF4C7908>,
        <matplotlib.axes.\_subplots.AxesSubplot object at 0x0000023BFEF37A08>,
        <matplotlib.axes.\_subplots.AxesSubplot object at 0x0000023BFF47CAC8>,
        <matplotlib.axes.\_subplots.AxesSubplot object at 0x0000023BFEE6FBC8>],
       [<matplotlib.axes.\_subplots.AxesSubplot object at 0x0000023BFEE5E7C8>,
        <matplotlib.axes.\_subplots.AxesSubplot object at 0x0000023BFEEE0988>,
        <matplotlib.axes.\_subplots.AxesSubplot object at 0x0000023BFEDF1F08>,
        <matplotlib.axes.\_subplots.AxesSubplot object at 0x0000023BFEDC3FC8>],
       [<matplotlib.axes.\_subplots.AxesSubplot object at 0x0000023BFEE3A148>,
        <matplotlib.axes.\_subplots.AxesSubplot object at 0x0000023B80307248>,
        <matplotlib.axes.\_subplots.AxesSubplot object at 0x0000023B8033F348>,
        <matplotlib.axes.\_subplots.AxesSubplot object at 0x0000023B80376488>]],
      dtype=object)
\end{Verbatim}
\end{tcolorbox}
        
    \begin{center}
    \adjustimage{max size={0.9\linewidth}{0.9\paperheight}}{特征工程_files/特征工程_52_1.png}
    \end{center}
    { \hspace*{\fill} \\}
    
    \begin{tcolorbox}[breakable, size=fbox, boxrule=1pt, pad at break*=1mm,colback=cellbackground, colframe=cellborder]
\prompt{In}{incolor}{42}{\boxspacing}
\begin{Verbatim}[commandchars=\\\{\}]
\PY{n}{data\PYZus{}housing}\PY{o}{.}\PY{n}{plot}\PY{p}{(}\PY{n}{kind} \PY{o}{=} \PY{l+s+s1}{\PYZsq{}}\PY{l+s+s1}{scatter}\PY{l+s+s1}{\PYZsq{}}\PY{p}{,}\PY{n}{x}\PY{o}{=}\PY{l+s+s1}{\PYZsq{}}\PY{l+s+s1}{median\PYZus{}income}\PY{l+s+s1}{\PYZsq{}}\PY{p}{,} \PY{n}{y} \PY{o}{=} \PY{l+s+s1}{\PYZsq{}}\PY{l+s+s1}{median\PYZus{}house\PYZus{}value}\PY{l+s+s1}{\PYZsq{}}\PY{p}{,} \PY{n}{alpha} \PY{o}{=} \PY{l+m+mf}{0.1}\PY{p}{)}
\PY{n}{plt}\PY{o}{.}\PY{n}{axis}\PY{p}{(}\PY{p}{[}\PY{l+m+mi}{0}\PY{p}{,} \PY{l+m+mi}{16}\PY{p}{,} \PY{l+m+mi}{0}\PY{p}{,} \PY{l+m+mi}{550000}\PY{p}{]}\PY{p}{)}
\PY{c+c1}{\PYZsh{} 50W直线为人为设置的上限,45w,35W,28W也有虚线,可能是异常值,后期处理删除异常值}
\end{Verbatim}
\end{tcolorbox}

            \begin{tcolorbox}[breakable, size=fbox, boxrule=.5pt, pad at break*=1mm, opacityfill=0]
\prompt{Out}{outcolor}{42}{\boxspacing}
\begin{Verbatim}[commandchars=\\\{\}]
[0, 16, 0, 550000]
\end{Verbatim}
\end{tcolorbox}
        
    \begin{center}
    \adjustimage{max size={0.9\linewidth}{0.9\paperheight}}{特征工程_files/特征工程_53_1.png}
    \end{center}
    { \hspace*{\fill} \\}
    
    \hypertarget{ux4e0dux540cux7279ux5f81ux7684ux7ec4ux5408ux76f8ux5173ux6027}{%
\paragraph{不同特征的组合相关性}\label{ux4e0dux540cux7279ux5f81ux7684ux7ec4ux5408ux76f8ux5173ux6027}}

    \begin{tcolorbox}[breakable, size=fbox, boxrule=1pt, pad at break*=1mm,colback=cellbackground, colframe=cellborder]
\prompt{In}{incolor}{43}{\boxspacing}
\begin{Verbatim}[commandchars=\\\{\}]
\PY{n}{data\PYZus{}housing}\PY{p}{[}\PY{l+s+s1}{\PYZsq{}}\PY{l+s+s1}{rooms\PYZus{}per\PYZus{}household}\PY{l+s+s1}{\PYZsq{}}\PY{p}{]} \PY{o}{=} \PY{n}{data\PYZus{}housing}\PY{p}{[}\PY{l+s+s1}{\PYZsq{}}\PY{l+s+s1}{total\PYZus{}rooms}\PY{l+s+s1}{\PYZsq{}}\PY{p}{]}\PY{o}{/}\PY{n}{data\PYZus{}housing}\PY{p}{[}\PY{l+s+s1}{\PYZsq{}}\PY{l+s+s1}{households}\PY{l+s+s1}{\PYZsq{}}\PY{p}{]}
\PY{n}{data\PYZus{}housing}\PY{p}{[}\PY{l+s+s1}{\PYZsq{}}\PY{l+s+s1}{bedrooms\PYZus{}per\PYZus{}room}\PY{l+s+s1}{\PYZsq{}}\PY{p}{]} \PY{o}{=} \PY{n}{data\PYZus{}housing}\PY{p}{[}\PY{l+s+s1}{\PYZsq{}}\PY{l+s+s1}{total\PYZus{}bedrooms}\PY{l+s+s1}{\PYZsq{}}\PY{p}{]}\PY{o}{/}\PY{n}{data\PYZus{}housing}\PY{p}{[}\PY{l+s+s1}{\PYZsq{}}\PY{l+s+s1}{total\PYZus{}rooms}\PY{l+s+s1}{\PYZsq{}}\PY{p}{]}
\PY{n}{data\PYZus{}housing}\PY{p}{[}\PY{l+s+s1}{\PYZsq{}}\PY{l+s+s1}{population\PYZus{}per\PYZus{}household}\PY{l+s+s1}{\PYZsq{}}\PY{p}{]} \PY{o}{=} \PY{n}{data\PYZus{}housing}\PY{p}{[}\PY{l+s+s1}{\PYZsq{}}\PY{l+s+s1}{population}\PY{l+s+s1}{\PYZsq{}}\PY{p}{]}\PY{o}{/}\PY{n}{data\PYZus{}housing}\PY{p}{[}\PY{l+s+s1}{\PYZsq{}}\PY{l+s+s1}{households}\PY{l+s+s1}{\PYZsq{}}\PY{p}{]}
\end{Verbatim}
\end{tcolorbox}

    \begin{tcolorbox}[breakable, size=fbox, boxrule=1pt, pad at break*=1mm,colback=cellbackground, colframe=cellborder]
\prompt{In}{incolor}{44}{\boxspacing}
\begin{Verbatim}[commandchars=\\\{\}]
\PY{n}{corr\PYZus{}matrix} \PY{o}{=} \PY{n}{data\PYZus{}housing}\PY{o}{.}\PY{n}{corr}\PY{p}{(}\PY{p}{)}
\PY{n}{corr\PYZus{}matrix}\PY{p}{[}\PY{l+s+s1}{\PYZsq{}}\PY{l+s+s1}{median\PYZus{}house\PYZus{}value}\PY{l+s+s1}{\PYZsq{}}\PY{p}{]}\PY{o}{.}\PY{n}{sort\PYZus{}values}\PY{p}{(}\PY{n}{ascending} \PY{o}{=} \PY{k+kc}{False}\PY{p}{)}
\end{Verbatim}
\end{tcolorbox}

            \begin{tcolorbox}[breakable, size=fbox, boxrule=.5pt, pad at break*=1mm, opacityfill=0]
\prompt{Out}{outcolor}{44}{\boxspacing}
\begin{Verbatim}[commandchars=\\\{\}]
median\_house\_value          1.000000
median\_income               0.688075
rooms\_per\_household         0.151948
total\_rooms                 0.134153
housing\_median\_age          0.105623
households                  0.065843
total\_bedrooms              0.049686
population\_per\_household   -0.023737
population                 -0.024650
longitude                  -0.045967
latitude                   -0.144160
bedrooms\_per\_room          -0.255880
Name: median\_house\_value, dtype: float64
\end{Verbatim}
\end{tcolorbox}
        
    \hypertarget{ux673aux5668ux5b66ux4e60ux7b97ux6cd5ux7684ux6570ux636eux51c6ux5907}{%
\section{机器学习算法的数据准备}\label{ux673aux5668ux5b66ux4e60ux7b97ux6cd5ux7684ux6570ux636eux51c6ux5907}}

    \begin{tcolorbox}[breakable, size=fbox, boxrule=1pt, pad at break*=1mm,colback=cellbackground, colframe=cellborder]
\prompt{In}{incolor}{45}{\boxspacing}
\begin{Verbatim}[commandchars=\\\{\}]
\PY{c+c1}{\PYZsh{} 得到一个分层抽样代表全局数据集的 训练集和测试集}
\PY{n}{strat\PYZus{}train\PYZus{}set}\PY{o}{.}\PY{n}{head}\PY{p}{(}\PY{p}{)}
\end{Verbatim}
\end{tcolorbox}

            \begin{tcolorbox}[breakable, size=fbox, boxrule=.5pt, pad at break*=1mm, opacityfill=0]
\prompt{Out}{outcolor}{45}{\boxspacing}
\begin{Verbatim}[commandchars=\\\{\}]
       longitude  latitude  housing\_median\_age  total\_rooms  total\_bedrooms  \textbackslash{}
5288     -118.47     34.05                27.0       4401.0          1033.0
12865    -121.34     38.69                17.0       1968.0           364.0
9174     -118.52     34.39                21.0       5477.0          1275.0
17247    -119.70     34.43                52.0       1364.0           460.0
14138    -117.07     32.74                37.0       1042.0           205.0

       population  households  median\_income  median\_house\_value  \textbackslash{}
5288       1725.0       962.0         4.1750            500001.0
12865       996.0       331.0         3.7031            114300.0
9174       3384.0      1222.0         3.6625            228100.0
17247       804.0       400.0         2.3750            293800.0
14138       589.0       208.0         2.6629            116900.0

      ocean\_proximity income\_cat
5288        <1H OCEAN          3
12865          INLAND          3
9174        <1H OCEAN          3
17247       <1H OCEAN          2
14138      NEAR OCEAN          2
\end{Verbatim}
\end{tcolorbox}
        
    \begin{tcolorbox}[breakable, size=fbox, boxrule=1pt, pad at break*=1mm,colback=cellbackground, colframe=cellborder]
\prompt{In}{incolor}{46}{\boxspacing}
\begin{Verbatim}[commandchars=\\\{\}]
\PY{c+c1}{\PYZsh{} 将数据集的数据和标签分开}
\PY{n}{housing\PYZus{}labels} \PY{o}{=} \PY{n}{strat\PYZus{}train\PYZus{}set}\PY{p}{[}\PY{l+s+s1}{\PYZsq{}}\PY{l+s+s1}{median\PYZus{}house\PYZus{}value}\PY{l+s+s1}{\PYZsq{}}\PY{p}{]}\PY{o}{.}\PY{n}{copy}\PY{p}{(}\PY{p}{)}
\end{Verbatim}
\end{tcolorbox}

    \begin{tcolorbox}[breakable, size=fbox, boxrule=1pt, pad at break*=1mm,colback=cellbackground, colframe=cellborder]
\prompt{In}{incolor}{47}{\boxspacing}
\begin{Verbatim}[commandchars=\\\{\}]
\PY{n}{housing} \PY{o}{=} \PY{n}{strat\PYZus{}train\PYZus{}set}\PY{o}{.}\PY{n}{drop}\PY{p}{(}\PY{l+s+s1}{\PYZsq{}}\PY{l+s+s1}{median\PYZus{}house\PYZus{}value}\PY{l+s+s1}{\PYZsq{}}\PY{p}{,} \PY{n}{axis} \PY{o}{=} \PY{l+m+mi}{1}\PY{p}{)}
\PY{n}{housing}\PY{o}{.}\PY{n}{head}\PY{p}{(}\PY{p}{)}
\end{Verbatim}
\end{tcolorbox}

            \begin{tcolorbox}[breakable, size=fbox, boxrule=.5pt, pad at break*=1mm, opacityfill=0]
\prompt{Out}{outcolor}{47}{\boxspacing}
\begin{Verbatim}[commandchars=\\\{\}]
       longitude  latitude  housing\_median\_age  total\_rooms  total\_bedrooms  \textbackslash{}
5288     -118.47     34.05                27.0       4401.0          1033.0
12865    -121.34     38.69                17.0       1968.0           364.0
9174     -118.52     34.39                21.0       5477.0          1275.0
17247    -119.70     34.43                52.0       1364.0           460.0
14138    -117.07     32.74                37.0       1042.0           205.0

       population  households  median\_income ocean\_proximity income\_cat
5288       1725.0       962.0         4.1750       <1H OCEAN          3
12865       996.0       331.0         3.7031          INLAND          3
9174       3384.0      1222.0         3.6625       <1H OCEAN          3
17247       804.0       400.0         2.3750       <1H OCEAN          2
14138       589.0       208.0         2.6629      NEAR OCEAN          2
\end{Verbatim}
\end{tcolorbox}
        
    \begin{tcolorbox}[breakable, size=fbox, boxrule=1pt, pad at break*=1mm,colback=cellbackground, colframe=cellborder]
\prompt{In}{incolor}{48}{\boxspacing}
\begin{Verbatim}[commandchars=\\\{\}]
\PY{n}{housing}\PY{o}{.}\PY{n}{info}\PY{p}{(}\PY{p}{)}
\end{Verbatim}
\end{tcolorbox}

    \begin{Verbatim}[commandchars=\\\{\}]
<class 'pandas.core.frame.DataFrame'>
Int64Index: 16512 entries, 5288 to 18453
Data columns (total 10 columns):
longitude             16512 non-null float64
latitude              16512 non-null float64
housing\_median\_age    16512 non-null float64
total\_rooms           16512 non-null float64
total\_bedrooms        16348 non-null float64
population            16512 non-null float64
households            16512 non-null float64
median\_income         16512 non-null float64
ocean\_proximity       16512 non-null object
income\_cat            16512 non-null category
dtypes: category(1), float64(8), object(1)
memory usage: 1.3+ MB
    \end{Verbatim}

    \hypertarget{ux6570ux636eux6e05ux6d17}{%
\subsection{数据清洗}\label{ux6570ux636eux6e05ux6d17}}

\begin{itemize}
\tightlist
\item
  大多数机器学习算法无法在缺失的特征上工作,所以创建一些函数复制
\item
  total\_bedrooms有缺失,可以放弃这些区域或放弃这个属性
\item
  最好将缺失值设置为某个值(0,平均值或者中位数都可以)
\end{itemize}

    \begin{tcolorbox}[breakable, size=fbox, boxrule=1pt, pad at break*=1mm,colback=cellbackground, colframe=cellborder]
\prompt{In}{incolor}{49}{\boxspacing}
\begin{Verbatim}[commandchars=\\\{\}]
\PY{c+c1}{\PYZsh{} isnull() 会返回一个包含True\PYZbs{}False的列,any可以将其取出,axis=1按列取出}
\PY{n}{rows} \PY{o}{=} \PY{n}{housing}\PY{p}{[}\PY{n}{housing}\PY{o}{.}\PY{n}{isnull}\PY{p}{(}\PY{p}{)}\PY{o}{.}\PY{n}{any}\PY{p}{(}\PY{n}{axis} \PY{o}{=} \PY{l+m+mi}{1}\PY{p}{)}\PY{p}{]}
\PY{n}{rows}\PY{o}{.}\PY{n}{head}\PY{p}{(}\PY{p}{)}
\end{Verbatim}
\end{tcolorbox}

            \begin{tcolorbox}[breakable, size=fbox, boxrule=.5pt, pad at break*=1mm, opacityfill=0]
\prompt{Out}{outcolor}{49}{\boxspacing}
\begin{Verbatim}[commandchars=\\\{\}]
       longitude  latitude  housing\_median\_age  total\_rooms  total\_bedrooms  \textbackslash{}
19890    -119.15     36.29                18.0       1435.0             NaN
15607    -116.66     32.79                13.0        843.0             NaN
15479    -117.14     33.16                16.0       1660.0             NaN
10495    -117.66     33.51                18.0       2626.0             NaN
19060    -122.41     38.16                37.0       1549.0             NaN

       population  households  median\_income ocean\_proximity income\_cat
19890       657.0       254.0         2.4281          INLAND          2
15607       918.0       152.0         6.2152       <1H OCEAN          5
15479       733.0       214.0         5.6874       <1H OCEAN          4
10495      1302.0       522.0         4.0167       <1H OCEAN          3
19060       863.0       275.0         2.7457        NEAR BAY          2
\end{Verbatim}
\end{tcolorbox}
        
    \begin{tcolorbox}[breakable, size=fbox, boxrule=1pt, pad at break*=1mm,colback=cellbackground, colframe=cellborder]
\prompt{In}{incolor}{50}{\boxspacing}
\begin{Verbatim}[commandchars=\\\{\}]
\PY{c+c1}{\PYZsh{} \PYZsh{} 去除bedrooms为空的列}
\PY{c+c1}{\PYZsh{} rows.dropna(subset = [\PYZsq{}total\PYZus{}bedrooms\PYZsq{}])}
\PY{c+c1}{\PYZsh{} \PYZsh{} 放弃这个属性}
\PY{c+c1}{\PYZsh{} rows.drop(\PYZsq{}total\PYZus{}bedrooms\PYZsq{}, axis = 1)}

\PY{c+c1}{\PYZsh{} 指定值}
\PY{c+c1}{\PYZsh{} 创建一个imputer实例,指定你要用属性中的中位数替换该属性的缺失值}
\PY{k+kn}{from} \PY{n+nn}{sklearn}\PY{n+nn}{.}\PY{n+nn}{impute} \PY{k+kn}{import} \PY{n}{SimpleImputer}

\PY{n}{imputer} \PY{o}{=} \PY{n}{SimpleImputer}\PY{p}{(}\PY{n}{strategy} \PY{o}{=} \PY{l+s+s1}{\PYZsq{}}\PY{l+s+s1}{median}\PY{l+s+s1}{\PYZsq{}}\PY{p}{)}

\PY{n}{housing\PYZus{}num} \PY{o}{=} \PY{n}{housing}\PY{o}{.}\PY{n}{drop}\PY{p}{(}\PY{l+s+s1}{\PYZsq{}}\PY{l+s+s1}{ocean\PYZus{}proximity}\PY{l+s+s1}{\PYZsq{}}\PY{p}{,} \PY{n}{axis} \PY{o}{=} \PY{l+m+mi}{1}\PY{p}{)}  \PY{c+c1}{\PYZsh{} 删除非数字的列}

\PY{c+c1}{\PYZsh{} 使用fit()方法将imputer实例适配到训练集}
\PY{n}{imputer}\PY{o}{.}\PY{n}{fit}\PY{p}{(}\PY{n}{housing\PYZus{}num}\PY{p}{)}
\end{Verbatim}
\end{tcolorbox}

            \begin{tcolorbox}[breakable, size=fbox, boxrule=.5pt, pad at break*=1mm, opacityfill=0]
\prompt{Out}{outcolor}{50}{\boxspacing}
\begin{Verbatim}[commandchars=\\\{\}]
SimpleImputer(add\_indicator=False, copy=True, fill\_value=None,
              missing\_values=nan, strategy='median', verbose=0)
\end{Verbatim}
\end{tcolorbox}
        
    \begin{tcolorbox}[breakable, size=fbox, boxrule=1pt, pad at break*=1mm,colback=cellbackground, colframe=cellborder]
\prompt{In}{incolor}{51}{\boxspacing}
\begin{Verbatim}[commandchars=\\\{\}]
\PY{c+c1}{\PYZsh{} 查看到9个特征的中位数}
\PY{n}{imputer}\PY{o}{.}\PY{n}{statistics\PYZus{}}
\end{Verbatim}
\end{tcolorbox}

            \begin{tcolorbox}[breakable, size=fbox, boxrule=.5pt, pad at break*=1mm, opacityfill=0]
\prompt{Out}{outcolor}{51}{\boxspacing}
\begin{Verbatim}[commandchars=\\\{\}]
array([-118.49  ,   34.26  ,   29.    , 2123.5   ,  434.    , 1166.    ,
        409.    ,    3.5341,    3.    ])
\end{Verbatim}
\end{tcolorbox}
        
    \begin{tcolorbox}[breakable, size=fbox, boxrule=1pt, pad at break*=1mm,colback=cellbackground, colframe=cellborder]
\prompt{In}{incolor}{52}{\boxspacing}
\begin{Verbatim}[commandchars=\\\{\}]
\PY{c+c1}{\PYZsh{} 查看数据源的中位数}
\PY{n}{housing\PYZus{}num}\PY{o}{.}\PY{n}{median}\PY{p}{(}\PY{p}{)}\PY{o}{.}\PY{n}{values}
\end{Verbatim}
\end{tcolorbox}

            \begin{tcolorbox}[breakable, size=fbox, boxrule=.5pt, pad at break*=1mm, opacityfill=0]
\prompt{Out}{outcolor}{52}{\boxspacing}
\begin{Verbatim}[commandchars=\\\{\}]
array([-118.49  ,   34.26  ,   29.    , 2123.5   ,  434.    , 1166.    ,
        409.    ,    3.5341,    3.    ])
\end{Verbatim}
\end{tcolorbox}
        
    \begin{tcolorbox}[breakable, size=fbox, boxrule=1pt, pad at break*=1mm,colback=cellbackground, colframe=cellborder]
\prompt{In}{incolor}{53}{\boxspacing}
\begin{Verbatim}[commandchars=\\\{\}]
\PY{c+c1}{\PYZsh{} 转换数据集}
\PY{n}{X} \PY{o}{=} \PY{n}{imputer}\PY{o}{.}\PY{n}{transform}\PY{p}{(}\PY{n}{housing\PYZus{}num}\PY{p}{)}
\PY{n}{X}
\end{Verbatim}
\end{tcolorbox}

            \begin{tcolorbox}[breakable, size=fbox, boxrule=.5pt, pad at break*=1mm, opacityfill=0]
\prompt{Out}{outcolor}{53}{\boxspacing}
\begin{Verbatim}[commandchars=\\\{\}]
array([[-118.47  ,   34.05  ,   27.    , {\ldots},  962.    ,    4.175 ,
           3.    ],
       [-121.34  ,   38.69  ,   17.    , {\ldots},  331.    ,    3.7031,
           3.    ],
       [-118.52  ,   34.39  ,   21.    , {\ldots}, 1222.    ,    3.6625,
           3.    ],
       {\ldots},
       [-118.27  ,   33.94  ,   43.    , {\ldots},  340.    ,    1.6625,
           2.    ],
       [-122.21  ,   37.78  ,   43.    , {\ldots},  407.    ,    1.7188,
           2.    ],
       [-121.78  ,   37.22  ,   18.    , {\ldots},  402.    ,    5.0958,
           4.    ]])
\end{Verbatim}
\end{tcolorbox}
        
    \begin{tcolorbox}[breakable, size=fbox, boxrule=1pt, pad at break*=1mm,colback=cellbackground, colframe=cellborder]
\prompt{In}{incolor}{54}{\boxspacing}
\begin{Verbatim}[commandchars=\\\{\}]
\PY{c+c1}{\PYZsh{} 将X转为DataFrame}
\PY{n}{housing\PYZus{}tr} \PY{o}{=} \PY{n}{pd}\PY{o}{.}\PY{n}{DataFrame}\PY{p}{(}\PY{n}{X}\PY{p}{,} \PY{n}{columns} \PY{o}{=} \PY{n}{housing\PYZus{}num}\PY{o}{.}\PY{n}{columns}\PY{p}{,} \PY{n}{index} \PY{o}{=} \PY{n}{housing}\PY{o}{.}\PY{n}{index}\PY{p}{)}
\PY{n}{housing\PYZus{}tr}\PY{o}{.}\PY{n}{head}\PY{p}{(}\PY{p}{)}
\end{Verbatim}
\end{tcolorbox}

            \begin{tcolorbox}[breakable, size=fbox, boxrule=.5pt, pad at break*=1mm, opacityfill=0]
\prompt{Out}{outcolor}{54}{\boxspacing}
\begin{Verbatim}[commandchars=\\\{\}]
       longitude  latitude  housing\_median\_age  total\_rooms  total\_bedrooms  \textbackslash{}
5288     -118.47     34.05                27.0       4401.0          1033.0
12865    -121.34     38.69                17.0       1968.0           364.0
9174     -118.52     34.39                21.0       5477.0          1275.0
17247    -119.70     34.43                52.0       1364.0           460.0
14138    -117.07     32.74                37.0       1042.0           205.0

       population  households  median\_income  income\_cat
5288       1725.0       962.0         4.1750         3.0
12865       996.0       331.0         3.7031         3.0
9174       3384.0      1222.0         3.6625         3.0
17247       804.0       400.0         2.3750         2.0
14138       589.0       208.0         2.6629         2.0
\end{Verbatim}
\end{tcolorbox}
        
    此时会发现 - total\_bedrooms的空值已经被替换为了中位数 -
非数据集的列被删除了

    \begin{tcolorbox}[breakable, size=fbox, boxrule=1pt, pad at break*=1mm,colback=cellbackground, colframe=cellborder]
\prompt{In}{incolor}{55}{\boxspacing}
\begin{Verbatim}[commandchars=\\\{\}]
\PY{n}{housing\PYZus{}tr}\PY{o}{.}\PY{n}{info}\PY{p}{(}\PY{p}{)}
\end{Verbatim}
\end{tcolorbox}

    \begin{Verbatim}[commandchars=\\\{\}]
<class 'pandas.core.frame.DataFrame'>
Int64Index: 16512 entries, 5288 to 18453
Data columns (total 9 columns):
longitude             16512 non-null float64
latitude              16512 non-null float64
housing\_median\_age    16512 non-null float64
total\_rooms           16512 non-null float64
total\_bedrooms        16512 non-null float64
population            16512 non-null float64
households            16512 non-null float64
median\_income         16512 non-null float64
income\_cat            16512 non-null float64
dtypes: float64(9)
memory usage: 1.3 MB
    \end{Verbatim}

    \hypertarget{scikit-learn}{%
\subsection{scikit-Learn}\label{scikit-learn}}

\begin{itemize}
\tightlist
\item
  一致性

  \begin{enumerate}
  \def\labelenumi{\arabic{enumi}.}
  \tightlist
  \item
    估算器:各种机器学习算法,均可执行fit()执行估算器
  \item
    转换器
  \end{enumerate}

  \begin{itemize}
  \tightlist
  \item
    LabelBinarizer
  \item
    transform() 执行转换数据集
  \item
    fit\_transform() 先估算,再转换
  \end{itemize}

  \begin{enumerate}
  \def\labelenumi{\arabic{enumi}.}
  \setcounter{enumi}{2}
  \tightlist
  \item
    预测器
  \end{enumerate}

  \begin{itemize}
  \tightlist
  \item
    predict() 对给定的心数据集进行预测
  \item
    score() 评估测试集的预测质量
  \end{itemize}
\item
  检查

  \begin{itemize}
  \tightlist
  \item
    imputer.strategy\_ 学习参数通过公共实例变量访问
  \end{itemize}
\end{itemize}

    \hypertarget{ux5904ux7406ux6587ux672cux548cux5206ux7c7bux5c5eux6027ux8f6cux6362ux5668}{%
\subsection{处理文本和分类属性------转换器}\label{ux5904ux7406ux6587ux672cux548cux5206ux7c7bux5c5eux6027ux8f6cux6362ux5668}}

    \begin{tcolorbox}[breakable, size=fbox, boxrule=1pt, pad at break*=1mm,colback=cellbackground, colframe=cellborder]
\prompt{In}{incolor}{120}{\boxspacing}
\begin{Verbatim}[commandchars=\\\{\}]
\PY{n}{housing\PYZus{}cat} \PY{o}{=} \PY{n}{housing}\PY{p}{[}\PY{p}{[}\PY{l+s+s1}{\PYZsq{}}\PY{l+s+s1}{ocean\PYZus{}proximity}\PY{l+s+s1}{\PYZsq{}}\PY{p}{]}\PY{p}{]} \PY{c+c1}{\PYZsh{} DataFrame}
\PY{n}{housing\PYZus{}cat}\PY{o}{.}\PY{n}{head}\PY{p}{(}\PY{p}{)}
\end{Verbatim}
\end{tcolorbox}

            \begin{tcolorbox}[breakable, size=fbox, boxrule=.5pt, pad at break*=1mm, opacityfill=0]
\prompt{Out}{outcolor}{120}{\boxspacing}
\begin{Verbatim}[commandchars=\\\{\}]
      ocean\_proximity
5288        <1H OCEAN
12865          INLAND
9174        <1H OCEAN
17247       <1H OCEAN
14138      NEAR OCEAN
\end{Verbatim}
\end{tcolorbox}
        
    \begin{tcolorbox}[breakable, size=fbox, boxrule=1pt, pad at break*=1mm,colback=cellbackground, colframe=cellborder]
\prompt{In}{incolor}{57}{\boxspacing}
\begin{Verbatim}[commandchars=\\\{\}]
\PY{c+c1}{\PYZsh{} 查看不同类别值的数量}
\PY{n}{housing}\PY{p}{[}\PY{l+s+s1}{\PYZsq{}}\PY{l+s+s1}{ocean\PYZus{}proximity}\PY{l+s+s1}{\PYZsq{}}\PY{p}{]}\PY{o}{.}\PY{n}{hist}\PY{p}{(}\PY{p}{)}
\end{Verbatim}
\end{tcolorbox}

            \begin{tcolorbox}[breakable, size=fbox, boxrule=.5pt, pad at break*=1mm, opacityfill=0]
\prompt{Out}{outcolor}{57}{\boxspacing}
\begin{Verbatim}[commandchars=\\\{\}]
<matplotlib.axes.\_subplots.AxesSubplot at 0x23b84f11fc8>
\end{Verbatim}
\end{tcolorbox}
        
    \begin{center}
    \adjustimage{max size={0.9\linewidth}{0.9\paperheight}}{特征工程_files/特征工程_74_1.png}
    \end{center}
    { \hspace*{\fill} \\}
    
    \begin{tcolorbox}[breakable, size=fbox, boxrule=1pt, pad at break*=1mm,colback=cellbackground, colframe=cellborder]
\prompt{In}{incolor}{58}{\boxspacing}
\begin{Verbatim}[commandchars=\\\{\}]
\PY{c+c1}{\PYZsh{} 查看不同类别值的数量}
\PY{n}{housing}\PY{p}{[}\PY{l+s+s1}{\PYZsq{}}\PY{l+s+s1}{ocean\PYZus{}proximity}\PY{l+s+s1}{\PYZsq{}}\PY{p}{]}\PY{o}{.}\PY{n}{value\PYZus{}counts}\PY{p}{(}\PY{p}{)}
\end{Verbatim}
\end{tcolorbox}

            \begin{tcolorbox}[breakable, size=fbox, boxrule=.5pt, pad at break*=1mm, opacityfill=0]
\prompt{Out}{outcolor}{58}{\boxspacing}
\begin{Verbatim}[commandchars=\\\{\}]
<1H OCEAN     7307
INLAND        5249
NEAR OCEAN    2143
NEAR BAY      1810
ISLAND           3
Name: ocean\_proximity, dtype: int64
\end{Verbatim}
\end{tcolorbox}
        
    \hypertarget{labelencoderux6587ux672cux8f6cux6570ux5b57ux5206ux7c7b}{%
\subsubsection{LabelEncoder------文本转数字分类}\label{labelencoderux6587ux672cux8f6cux6570ux5b57ux5206ux7c7b}}

    \begin{tcolorbox}[breakable, size=fbox, boxrule=1pt, pad at break*=1mm,colback=cellbackground, colframe=cellborder]
\prompt{In}{incolor}{121}{\boxspacing}
\begin{Verbatim}[commandchars=\\\{\}]
\PY{c+c1}{\PYZsh{} 将文本转换为对应的数字分类,使用LabelEncoder}
\PY{k+kn}{from} \PY{n+nn}{sklearn} \PY{k+kn}{import} \PY{n}{preprocessing}
\PY{n}{encoder} \PY{o}{=} \PY{n}{preprocessing}\PY{o}{.}\PY{n}{LabelEncoder}\PY{p}{(}\PY{p}{)} \PY{c+c1}{\PYZsh{} 创建一个LabelEncoder()对象}
\PY{n}{housing\PYZus{}cat} \PY{o}{=} \PY{n}{housing}\PY{p}{[}\PY{l+s+s1}{\PYZsq{}}\PY{l+s+s1}{ocean\PYZus{}proximity}\PY{l+s+s1}{\PYZsq{}}\PY{p}{]} \PY{c+c1}{\PYZsh{} 取文本列}
\PY{n}{housing\PYZus{}cat\PYZus{}encoder} \PY{o}{=} \PY{n}{encoder}\PY{o}{.}\PY{n}{fit\PYZus{}transform}\PY{p}{(}\PY{n}{housing\PYZus{}cat}\PY{p}{)} \PY{c+c1}{\PYZsh{} fit\PYZus{}transform}
\PY{n}{housing\PYZus{}cat\PYZus{}encoder} 
\end{Verbatim}
\end{tcolorbox}

            \begin{tcolorbox}[breakable, size=fbox, boxrule=.5pt, pad at break*=1mm, opacityfill=0]
\prompt{Out}{outcolor}{121}{\boxspacing}
\begin{Verbatim}[commandchars=\\\{\}]
array([0, 1, 0, {\ldots}, 0, 3, 0])
\end{Verbatim}
\end{tcolorbox}
        
    \begin{tcolorbox}[breakable, size=fbox, boxrule=1pt, pad at break*=1mm,colback=cellbackground, colframe=cellborder]
\prompt{In}{incolor}{60}{\boxspacing}
\begin{Verbatim}[commandchars=\\\{\}]
\PY{n}{encoder}\PY{o}{.}\PY{n}{classes\PYZus{}}  \PY{c+c1}{\PYZsh{} 返回编码器中所有类别,已经排除了重复项}
\end{Verbatim}
\end{tcolorbox}

            \begin{tcolorbox}[breakable, size=fbox, boxrule=.5pt, pad at break*=1mm, opacityfill=0]
\prompt{Out}{outcolor}{60}{\boxspacing}
\begin{Verbatim}[commandchars=\\\{\}]
array(['<1H OCEAN', 'INLAND', 'ISLAND', 'NEAR BAY', 'NEAR OCEAN'],
      dtype=object)
\end{Verbatim}
\end{tcolorbox}
        
    \hypertarget{onehotencoderux6570ux5b57ux5206ux7c7bux8f6cux72ecux70edux7f16ux7801}{%
\subsubsection{OneHotEncoder------数字分类转独热编码}\label{onehotencoderux6570ux5b57ux5206ux7c7bux8f6cux72ecux70edux7f16ux7801}}

    \begin{tcolorbox}[breakable, size=fbox, boxrule=1pt, pad at break*=1mm,colback=cellbackground, colframe=cellborder]
\prompt{In}{incolor}{61}{\boxspacing}
\begin{Verbatim}[commandchars=\\\{\}]
\PY{c+c1}{\PYZsh{} 取OneHotEncoder}
\PY{n}{encoder} \PY{o}{=} \PY{n}{preprocessing}\PY{o}{.}\PY{n}{OneHotEncoder}\PY{p}{(}\PY{p}{)}
\end{Verbatim}
\end{tcolorbox}

    \begin{tcolorbox}[breakable, size=fbox, boxrule=1pt, pad at break*=1mm,colback=cellbackground, colframe=cellborder]
\prompt{In}{incolor}{62}{\boxspacing}
\begin{Verbatim}[commandchars=\\\{\}]
\PY{n}{housing\PYZus{}cat\PYZus{}encoder}\PY{o}{.}\PY{n}{reshape}\PY{p}{(}\PY{o}{\PYZhy{}}\PY{l+m+mi}{1}\PY{p}{,} \PY{l+m+mi}{1}\PY{p}{)} \PY{c+c1}{\PYZsh{} 数组转矩阵}
\end{Verbatim}
\end{tcolorbox}

            \begin{tcolorbox}[breakable, size=fbox, boxrule=.5pt, pad at break*=1mm, opacityfill=0]
\prompt{Out}{outcolor}{62}{\boxspacing}
\begin{Verbatim}[commandchars=\\\{\}]
array([[0],
       [1],
       [0],
       {\ldots},
       [0],
       [3],
       [0]])
\end{Verbatim}
\end{tcolorbox}
        
    \begin{tcolorbox}[breakable, size=fbox, boxrule=1pt, pad at break*=1mm,colback=cellbackground, colframe=cellborder]
\prompt{In}{incolor}{63}{\boxspacing}
\begin{Verbatim}[commandchars=\\\{\}]
\PY{n}{housing\PYZus{}cat\PYZus{}hot} \PY{o}{=} \PY{n}{encoder}\PY{o}{.}\PY{n}{fit\PYZus{}transform}\PY{p}{(}\PY{n}{housing\PYZus{}cat\PYZus{}encoder}\PY{o}{.}\PY{n}{reshape}\PY{p}{(}\PY{o}{\PYZhy{}}\PY{l+m+mi}{1}\PY{p}{,} \PY{l+m+mi}{1}\PY{p}{)}\PY{p}{)}
\PY{n}{housing\PYZus{}cat\PYZus{}hot}  \PY{c+c1}{\PYZsh{} 稀疏矩阵}
\end{Verbatim}
\end{tcolorbox}

            \begin{tcolorbox}[breakable, size=fbox, boxrule=.5pt, pad at break*=1mm, opacityfill=0]
\prompt{Out}{outcolor}{63}{\boxspacing}
\begin{Verbatim}[commandchars=\\\{\}]
<16512x5 sparse matrix of type '<class 'numpy.float64'>'
        with 16512 stored elements in Compressed Sparse Row format>
\end{Verbatim}
\end{tcolorbox}
        
    \begin{tcolorbox}[breakable, size=fbox, boxrule=1pt, pad at break*=1mm,colback=cellbackground, colframe=cellborder]
\prompt{In}{incolor}{64}{\boxspacing}
\begin{Verbatim}[commandchars=\\\{\}]
\PY{n}{housing\PYZus{}cat\PYZus{}hot}\PY{o}{.}\PY{n}{toarray}\PY{p}{(}\PY{p}{)}  \PY{c+c1}{\PYZsh{} 稀疏矩阵转numpy.array}
\end{Verbatim}
\end{tcolorbox}

            \begin{tcolorbox}[breakable, size=fbox, boxrule=.5pt, pad at break*=1mm, opacityfill=0]
\prompt{Out}{outcolor}{64}{\boxspacing}
\begin{Verbatim}[commandchars=\\\{\}]
array([[1., 0., 0., 0., 0.],
       [0., 1., 0., 0., 0.],
       [1., 0., 0., 0., 0.],
       {\ldots},
       [1., 0., 0., 0., 0.],
       [0., 0., 0., 1., 0.],
       [1., 0., 0., 0., 0.]])
\end{Verbatim}
\end{tcolorbox}
        
    \hypertarget{labelbinarizerux4e00ux6b21ux5b8cux6210ux4e24ux4e2aux8f6cux6362-ux6587ux672cux6574ux6570ux7c7bux578b-ux72ecux70edux7c7bux578b}{%
\subsubsection{LabelBinarizer------一次完成两个转换
文本--整数类型-独热类型}\label{labelbinarizerux4e00ux6b21ux5b8cux6210ux4e24ux4e2aux8f6cux6362-ux6587ux672cux6574ux6570ux7c7bux578b-ux72ecux70edux7c7bux578b}}

    \begin{tcolorbox}[breakable, size=fbox, boxrule=1pt, pad at break*=1mm,colback=cellbackground, colframe=cellborder]
\prompt{In}{incolor}{112}{\boxspacing}
\begin{Verbatim}[commandchars=\\\{\}]
\PY{k+kn}{from} \PY{n+nn}{sklearn}\PY{n+nn}{.}\PY{n+nn}{preprocessing} \PY{k+kn}{import} \PY{n}{LabelBinarizer}
\PY{n}{encoder} \PY{o}{=} \PY{n}{preprocessing}\PY{o}{.}\PY{n}{LabelBinarizer}\PY{p}{(}\PY{p}{)}
\PY{n}{housing\PYZus{}cat\PYZus{}1hot} \PY{o}{=} \PY{n}{encoder}\PY{o}{.}\PY{n}{fit\PYZus{}transform}\PY{p}{(}\PY{n}{housing\PYZus{}cat}\PY{p}{)}
\PY{n}{housing\PYZus{}cat\PYZus{}1hot}
\end{Verbatim}
\end{tcolorbox}

            \begin{tcolorbox}[breakable, size=fbox, boxrule=.5pt, pad at break*=1mm, opacityfill=0]
\prompt{Out}{outcolor}{112}{\boxspacing}
\begin{Verbatim}[commandchars=\\\{\}]
array([[1, 0, 0, 0, 0],
       [0, 1, 0, 0, 0],
       [1, 0, 0, 0, 0],
       {\ldots},
       [1, 0, 0, 0, 0],
       [0, 0, 0, 1, 0],
       [1, 0, 0, 0, 0]])
\end{Verbatim}
\end{tcolorbox}
        
    \begin{tcolorbox}[breakable, size=fbox, boxrule=1pt, pad at break*=1mm,colback=cellbackground, colframe=cellborder]
\prompt{In}{incolor}{110}{\boxspacing}
\begin{Verbatim}[commandchars=\\\{\}]
\PY{c+c1}{\PYZsh{} 输出稀疏矩阵}
\PY{n}{encoder} \PY{o}{=} \PY{n}{preprocessing}\PY{o}{.}\PY{n}{LabelBinarizer}\PY{p}{(}\PY{n}{sparse\PYZus{}output} \PY{o}{=} \PY{k+kc}{True}\PY{p}{)}
\PY{n}{housing\PYZus{}cat\PYZus{}1hot} \PY{o}{=} \PY{n}{encoder}\PY{o}{.}\PY{n}{fit\PYZus{}transform}\PY{p}{(}\PY{n}{housing\PYZus{}cat\PYZus{}encoder}\PY{o}{.}\PY{n}{reshape}\PY{p}{(}\PY{o}{\PYZhy{}}\PY{l+m+mi}{1}\PY{p}{,} \PY{l+m+mi}{1}\PY{p}{)}\PY{p}{)}
\PY{n}{housing\PYZus{}cat\PYZus{}1hot}  \PY{c+c1}{\PYZsh{} 稀疏矩阵}
\end{Verbatim}
\end{tcolorbox}

            \begin{tcolorbox}[breakable, size=fbox, boxrule=.5pt, pad at break*=1mm, opacityfill=0]
\prompt{Out}{outcolor}{110}{\boxspacing}
\begin{Verbatim}[commandchars=\\\{\}]
<16512x5 sparse matrix of type '<class 'numpy.int32'>'
        with 16512 stored elements in Compressed Sparse Row format>
\end{Verbatim}
\end{tcolorbox}
        
    \hypertarget{ux81eaux5b9aux4e49ux8f6cux6362ux5668}{%
\subsection{自定义转换器}\label{ux81eaux5b9aux4e49ux8f6cux6362ux5668}}

\begin{itemize}
\tightlist
\item
  重写fit()和transfrom()
\end{itemize}

    \begin{tcolorbox}[breakable, size=fbox, boxrule=1pt, pad at break*=1mm,colback=cellbackground, colframe=cellborder]
\prompt{In}{incolor}{122}{\boxspacing}
\begin{Verbatim}[commandchars=\\\{\}]
\PY{c+c1}{\PYZsh{} 至少继承这两个类}
\PY{k+kn}{from} \PY{n+nn}{sklearn}\PY{n+nn}{.}\PY{n+nn}{base} \PY{k+kn}{import} \PY{n}{BaseEstimator}\PY{p}{,} \PY{n}{TransformerMixin}

\PY{c+c1}{\PYZsh{} 列表生成式}
\PY{n}{rooms\PYZus{}ix}\PY{p}{,} \PY{n}{bedrooms\PYZus{}ix}\PY{p}{,} \PY{n}{population\PYZus{}ix}\PY{p}{,} \PY{n}{household\PYZus{}ix} \PY{o}{=} \PY{p}{[}\PY{n+nb}{list}\PY{p}{(}\PY{n}{housing}\PY{o}{.}\PY{n}{columns}\PY{p}{)}\PY{o}{.}\PY{n}{index}\PY{p}{(}\PY{n}{col}\PY{p}{)} \PY{k}{for} \PY{n}{col} \PY{o+ow}{in} \PY{p}{(}\PY{l+s+s2}{\PYZdq{}}\PY{l+s+s2}{total\PYZus{}rooms}\PY{l+s+s2}{\PYZdq{}}\PY{p}{,} \PY{l+s+s2}{\PYZdq{}}\PY{l+s+s2}{total\PYZus{}bedrooms}\PY{l+s+s2}{\PYZdq{}}\PY{p}{,} \PY{l+s+s2}{\PYZdq{}}\PY{l+s+s2}{population}\PY{l+s+s2}{\PYZdq{}}\PY{p}{,} \PY{l+s+s2}{\PYZdq{}}\PY{l+s+s2}{households}\PY{l+s+s2}{\PYZdq{}}\PY{p}{)}\PY{p}{]}
\PY{n}{rooms\PYZus{}ix}\PY{p}{,} \PY{n}{bedrooms\PYZus{}ix}\PY{p}{,} \PY{n}{population\PYZus{}ix}\PY{p}{,} \PY{n}{household\PYZus{}ix} 
\PY{c+c1}{\PYZsh{} 继承BaseEstimator, TransformerMixin,重写fit()和transfrom()}
\PY{c+c1}{\PYZsh{} 不同特征的组合}
\PY{k}{class} \PY{n+nc}{CombinedAttributesAdder}\PY{p}{(}\PY{n}{BaseEstimator}\PY{p}{,} \PY{n}{TransformerMixin}\PY{p}{)}\PY{p}{:}
    \PY{k}{def} \PY{n+nf+fm}{\PYZus{}\PYZus{}init\PYZus{}\PYZus{}}\PY{p}{(}\PY{n+nb+bp}{self}\PY{p}{,} \PY{n}{add\PYZus{}bedrooms\PYZus{}per\PYZus{}room} \PY{o}{=} \PY{k+kc}{True}\PY{p}{)}\PY{p}{:}
        \PY{n+nb+bp}{self}\PY{o}{.}\PY{n}{add\PYZus{}bedrooms\PYZus{}per\PYZus{}room} \PY{o}{=} \PY{n}{add\PYZus{}bedrooms\PYZus{}per\PYZus{}room}
    \PY{k}{def} \PY{n+nf}{fit}\PY{p}{(}\PY{n+nb+bp}{self}\PY{p}{,} \PY{n}{X}\PY{p}{,} \PY{n}{y} \PY{o}{=} \PY{k+kc}{None}\PY{p}{)}\PY{p}{:}
        \PY{k}{return} \PY{n+nb+bp}{self}
    \PY{k}{def} \PY{n+nf}{transform}\PY{p}{(}\PY{n+nb+bp}{self}\PY{p}{,} \PY{n}{X}\PY{p}{,} \PY{n}{y} \PY{o}{=} \PY{k+kc}{None}\PY{p}{)}\PY{p}{:}
        \PY{n}{rooms\PYZus{}per\PYZus{}household} \PY{o}{=} \PY{n}{X}\PY{p}{[}\PY{p}{:}\PY{p}{,} \PY{n}{rooms\PYZus{}ix}\PY{p}{]} \PY{o}{/} \PY{n}{X}\PY{p}{[}\PY{p}{:}\PY{p}{,} \PY{n}{household\PYZus{}ix}\PY{p}{]}
        \PY{n}{population\PYZus{}per\PYZus{}household} \PY{o}{=} \PY{n}{X}\PY{p}{[}\PY{p}{:}\PY{p}{,} \PY{n}{population\PYZus{}ix}\PY{p}{]} \PY{o}{/} \PY{n}{X}\PY{p}{[}\PY{p}{:}\PY{p}{,} \PY{n}{household\PYZus{}ix}\PY{p}{]}
        \PY{k}{if} \PY{n+nb+bp}{self}\PY{o}{.}\PY{n}{add\PYZus{}bedrooms\PYZus{}per\PYZus{}room}\PY{p}{:}
            \PY{n}{bedrooms\PYZus{}per\PYZus{}room} \PY{o}{=} \PY{n}{X}\PY{p}{[}\PY{p}{:}\PY{p}{,} \PY{n}{bedrooms\PYZus{}ix}\PY{p}{]} \PY{o}{/} \PY{n}{X}\PY{p}{[}\PY{p}{:}\PY{p}{,} \PY{n}{rooms\PYZus{}ix}\PY{p}{]}
            \PY{k}{return} \PY{n}{np}\PY{o}{.}\PY{n}{c\PYZus{}}\PY{p}{[}\PY{n}{X}\PY{p}{,} \PY{n}{rooms\PYZus{}per\PYZus{}household}\PY{p}{,}\PY{n}{population\PYZus{}per\PYZus{}household}\PY{p}{,} \PY{n}{bedrooms\PYZus{}per\PYZus{}room}\PY{p}{]}
        \PY{k}{else}\PY{p}{:}
            \PY{k}{return} \PY{n}{np}\PY{o}{.}\PY{n}{c\PYZus{}}\PY{p}{[}\PY{n}{X}\PY{p}{,} \PY{n}{rooms\PYZus{}per\PYZus{}household}\PY{p}{,}\PY{n}{population\PYZus{}per\PYZus{}household}\PY{p}{]}
        
        
        
\PY{n}{attr\PYZus{}adder} \PY{o}{=} \PY{n}{CombinedAttributesAdder}\PY{p}{(}\PY{n}{add\PYZus{}bedrooms\PYZus{}per\PYZus{}room} \PY{o}{=} \PY{k+kc}{False}\PY{p}{)}
\PY{n}{housing\PYZus{}extra\PYZus{}attribs} \PY{o}{=} \PY{n}{attr\PYZus{}adder}\PY{o}{.}\PY{n}{transform}\PY{p}{(}\PY{n}{housing}\PY{o}{.}\PY{n}{values}\PY{p}{)}  \PY{c+c1}{\PYZsh{} 传入numpy矩阵}
\end{Verbatim}
\end{tcolorbox}

    \begin{tcolorbox}[breakable, size=fbox, boxrule=1pt, pad at break*=1mm,colback=cellbackground, colframe=cellborder]
\prompt{In}{incolor}{68}{\boxspacing}
\begin{Verbatim}[commandchars=\\\{\}]
\PY{n}{housing\PYZus{}extra\PYZus{}attribs}
\end{Verbatim}
\end{tcolorbox}

            \begin{tcolorbox}[breakable, size=fbox, boxrule=.5pt, pad at break*=1mm, opacityfill=0]
\prompt{Out}{outcolor}{68}{\boxspacing}
\begin{Verbatim}[commandchars=\\\{\}]
array([[-118.47, 34.05, 27.0, {\ldots}, 3, 4.574844074844075,
        1.793139293139293],
       [-121.34, 38.69, 17.0, {\ldots}, 3, 5.945619335347432,
        3.009063444108761],
       [-118.52, 34.39, 21.0, {\ldots}, 3, 4.481996726677577,
        2.769230769230769],
       {\ldots},
       [-118.27, 33.94, 43.0, {\ldots}, 2, 3.85, 3.476470588235294],
       [-122.21, 37.78, 43.0, {\ldots}, 2, 4.181818181818182,
        3.0147420147420148],
       [-121.78, 37.22, 18.0, {\ldots}, 4, 5.291044776119403,
        3.8482587064676617]], dtype=object)
\end{Verbatim}
\end{tcolorbox}
        
    \begin{tcolorbox}[breakable, size=fbox, boxrule=1pt, pad at break*=1mm,colback=cellbackground, colframe=cellborder]
\prompt{In}{incolor}{69}{\boxspacing}
\begin{Verbatim}[commandchars=\\\{\}]
\PY{n}{housing\PYZus{}tr} \PY{o}{=} \PY{n}{pd}\PY{o}{.}\PY{n}{DataFrame}\PY{p}{(}\PY{n}{housing\PYZus{}extra\PYZus{}attribs}\PY{p}{)}  \PY{c+c1}{\PYZsh{} 转DataFrame}
\PY{n}{housing\PYZus{}tr}\PY{o}{.}\PY{n}{head}\PY{p}{(}\PY{p}{)}
\end{Verbatim}
\end{tcolorbox}

            \begin{tcolorbox}[breakable, size=fbox, boxrule=.5pt, pad at break*=1mm, opacityfill=0]
\prompt{Out}{outcolor}{69}{\boxspacing}
\begin{Verbatim}[commandchars=\\\{\}]
        0      1   2     3     4     5     6       7           8  9       10  \textbackslash{}
0 -118.47  34.05  27  4401  1033  1725   962   4.175   <1H OCEAN  3  4.57484
1 -121.34  38.69  17  1968   364   996   331  3.7031      INLAND  3  5.94562
2 -118.52  34.39  21  5477  1275  3384  1222  3.6625   <1H OCEAN  3    4.482
3  -119.7  34.43  52  1364   460   804   400   2.375   <1H OCEAN  2     3.41
4 -117.07  32.74  37  1042   205   589   208  2.6629  NEAR OCEAN  2  5.00962

        11
0  1.79314
1  3.00906
2  2.76923
3     2.01
4  2.83173
\end{Verbatim}
\end{tcolorbox}
        
    \begin{tcolorbox}[breakable, size=fbox, boxrule=1pt, pad at break*=1mm,colback=cellbackground, colframe=cellborder]
\prompt{In}{incolor}{70}{\boxspacing}
\begin{Verbatim}[commandchars=\\\{\}]
\PY{c+c1}{\PYZsh{} 转DataFrame}
\PY{n}{housing\PYZus{}extra\PYZus{}attribs} \PY{o}{=} \PY{n}{pd}\PY{o}{.}\PY{n}{DataFrame}\PY{p}{(}
    \PY{n}{housing\PYZus{}extra\PYZus{}attribs}\PY{p}{,}
    \PY{n}{columns}\PY{o}{=}\PY{n+nb}{list}\PY{p}{(}\PY{n}{housing}\PY{o}{.}\PY{n}{columns}\PY{p}{)}\PY{o}{+}\PY{p}{[}\PY{l+s+s2}{\PYZdq{}}\PY{l+s+s2}{rooms\PYZus{}per\PYZus{}household}\PY{l+s+s2}{\PYZdq{}}\PY{p}{,} \PY{l+s+s2}{\PYZdq{}}\PY{l+s+s2}{population\PYZus{}per\PYZus{}household}\PY{l+s+s2}{\PYZdq{}}\PY{p}{]}\PY{p}{,}
    \PY{n}{index}\PY{o}{=}\PY{n}{housing}\PY{o}{.}\PY{n}{index}\PY{p}{)}
\PY{n}{housing\PYZus{}extra\PYZus{}attribs}\PY{o}{.}\PY{n}{head}\PY{p}{(}\PY{p}{)}
\end{Verbatim}
\end{tcolorbox}

            \begin{tcolorbox}[breakable, size=fbox, boxrule=.5pt, pad at break*=1mm, opacityfill=0]
\prompt{Out}{outcolor}{70}{\boxspacing}
\begin{Verbatim}[commandchars=\\\{\}]
      longitude latitude housing\_median\_age total\_rooms total\_bedrooms  \textbackslash{}
5288    -118.47    34.05                 27        4401           1033
12865   -121.34    38.69                 17        1968            364
9174    -118.52    34.39                 21        5477           1275
17247    -119.7    34.43                 52        1364            460
14138   -117.07    32.74                 37        1042            205

      population households median\_income ocean\_proximity income\_cat  \textbackslash{}
5288        1725        962         4.175       <1H OCEAN          3
12865        996        331        3.7031          INLAND          3
9174        3384       1222        3.6625       <1H OCEAN          3
17247        804        400         2.375       <1H OCEAN          2
14138        589        208        2.6629      NEAR OCEAN          2

      rooms\_per\_household population\_per\_household
5288              4.57484                  1.79314
12865             5.94562                  3.00906
9174                4.482                  2.76923
17247                3.41                     2.01
14138             5.00962                  2.83173
\end{Verbatim}
\end{tcolorbox}
        
    \begin{tcolorbox}[breakable, size=fbox, boxrule=1pt, pad at break*=1mm,colback=cellbackground, colframe=cellborder]
\prompt{In}{incolor}{71}{\boxspacing}
\begin{Verbatim}[commandchars=\\\{\}]
\PY{n}{housing\PYZus{}extra\PYZus{}attribs}\PY{o}{.}\PY{n}{sample}\PY{p}{(}\PY{l+m+mi}{20}\PY{p}{)}
\end{Verbatim}
\end{tcolorbox}

            \begin{tcolorbox}[breakable, size=fbox, boxrule=.5pt, pad at break*=1mm, opacityfill=0]
\prompt{Out}{outcolor}{71}{\boxspacing}
\begin{Verbatim}[commandchars=\\\{\}]
      longitude latitude housing\_median\_age total\_rooms total\_bedrooms  \textbackslash{}
12630    -121.5    38.49                 29        3606            690
4951    -118.31    34.02                 41        1046            216
8186     -118.1    33.79                 36        3359            596
17801   -121.79    37.38                 22        3650            527
11244      -118    33.81                 22        2642            640
10046   -120.99    39.22                 16        1497            275
15894   -122.39    37.73                 52        1070            224
9127    -118.08    34.58                 12        3851            857
16368   -121.31    38.01                 22        2101            514
11041   -117.81    33.82                 22        2898            335
19187   -122.66    38.45                 26        2081            339
1176    -121.55     39.5                 26        3215            827
15719   -122.44    37.78                 39        1181            310
11632   -118.03    33.82                 20        2662            464
8568    -118.42     33.9                 29        1929            523
20615   -121.54    39.08                 23        1076            216
20452   -118.81    34.28                 20        3678            684
1841    -122.29    37.91                 46        2085            346
11296   -117.92    33.79                 26        2737            614
17532   -121.89    37.34                 20        1106            494

      population households median\_income ocean\_proximity income\_cat  \textbackslash{}
12630       2317        696        2.7368          INLAND          2
4951         727        201        1.6667       <1H OCEAN          2
8186        1522        565        5.1805       <1H OCEAN          4
17801       1637        520        5.3774       <1H OCEAN          4
11244       1702        588        3.5268       <1H OCEAN          3
10046        737        243        2.8942          INLAND          2
15894        567        207        2.8603        NEAR BAY          2
9127        2169        811        3.0101          INLAND          3
16368       1304        511        2.8348          INLAND          2
11041       1057        324       10.8111       <1H OCEAN          5
19187        906        323        4.4375       <1H OCEAN          3
1176        2041        737        1.0585          INLAND          1
15719        901        281        1.4866        NEAR BAY          1
11632       1275        472        6.0162       <1H OCEAN          5
8568         686        455        5.5347       <1H OCEAN          4
20615        724        197        2.3598          INLAND          2
20452       1882        694        4.1607       <1H OCEAN          3
1841         748        354        4.0536        NEAR BAY          3
11296       1877        606        2.8622       <1H OCEAN          2
17532        851        448        0.8894       <1H OCEAN          1

      rooms\_per\_household population\_per\_household
12630             5.18103                  3.32902
4951              5.20398                  3.61692
8186              5.94513                  2.69381
17801             7.01923                  3.14808
11244              4.4932                  2.89456
10046             6.16049                  3.03292
15894             5.16908                  2.73913
9127              4.74846                  2.67448
16368             4.11155                  2.55186
11041             8.94444                  3.26235
19187             6.44272                  2.80495
1176              4.36228                  2.76934
15719             4.20285                  3.20641
11632             5.63983                  2.70127
8568              4.23956                  1.50769
20615             5.46193                  3.67513
20452             5.29971                  2.71182
1841              5.88983                  2.11299
11296              4.5165                  3.09736
17532             2.46875                  1.89955
\end{Verbatim}
\end{tcolorbox}
        
    \hypertarget{ux7279ux5f81ux7f29ux653e}{%
\subsection{特征缩放}\label{ux7279ux5f81ux7f29ux653e}}

    \begin{tcolorbox}[breakable, size=fbox, boxrule=1pt, pad at break*=1mm,colback=cellbackground, colframe=cellborder]
\prompt{In}{incolor}{72}{\boxspacing}
\begin{Verbatim}[commandchars=\\\{\}]
\PY{c+c1}{\PYZsh{} 最重要也是最需要应用在数据上的转换器,就是特征缩放,输入数值属性有很大的比例差异,会导致机器学习算法性能表现不佳}
\PY{c+c1}{\PYZsh{} 房间总数 范围6到39320,收入中位数范围0到15}
\PY{c+c1}{\PYZsh{} 目标值通常不需要缩放}
\PY{c+c1}{\PYZsh{} 同比缩放所有属性,2种方法 最小\PYZhy{}最大缩放和标准化}
\PY{c+c1}{\PYZsh{} 最小\PYZhy{}最大缩放,又叫归一化,将值重新缩放到0到1之间,将值减去最小值并除以最大值和最小值差,如果你不希望是0到1,可以调整超参数feature\PYZus{}range}
\PY{c+c1}{\PYZsh{} 标准化 减去平均值 ,所以标准化均值总是0,然后除以方差,结果的分布具备单位方差,不同于归一化,标准化不将值绑定到特定范围,受异常值影响小}
\PY{c+c1}{\PYZsh{} 缩放器仅用来拟合训练集,不是完成的数据集}
\end{Verbatim}
\end{tcolorbox}

    \hypertarget{ux8f6cux6362ux6d41ux6c34ux7ebf}{%
\subsection{转换流水线}\label{ux8f6cux6362ux6d41ux6c34ux7ebf}}

    \begin{tcolorbox}[breakable, size=fbox, boxrule=1pt, pad at break*=1mm,colback=cellbackground, colframe=cellborder]
\prompt{In}{incolor}{73}{\boxspacing}
\begin{Verbatim}[commandchars=\\\{\}]
\PY{c+c1}{\PYZsh{} 许多数据转换的步骤需要以正确的顺序来执行, PipeLine来支持这样的转换}
\PY{c+c1}{\PYZsh{} pipline构造函数会通过一系列名称/估算器的配对来定义步骤的序列,必须是转换器,必须有fit\PYZus{}fransform()方法}
\PY{c+c1}{\PYZsh{} 调用流水线的fit方法时,会在所有转换器上按照顺序依次调用fit\PYZus{}transform(),将一个调用的输出作为参数传递给下一个调用方法,直到传递到最终}
\PY{c+c1}{\PYZsh{} 估算器,只会调用fit方法}
\PY{k+kn}{from} \PY{n+nn}{sklearn}\PY{n+nn}{.}\PY{n+nn}{pipeline} \PY{k+kn}{import} \PY{n}{Pipeline}
\PY{k+kn}{from} \PY{n+nn}{sklearn}\PY{n+nn}{.}\PY{n+nn}{preprocessing} \PY{k+kn}{import} \PY{n}{StandardScaler}

\PY{n}{num\PYZus{}pipeline} \PY{o}{=} \PY{n}{Pipeline}\PY{p}{(}\PY{p}{[}
        \PY{p}{(}\PY{l+s+s1}{\PYZsq{}}\PY{l+s+s1}{imputer}\PY{l+s+s1}{\PYZsq{}}\PY{p}{,} \PY{n}{SimpleImputer}\PY{p}{(}\PY{n}{strategy}\PY{o}{=}\PY{l+s+s2}{\PYZdq{}}\PY{l+s+s2}{median}\PY{l+s+s2}{\PYZdq{}}\PY{p}{)}\PY{p}{)}\PY{p}{,}
        \PY{p}{(}\PY{l+s+s1}{\PYZsq{}}\PY{l+s+s1}{attribs\PYZus{}adder}\PY{l+s+s1}{\PYZsq{}}\PY{p}{,} \PY{n}{CombinedAttributesAdder}\PY{p}{(}\PY{p}{)}\PY{p}{)}\PY{p}{,}
        \PY{p}{(}\PY{l+s+s1}{\PYZsq{}}\PY{l+s+s1}{std\PYZus{}scaler}\PY{l+s+s1}{\PYZsq{}}\PY{p}{,} \PY{n}{StandardScaler}\PY{p}{(}\PY{p}{)}\PY{p}{)}
    \PY{p}{]}\PY{p}{)}

\PY{n}{housing\PYZus{}num\PYZus{}tr} \PY{o}{=} \PY{n}{num\PYZus{}pipeline}\PY{o}{.}\PY{n}{fit\PYZus{}transform}\PY{p}{(}\PY{n}{housing\PYZus{}num}\PY{p}{)}
\end{Verbatim}
\end{tcolorbox}

    \begin{tcolorbox}[breakable, size=fbox, boxrule=1pt, pad at break*=1mm,colback=cellbackground, colframe=cellborder]
\prompt{In}{incolor}{74}{\boxspacing}
\begin{Verbatim}[commandchars=\\\{\}]
\PY{n}{housing\PYZus{}num}\PY{o}{.}\PY{n}{head}\PY{p}{(}\PY{p}{)}
\end{Verbatim}
\end{tcolorbox}

            \begin{tcolorbox}[breakable, size=fbox, boxrule=.5pt, pad at break*=1mm, opacityfill=0]
\prompt{Out}{outcolor}{74}{\boxspacing}
\begin{Verbatim}[commandchars=\\\{\}]
       longitude  latitude  housing\_median\_age  total\_rooms  total\_bedrooms  \textbackslash{}
5288     -118.47     34.05                27.0       4401.0          1033.0
12865    -121.34     38.69                17.0       1968.0           364.0
9174     -118.52     34.39                21.0       5477.0          1275.0
17247    -119.70     34.43                52.0       1364.0           460.0
14138    -117.07     32.74                37.0       1042.0           205.0

       population  households  median\_income income\_cat
5288       1725.0       962.0         4.1750          3
12865       996.0       331.0         3.7031          3
9174       3384.0      1222.0         3.6625          3
17247       804.0       400.0         2.3750          2
14138       589.0       208.0         2.6629          2
\end{Verbatim}
\end{tcolorbox}
        
    \begin{tcolorbox}[breakable, size=fbox, boxrule=1pt, pad at break*=1mm,colback=cellbackground, colframe=cellborder]
\prompt{In}{incolor}{75}{\boxspacing}
\begin{Verbatim}[commandchars=\\\{\}]
\PY{c+c1}{\PYZsh{} dataFrame \PYZhy{}\PYZgt{} series \PYZhy{}\PYZgt{} ndarray}
\PY{k}{class} \PY{n+nc}{DataFrameSelector}\PY{p}{(}\PY{n}{BaseEstimator}\PY{p}{,} \PY{n}{TransformerMixin}\PY{p}{)}\PY{p}{:}
    \PY{k}{def} \PY{n+nf+fm}{\PYZus{}\PYZus{}init\PYZus{}\PYZus{}}\PY{p}{(}\PY{n+nb+bp}{self}\PY{p}{,} \PY{n}{attribute\PYZus{}names}\PY{p}{)}\PY{p}{:}
        \PY{n+nb+bp}{self}\PY{o}{.}\PY{n}{attribute\PYZus{}names} \PY{o}{=} \PY{n}{attribute\PYZus{}names}
    \PY{k}{def} \PY{n+nf}{fit}\PY{p}{(}\PY{n+nb+bp}{self}\PY{p}{,} \PY{n}{X}\PY{p}{,} \PY{n}{y}\PY{o}{=}\PY{k+kc}{None}\PY{p}{)}\PY{p}{:}
        \PY{k}{return} \PY{n+nb+bp}{self}
    \PY{k}{def} \PY{n+nf}{transform}\PY{p}{(}\PY{n+nb+bp}{self}\PY{p}{,} \PY{n}{X}\PY{p}{)}\PY{p}{:}
        \PY{k}{return} \PY{n}{X}\PY{p}{[}\PY{n+nb+bp}{self}\PY{o}{.}\PY{n}{attribute\PYZus{}names}\PY{p}{]}\PY{o}{.}\PY{n}{values} \PY{c+c1}{\PYZsh{} 取DataFrame的数组}
    
\PY{n}{num\PYZus{}attribs} \PY{o}{=}\PY{n+nb}{list}\PY{p}{(}\PY{n}{housing\PYZus{}num}\PY{p}{)} 

\PY{n}{num\PYZus{}pipeline} \PY{o}{=} \PY{n}{Pipeline}\PY{p}{(}\PY{p}{[}
        \PY{c+c1}{\PYZsh{} dataFrame \PYZhy{}\PYZgt{} series \PYZhy{}\PYZgt{} ndarray}
        \PY{p}{(}\PY{l+s+s1}{\PYZsq{}}\PY{l+s+s1}{selector}\PY{l+s+s1}{\PYZsq{}}\PY{p}{,} \PY{n}{DataFrameSelector}\PY{p}{(}\PY{n}{num\PYZus{}attribs}\PY{p}{)}\PY{p}{)}\PY{p}{,} 
        \PY{c+c1}{\PYZsh{} imputer使用中位数补全缺失值}
        \PY{p}{(}\PY{l+s+s1}{\PYZsq{}}\PY{l+s+s1}{imputer}\PY{l+s+s1}{\PYZsq{}}\PY{p}{,} \PY{n}{SimpleImputer}\PY{p}{(}\PY{n}{strategy}\PY{o}{=}\PY{l+s+s2}{\PYZdq{}}\PY{l+s+s2}{median}\PY{l+s+s2}{\PYZdq{}}\PY{p}{)}\PY{p}{)}\PY{p}{,}
        \PY{c+c1}{\PYZsh{} 合成两个新的属性}
        \PY{p}{(}\PY{l+s+s1}{\PYZsq{}}\PY{l+s+s1}{attribs\PYZus{}adder}\PY{l+s+s1}{\PYZsq{}}\PY{p}{,} \PY{n}{CombinedAttributesAdder}\PY{p}{(}\PY{p}{)}\PY{p}{)}\PY{p}{,}
        \PY{c+c1}{\PYZsh{} 标准化缩放}
        \PY{p}{(}\PY{l+s+s1}{\PYZsq{}}\PY{l+s+s1}{std\PYZus{}scaler}\PY{l+s+s1}{\PYZsq{}}\PY{p}{,} \PY{n}{StandardScaler}\PY{p}{(}\PY{p}{)}\PY{p}{)}\PY{p}{,}
    \PY{p}{]}\PY{p}{)}

\PY{k+kn}{from} \PY{n+nn}{sklearn}\PY{n+nn}{.}\PY{n+nn}{base} \PY{k+kn}{import} \PY{n}{TransformerMixin} 
\PY{k}{class} \PY{n+nc}{MyLabelBinarizer}\PY{p}{(}\PY{n}{TransformerMixin}\PY{p}{)}\PY{p}{:}
    \PY{k}{def} \PY{n+nf+fm}{\PYZus{}\PYZus{}init\PYZus{}\PYZus{}}\PY{p}{(}\PY{n+nb+bp}{self}\PY{p}{,} \PY{o}{*}\PY{n}{args}\PY{p}{,} \PY{o}{*}\PY{o}{*}\PY{n}{kwargs}\PY{p}{)}\PY{p}{:}
        \PY{n+nb+bp}{self}\PY{o}{.}\PY{n}{encoder} \PY{o}{=} \PY{n}{LabelBinarizer}\PY{p}{(}\PY{o}{*}\PY{n}{args}\PY{p}{,} \PY{o}{*}\PY{o}{*}\PY{n}{kwargs}\PY{p}{)}
    \PY{k}{def} \PY{n+nf}{fit}\PY{p}{(}\PY{n+nb+bp}{self}\PY{p}{,} \PY{n}{x}\PY{p}{,} \PY{n}{y}\PY{o}{=}\PY{l+m+mi}{0}\PY{p}{)}\PY{p}{:}
        \PY{n+nb+bp}{self}\PY{o}{.}\PY{n}{encoder}\PY{o}{.}\PY{n}{fit}\PY{p}{(}\PY{n}{x}\PY{p}{)}
        \PY{k}{return} \PY{n+nb+bp}{self}
    \PY{k}{def} \PY{n+nf}{transform}\PY{p}{(}\PY{n+nb+bp}{self}\PY{p}{,} \PY{n}{x}\PY{p}{,} \PY{n}{y}\PY{o}{=}\PY{l+m+mi}{0}\PY{p}{)}\PY{p}{:}
        \PY{k}{return} \PY{n+nb+bp}{self}\PY{o}{.}\PY{n}{encoder}\PY{o}{.}\PY{n}{transform}\PY{p}{(}\PY{n}{x}\PY{p}{)}
    
\PY{n}{cat\PYZus{}attribs} \PY{o}{=} \PY{p}{[}\PY{l+s+s1}{\PYZsq{}}\PY{l+s+s1}{ocean\PYZus{}proximity}\PY{l+s+s1}{\PYZsq{}}\PY{p}{]} \PY{c+c1}{\PYZsh{} 文本列的名字}
\PY{k+kn}{from} \PY{n+nn}{sklearn}\PY{n+nn}{.}\PY{n+nn}{preprocessing} \PY{k+kn}{import} \PY{n}{LabelBinarizer}

\PY{n}{cat\PYZus{}pipeline} \PY{o}{=} \PY{n}{Pipeline}\PY{p}{(}\PY{p}{[}
        \PY{c+c1}{\PYZsh{} dataFrame \PYZhy{}\PYZgt{} series \PYZhy{}\PYZgt{} ndarray}
        \PY{p}{(}\PY{l+s+s1}{\PYZsq{}}\PY{l+s+s1}{selector}\PY{l+s+s1}{\PYZsq{}}\PY{p}{,} \PY{n}{DataFrameSelector}\PY{p}{(}\PY{n}{cat\PYZus{}attribs}\PY{p}{)}\PY{p}{)}\PY{p}{,}               
        \PY{c+c1}{\PYZsh{} LabelBinarizer一次完成两个转换 文本\PYZhy{}\PYZhy{}整数类型\PYZhy{}独热类型}
        \PY{p}{(}\PY{l+s+s1}{\PYZsq{}}\PY{l+s+s1}{LabelBinarizer}\PY{l+s+s1}{\PYZsq{}}\PY{p}{,} \PY{n}{MyLabelBinarizer}\PY{p}{(}\PY{p}{)}\PY{p}{)}\PY{p}{,}
    \PY{p}{]}\PY{p}{)}

\PY{c+c1}{\PYZsh{} housing.head()}

\PY{k+kn}{from} \PY{n+nn}{sklearn}\PY{n+nn}{.}\PY{n+nn}{pipeline} \PY{k+kn}{import} \PY{n}{FeatureUnion}

\PY{n}{full\PYZus{}pipeline} \PY{o}{=} \PY{n}{FeatureUnion}\PY{p}{(}\PY{n}{transformer\PYZus{}list}\PY{o}{=}\PY{p}{[}
        \PY{p}{(}\PY{l+s+s2}{\PYZdq{}}\PY{l+s+s2}{num\PYZus{}pipline}\PY{l+s+s2}{\PYZdq{}}\PY{p}{,} \PY{n}{num\PYZus{}pipeline}\PY{p}{,}\PY{p}{)}\PY{p}{,}
        \PY{p}{(}\PY{l+s+s1}{\PYZsq{}}\PY{l+s+s1}{cat\PYZus{}pipline}\PY{l+s+s1}{\PYZsq{}}\PY{p}{,} \PY{n}{cat\PYZus{}pipeline}\PY{p}{)}\PY{p}{,}
    \PY{p}{]}\PY{p}{)}

\PY{n}{housing\PYZus{}prepared} \PY{o}{=} \PY{n}{full\PYZus{}pipeline}\PY{o}{.}\PY{n}{fit\PYZus{}transform}\PY{p}{(}\PY{n}{housing}\PY{p}{)}

\PY{n}{housing\PYZus{}prepared} \PY{o}{=} \PY{n}{pd}\PY{o}{.}\PY{n}{DataFrame}\PY{p}{(}\PY{n}{housing\PYZus{}prepared}\PY{p}{)}
\PY{n}{housing\PYZus{}prepared}\PY{o}{.}\PY{n}{head}\PY{p}{(}\PY{p}{)}
\end{Verbatim}
\end{tcolorbox}

            \begin{tcolorbox}[breakable, size=fbox, boxrule=.5pt, pad at break*=1mm, opacityfill=0]
\prompt{Out}{outcolor}{75}{\boxspacing}
\begin{Verbatim}[commandchars=\\\{\}]
          0         1         2         3         4         5         6  \textbackslash{}
0  0.548975 -0.740274 -0.131568  0.807803  1.176384  0.260890  1.196810
1 -0.882976  1.431585 -0.926752 -0.307191 -0.411955 -0.376057 -0.439373
2  0.524028 -0.581129 -0.608678  1.300912  1.750940  1.710403  1.870989
3 -0.064718 -0.562406  1.856394 -0.583992 -0.184032 -0.543812 -0.260456
4  1.247487 -1.353449  0.663617 -0.731558 -0.789453 -0.731664 -0.758311

          7         8         9        10        11   12   13   14   15   16
0  0.157434 -0.006202 -0.336208 -0.116551  0.331966  1.0  0.0  0.0  0.0  0.0
1 -0.089132 -0.006202  0.203144 -0.005195 -0.456304  0.0  1.0  0.0  0.0  0.0
2 -0.110346 -0.006202 -0.372741 -0.027159  0.301428  1.0  0.0  0.0  0.0  0.0
3 -0.783062 -0.954456 -0.794534 -0.096691  1.956094  1.0  0.0  0.0  0.0  0.0
4 -0.632635 -0.954456 -0.165141 -0.021436 -0.269729  0.0  0.0  0.0  0.0  1.0
\end{Verbatim}
\end{tcolorbox}
        
    \hypertarget{ux9009ux62e9ux548cux8badux7ec3ux6a21ux578b}{%
\subsection{选择和训练模型}\label{ux9009ux62e9ux548cux8badux7ec3ux6a21ux578b}}

    \begin{tcolorbox}[breakable, size=fbox, boxrule=1pt, pad at break*=1mm,colback=cellbackground, colframe=cellborder]
\prompt{In}{incolor}{76}{\boxspacing}
\begin{Verbatim}[commandchars=\\\{\}]
\PY{n}{housing\PYZus{}labels}
\end{Verbatim}
\end{tcolorbox}

            \begin{tcolorbox}[breakable, size=fbox, boxrule=.5pt, pad at break*=1mm, opacityfill=0]
\prompt{Out}{outcolor}{76}{\boxspacing}
\begin{Verbatim}[commandchars=\\\{\}]
5288     500001.0
12865    114300.0
9174     228100.0
17247    293800.0
14138    116900.0
           {\ldots}
641      264700.0
9958     304800.0
5196      88700.0
245      126800.0
18453    217100.0
Name: median\_house\_value, Length: 16512, dtype: float64
\end{Verbatim}
\end{tcolorbox}
        
    \hypertarget{ux7ebfux6027ux56deux5f52}{%
\subsection{线性回归}\label{ux7ebfux6027ux56deux5f52}}

    \hypertarget{ux521bux5efaux4e00ux4e2aux7ebfux6027ux56deux5f52ux9884ux6d4bux5bf9ux8c61-ux5e76ux4f7fux7528fitux81eaux9002ux5e94}{%
\paragraph{1. 创建一个线性回归预测对象
并使用fit自适应}\label{ux521bux5efaux4e00ux4e2aux7ebfux6027ux56deux5f52ux9884ux6d4bux5bf9ux8c61-ux5e76ux4f7fux7528fitux81eaux9002ux5e94}}

    \begin{tcolorbox}[breakable, size=fbox, boxrule=1pt, pad at break*=1mm,colback=cellbackground, colframe=cellborder]
\prompt{In}{incolor}{77}{\boxspacing}
\begin{Verbatim}[commandchars=\\\{\}]
\PY{c+c1}{\PYZsh{} 训练模型和评估训练集}
\PY{k+kn}{from} \PY{n+nn}{sklearn}\PY{n+nn}{.}\PY{n+nn}{linear\PYZus{}model} \PY{k+kn}{import} \PY{n}{LinearRegression}
\PY{n}{lin\PYZus{}reg} \PY{o}{=} \PY{n}{LinearRegression}\PY{p}{(}\PY{p}{)} \PY{c+c1}{\PYZsh{} 估算器}
\PY{n}{lin\PYZus{}reg}\PY{o}{.}\PY{n}{fit}\PY{p}{(}\PY{n}{housing\PYZus{}prepared}\PY{p}{,} \PY{n}{housing\PYZus{}labels}\PY{p}{)} \PY{c+c1}{\PYZsh{} 传入数据和数据列}
\end{Verbatim}
\end{tcolorbox}

            \begin{tcolorbox}[breakable, size=fbox, boxrule=.5pt, pad at break*=1mm, opacityfill=0]
\prompt{Out}{outcolor}{77}{\boxspacing}
\begin{Verbatim}[commandchars=\\\{\}]
LinearRegression(copy\_X=True, fit\_intercept=True, n\_jobs=None, normalize=False)
\end{Verbatim}
\end{tcolorbox}
        
    \hypertarget{ux4f7fux7528ux5c11ux91cfux6570ux636eux8fdbux884cux9884ux6d4b}{%
\paragraph{2.
使用少量数据进行预测}\label{ux4f7fux7528ux5c11ux91cfux6570ux636eux8fdbux884cux9884ux6d4b}}

    \begin{tcolorbox}[breakable, size=fbox, boxrule=1pt, pad at break*=1mm,colback=cellbackground, colframe=cellborder]
\prompt{In}{incolor}{78}{\boxspacing}
\begin{Verbatim}[commandchars=\\\{\}]
\PY{c+c1}{\PYZsh{} 预测数据}
\PY{n}{some\PYZus{}data} \PY{o}{=} \PY{n}{housing}\PY{o}{.}\PY{n}{iloc}\PY{p}{[}\PY{p}{:}\PY{l+m+mi}{5}\PY{p}{]}
\PY{n}{some\PYZus{}labels} \PY{o}{=} \PY{n}{housing\PYZus{}labels}\PY{o}{.}\PY{n}{iloc}\PY{p}{[}\PY{p}{:}\PY{l+m+mi}{5}\PY{p}{]}

\PY{n}{some\PYZus{}data\PYZus{}prepared} \PY{o}{=} \PY{n}{full\PYZus{}pipeline}\PY{o}{.}\PY{n}{transform}\PY{p}{(}\PY{n}{some\PYZus{}data}\PY{p}{)}

\PY{n+nb}{print}\PY{p}{(}\PY{l+s+s2}{\PYZdq{}}\PY{l+s+s2}{Predictions:}\PY{l+s+s2}{\PYZdq{}}\PY{p}{,} \PY{n}{lin\PYZus{}reg}\PY{o}{.}\PY{n}{predict}\PY{p}{(}\PY{n}{some\PYZus{}data\PYZus{}prepared}\PY{p}{)}\PY{p}{)}
\end{Verbatim}
\end{tcolorbox}

    \begin{Verbatim}[commandchars=\\\{\}]
Predictions: [286799.195644   122918.09851162 222641.70343038 246153.22245923
 174899.32872953]
    \end{Verbatim}

    \hypertarget{ux5747ux65b9ux8befux5dee}{%
\subsubsection{均方误差}\label{ux5747ux65b9ux8befux5dee}}

    \begin{tcolorbox}[breakable, size=fbox, boxrule=1pt, pad at break*=1mm,colback=cellbackground, colframe=cellborder]
\prompt{In}{incolor}{79}{\boxspacing}
\begin{Verbatim}[commandchars=\\\{\}]
\PY{c+c1}{\PYZsh{} 均方误差(MSE, mean squared error)}
\PY{k+kn}{from} \PY{n+nn}{sklearn}\PY{n+nn}{.}\PY{n+nn}{metrics} \PY{k+kn}{import} \PY{n}{mean\PYZus{}squared\PYZus{}error}

\PY{c+c1}{\PYZsh{} 预测数据}
\PY{n}{housing\PYZus{}predictions} \PY{o}{=} \PY{n}{lin\PYZus{}reg}\PY{o}{.}\PY{n}{predict}\PY{p}{(}\PY{n}{housing\PYZus{}prepared}\PY{p}{)}
\PY{c+c1}{\PYZsh{} 均方误差(MSE, mean squared error)}
\PY{n}{lin\PYZus{}mse} \PY{o}{=} \PY{n}{mean\PYZus{}squared\PYZus{}error}\PY{p}{(}\PY{n}{housing\PYZus{}labels}\PY{p}{,} \PY{n}{housing\PYZus{}predictions}\PY{p}{)}
\PY{n}{lin\PYZus{}mse}
\PY{n}{lin\PYZus{}rmse} \PY{o}{=} \PY{n}{np}\PY{o}{.}\PY{n}{sqrt}\PY{p}{(}\PY{n}{lin\PYZus{}mse}\PY{p}{)}
\PY{n}{lin\PYZus{}rmse}

\PY{c+c1}{\PYZsh{} 大多数地区的房屋中位数 在120000到265000美元之间,预测误差高达 68628,这是一个典型的模型对训练数据拟合不足的案例,}
\PY{c+c1}{\PYZsh{} 原因可能是特征无法提供足够的信息来做出更好的预测,或者模型本身不够强大,}
\PY{c+c1}{\PYZsh{} 1. 选择强大的模型,2 为算法提供更好的特征,3.减少对模型的限制等方法,}

\PY{c+c1}{\PYZsh{} 决策树可以找到复杂的非线性关系}
\end{Verbatim}
\end{tcolorbox}

            \begin{tcolorbox}[breakable, size=fbox, boxrule=.5pt, pad at break*=1mm, opacityfill=0]
\prompt{Out}{outcolor}{79}{\boxspacing}
\begin{Verbatim}[commandchars=\\\{\}]
67839.68172049934
\end{Verbatim}
\end{tcolorbox}
        
    \hypertarget{ux51b3ux7b56ux6811}{%
\subsection{决策树}\label{ux51b3ux7b56ux6811}}

    \hypertarget{ux521bux5efaux51b3ux7b56ux6811ux5bf9ux8c61-ux5e76ux4f7fux7528fitux81eaux9002ux5e94}{%
\paragraph{1. 创建决策树对象
并使用fit自适应}\label{ux521bux5efaux51b3ux7b56ux6811ux5bf9ux8c61-ux5e76ux4f7fux7528fitux81eaux9002ux5e94}}

    \begin{tcolorbox}[breakable, size=fbox, boxrule=1pt, pad at break*=1mm,colback=cellbackground, colframe=cellborder]
\prompt{In}{incolor}{80}{\boxspacing}
\begin{Verbatim}[commandchars=\\\{\}]
\PY{k+kn}{from} \PY{n+nn}{sklearn}\PY{n+nn}{.}\PY{n+nn}{tree} \PY{k+kn}{import} \PY{n}{DecisionTreeRegressor}
\PY{c+c1}{\PYZsh{} 创建一个决策树对象}
\PY{n}{tree\PYZus{}reg} \PY{o}{=} \PY{n}{DecisionTreeRegressor}\PY{p}{(}\PY{n}{random\PYZus{}state}\PY{o}{=}\PY{l+m+mi}{42}\PY{p}{)}
\PY{c+c1}{\PYZsh{} 取数据和数据列 fit自适应}
\PY{n}{tree\PYZus{}reg}\PY{o}{.}\PY{n}{fit}\PY{p}{(}\PY{n}{housing\PYZus{}prepared}\PY{p}{,} \PY{n}{housing\PYZus{}labels}\PY{p}{)}
\end{Verbatim}
\end{tcolorbox}

            \begin{tcolorbox}[breakable, size=fbox, boxrule=.5pt, pad at break*=1mm, opacityfill=0]
\prompt{Out}{outcolor}{80}{\boxspacing}
\begin{Verbatim}[commandchars=\\\{\}]
DecisionTreeRegressor(ccp\_alpha=0.0, criterion='mse', max\_depth=None,
                      max\_features=None, max\_leaf\_nodes=None,
                      min\_impurity\_decrease=0.0, min\_impurity\_split=None,
                      min\_samples\_leaf=1, min\_samples\_split=2,
                      min\_weight\_fraction\_leaf=0.0, presort='deprecated',
                      random\_state=42, splitter='best')
\end{Verbatim}
\end{tcolorbox}
        
    \hypertarget{ux9884ux6d4bux6570ux636eux5e76ux9a8cux8bc1ux9884ux6d4bux7684ux5747ux65b9ux8befux5dee}{%
\paragraph{2.
预测数据,并验证预测的均方误差}\label{ux9884ux6d4bux6570ux636eux5e76ux9a8cux8bc1ux9884ux6d4bux7684ux5747ux65b9ux8befux5dee}}

    \begin{tcolorbox}[breakable, size=fbox, boxrule=1pt, pad at break*=1mm,colback=cellbackground, colframe=cellborder]
\prompt{In}{incolor}{81}{\boxspacing}
\begin{Verbatim}[commandchars=\\\{\}]
\PY{c+c1}{\PYZsh{} 预测数据}
\PY{n}{housing\PYZus{}predictions} \PY{o}{=} \PY{n}{tree\PYZus{}reg}\PY{o}{.}\PY{n}{predict}\PY{p}{(}\PY{n}{housing\PYZus{}prepared}\PY{p}{)}
\PY{c+c1}{\PYZsh{} 均方误差(MSE, mean squared error)}
\PY{n}{tree\PYZus{}mse} \PY{o}{=} \PY{n}{mean\PYZus{}squared\PYZus{}error}\PY{p}{(}\PY{n}{housing\PYZus{}labels}\PY{p}{,} \PY{n}{housing\PYZus{}predictions}\PY{p}{)}
\PY{n}{tree\PYZus{}rmse} \PY{o}{=} \PY{n}{np}\PY{o}{.}\PY{n}{sqrt}\PY{p}{(}\PY{n}{tree\PYZus{}mse}\PY{p}{)}
\PY{n}{tree\PYZus{}rmse}
\end{Verbatim}
\end{tcolorbox}

            \begin{tcolorbox}[breakable, size=fbox, boxrule=.5pt, pad at break*=1mm, opacityfill=0]
\prompt{Out}{outcolor}{81}{\boxspacing}
\begin{Verbatim}[commandchars=\\\{\}]
0.0
\end{Verbatim}
\end{tcolorbox}
        
    \begin{tcolorbox}[breakable, size=fbox, boxrule=1pt, pad at break*=1mm,colback=cellbackground, colframe=cellborder]
\prompt{In}{incolor}{82}{\boxspacing}
\begin{Verbatim}[commandchars=\\\{\}]
 \PY{c+c1}{\PYZsh{} 完美,也可能是这个模型对数据严重过度拟合了,如何确认?轻易不要启动测试集,拿训练集中的一部分用于训练,另一部分用于模型的验证}
\end{Verbatim}
\end{tcolorbox}

    \hypertarget{ux4ea4ux53c9ux9a8cux8bc1ux66f4ux597dux7684ux8fdbux884cux8bc4ux4f30ux9884ux6d4b}{%
\subsection{交叉验证------更好的进行评估预测}\label{ux4ea4ux53c9ux9a8cux8bc1ux66f4ux597dux7684ux8fdbux884cux8bc4ux4f30ux9884ux6d4b}}

    \hypertarget{ux51b3ux7b56ux6811ux7684ux4ea4ux53c9ux9a8cux8bc1}{%
\subsubsection{决策树的交叉验证}\label{ux51b3ux7b56ux6811ux7684ux4ea4ux53c9ux9a8cux8bc1}}

    \begin{tcolorbox}[breakable, size=fbox, boxrule=1pt, pad at break*=1mm,colback=cellbackground, colframe=cellborder]
\prompt{In}{incolor}{83}{\boxspacing}
\begin{Verbatim}[commandchars=\\\{\}]
\PY{c+c1}{\PYZsh{} 使用train\PYZus{}test\PYZus{}split函数将训练集分为较小的训练集和验证集,然后根据这些较小的训练集来训练模型,并对其进行评估}
\PY{c+c1}{\PYZsh{} sklearn的交叉验证,将训练集随机分割成10个不同的子集,每个子集称为一个折叠,对模型进行10次训练和评估,每次挑选1个折叠进行评估,另外9个进行训练}
\PY{k+kn}{from} \PY{n+nn}{sklearn}\PY{n+nn}{.}\PY{n+nn}{model\PYZus{}selection} \PY{k+kn}{import} \PY{n}{cross\PYZus{}val\PYZus{}score}
\PY{c+c1}{\PYZsh{} neg\PYZus{}mean\PYZus{}squared\PYZus{}error‘ 也就是 均方差回归损失 该统计参数是预测数据和原始数据对应点误差的平方和的均值}

\PY{c+c1}{\PYZsh{} 决策树交叉验证}
\PY{c+c1}{\PYZsh{} tree\PYZus{}reg决策树模型对象,测试集和测试集标签,scoring评估算法\PYZhy{}neg\PYZus{}mean\PYZus{}squared\PYZus{}error均方差评估算法,cv交叉验证次数}
\PY{n}{scores} \PY{o}{=} \PY{n}{cross\PYZus{}val\PYZus{}score}\PY{p}{(}\PY{n}{tree\PYZus{}reg}\PY{p}{,} \PY{n}{housing\PYZus{}prepared}\PY{p}{,} \PY{n}{housing\PYZus{}labels}\PY{p}{,}
                         \PY{n}{scoring}\PY{o}{=}\PY{l+s+s2}{\PYZdq{}}\PY{l+s+s2}{neg\PYZus{}mean\PYZus{}squared\PYZus{}error}\PY{l+s+s2}{\PYZdq{}}\PY{p}{,} \PY{n}{cv}\PY{o}{=}\PY{l+m+mi}{10}\PY{p}{)}
\PY{n}{tree\PYZus{}rmse\PYZus{}scores} \PY{o}{=} \PY{n}{np}\PY{o}{.}\PY{n}{sqrt}\PY{p}{(}\PY{o}{\PYZhy{}}\PY{n}{scores}\PY{p}{)} \PY{c+c1}{\PYZsh{} 平方根得到的是负值,需要取反}
\PY{n}{tree\PYZus{}rmse\PYZus{}scores}
\end{Verbatim}
\end{tcolorbox}

            \begin{tcolorbox}[breakable, size=fbox, boxrule=.5pt, pad at break*=1mm, opacityfill=0]
\prompt{Out}{outcolor}{83}{\boxspacing}
\begin{Verbatim}[commandchars=\\\{\}]
array([71627.92164456, 73493.35335647, 68087.25081095, 66063.46397032,
       73560.70411398, 70528.06580062, 65964.58500517, 69211.60778723,
       66606.62921511, 72269.3260234 ])
\end{Verbatim}
\end{tcolorbox}
        
    \hypertarget{display_scores-ux8f93ux51faux9a8cux8bc1ux503cux5e73ux5747ux9a8cux8bc1ux503cux9a8cux8bc1ux503cux7684ux6ce2ux52a8}{%
\subsubsection{display\_scores
输出验证值、平均验证值、验证值的波动}\label{display_scores-ux8f93ux51faux9a8cux8bc1ux503cux5e73ux5747ux9a8cux8bc1ux503cux9a8cux8bc1ux503cux7684ux6ce2ux52a8}}

    \begin{tcolorbox}[breakable, size=fbox, boxrule=1pt, pad at break*=1mm,colback=cellbackground, colframe=cellborder]
\prompt{In}{incolor}{84}{\boxspacing}
\begin{Verbatim}[commandchars=\\\{\}]
\PY{c+c1}{\PYZsh{} 输出验证}
\PY{k}{def} \PY{n+nf}{display\PYZus{}scores}\PY{p}{(}\PY{n}{scores}\PY{p}{)}\PY{p}{:}
    \PY{n+nb}{print}\PY{p}{(}\PY{l+s+s2}{\PYZdq{}}\PY{l+s+s2}{Scores:}\PY{l+s+s2}{\PYZdq{}}\PY{p}{,} \PY{n}{scores}\PY{p}{)}  \PY{c+c1}{\PYZsh{} 输出验证值}
    \PY{n+nb}{print}\PY{p}{(}\PY{l+s+s2}{\PYZdq{}}\PY{l+s+s2}{Mean:}\PY{l+s+s2}{\PYZdq{}}\PY{p}{,} \PY{n}{scores}\PY{o}{.}\PY{n}{mean}\PY{p}{(}\PY{p}{)}\PY{p}{)}  \PY{c+c1}{\PYZsh{} 输出平均验证值}
    \PY{n+nb}{print}\PY{p}{(}\PY{l+s+s2}{\PYZdq{}}\PY{l+s+s2}{Standard deviation:}\PY{l+s+s2}{\PYZdq{}}\PY{p}{,} \PY{n}{scores}\PY{o}{.}\PY{n}{std}\PY{p}{(}\PY{p}{)}\PY{p}{)}  \PY{c+c1}{\PYZsh{} 输出验证值的波动}

\PY{n}{display\PYZus{}scores}\PY{p}{(}\PY{n}{tree\PYZus{}rmse\PYZus{}scores}\PY{p}{)}
\end{Verbatim}
\end{tcolorbox}

    \begin{Verbatim}[commandchars=\\\{\}]
Scores: [71627.92164456 73493.35335647 68087.25081095 66063.46397032
 73560.70411398 70528.06580062 65964.58500517 69211.60778723
 66606.62921511 72269.3260234 ]
Mean: 69741.29077278195
Standard deviation: 2826.086992779198
    \end{Verbatim}

    \hypertarget{ux7ebfux6027ux56deux5f52ux7684ux4ea4ux53c9ux9a8cux8bc1}{%
\subsubsection{线性回归的交叉验证}\label{ux7ebfux6027ux56deux5f52ux7684ux4ea4ux53c9ux9a8cux8bc1}}

    \begin{tcolorbox}[breakable, size=fbox, boxrule=1pt, pad at break*=1mm,colback=cellbackground, colframe=cellborder]
\prompt{In}{incolor}{85}{\boxspacing}
\begin{Verbatim}[commandchars=\\\{\}]
\PY{c+c1}{\PYZsh{} 线性 交叉验证}
\PY{n}{lin\PYZus{}scores} \PY{o}{=} \PY{n}{cross\PYZus{}val\PYZus{}score}\PY{p}{(}\PY{n}{lin\PYZus{}reg}\PY{p}{,} \PY{n}{housing\PYZus{}prepared}\PY{p}{,} \PY{n}{housing\PYZus{}labels}\PY{p}{,}
                             \PY{n}{scoring}\PY{o}{=}\PY{l+s+s2}{\PYZdq{}}\PY{l+s+s2}{neg\PYZus{}mean\PYZus{}squared\PYZus{}error}\PY{l+s+s2}{\PYZdq{}}\PY{p}{,} \PY{n}{cv}\PY{o}{=}\PY{l+m+mi}{10}\PY{p}{)}
\PY{n}{lin\PYZus{}rmse\PYZus{}scores} \PY{o}{=} \PY{n}{np}\PY{o}{.}\PY{n}{sqrt}\PY{p}{(}\PY{o}{\PYZhy{}}\PY{n}{lin\PYZus{}scores}\PY{p}{)}
\PY{n}{display\PYZus{}scores}\PY{p}{(}\PY{n}{lin\PYZus{}rmse\PYZus{}scores}\PY{p}{)}
\end{Verbatim}
\end{tcolorbox}

    \begin{Verbatim}[commandchars=\\\{\}]
Scores: [66173.64137099 71052.24844005 67580.25757791 67279.07810014
 74543.15436276 69444.5339994  67780.89653745 65077.9510503
 64890.46533845 67490.64445917]
Mean: 68131.28712366273
Standard deviation: 2767.488539093263
    \end{Verbatim}

    \hypertarget{ux968fux673aux68eeux6797}{%
\subsection{随机森林}\label{ux968fux673aux68eeux6797}}

    \hypertarget{ux521bux5efaux968fux673aux68eeux6797ux5bf9ux8c61-ux5e76ux4f7fux7528fitux81eaux9002ux5e94}{%
\paragraph{1. 创建随机森林对象
并使用fit自适应}\label{ux521bux5efaux968fux673aux68eeux6797ux5bf9ux8c61-ux5e76ux4f7fux7528fitux81eaux9002ux5e94}}

    \begin{tcolorbox}[breakable, size=fbox, boxrule=1pt, pad at break*=1mm,colback=cellbackground, colframe=cellborder]
\prompt{In}{incolor}{86}{\boxspacing}
\begin{Verbatim}[commandchars=\\\{\}]
\PY{k+kn}{from} \PY{n+nn}{sklearn}\PY{n+nn}{.}\PY{n+nn}{ensemble} \PY{k+kn}{import} \PY{n}{RandomForestRegressor}

\PY{c+c1}{\PYZsh{} 创建一个随机森林对象}
\PY{n}{forest\PYZus{}reg} \PY{o}{=} \PY{n}{RandomForestRegressor}\PY{p}{(}\PY{n}{n\PYZus{}estimators}\PY{o}{=}\PY{l+m+mi}{10}\PY{p}{,} \PY{n}{random\PYZus{}state}\PY{o}{=}\PY{l+m+mi}{42}\PY{p}{)}
\PY{c+c1}{\PYZsh{} 取数据和数据列 fit自适应}
\PY{n}{forest\PYZus{}reg}\PY{o}{.}\PY{n}{fit}\PY{p}{(}\PY{n}{housing\PYZus{}prepared}\PY{p}{,} \PY{n}{housing\PYZus{}labels}\PY{p}{)}
\end{Verbatim}
\end{tcolorbox}

            \begin{tcolorbox}[breakable, size=fbox, boxrule=.5pt, pad at break*=1mm, opacityfill=0]
\prompt{Out}{outcolor}{86}{\boxspacing}
\begin{Verbatim}[commandchars=\\\{\}]
RandomForestRegressor(bootstrap=True, ccp\_alpha=0.0, criterion='mse',
                      max\_depth=None, max\_features='auto', max\_leaf\_nodes=None,
                      max\_samples=None, min\_impurity\_decrease=0.0,
                      min\_impurity\_split=None, min\_samples\_leaf=1,
                      min\_samples\_split=2, min\_weight\_fraction\_leaf=0.0,
                      n\_estimators=10, n\_jobs=None, oob\_score=False,
                      random\_state=42, verbose=0, warm\_start=False)
\end{Verbatim}
\end{tcolorbox}
        
    \hypertarget{ux9884ux6d4bux6570ux636eux5e76ux9a8cux8bc1ux9884ux6d4bux7684ux5747ux65b9ux8befux5dee}{%
\paragraph{2.
预测数据,并验证预测的均方误差}\label{ux9884ux6d4bux6570ux636eux5e76ux9a8cux8bc1ux9884ux6d4bux7684ux5747ux65b9ux8befux5dee}}

    \begin{tcolorbox}[breakable, size=fbox, boxrule=1pt, pad at break*=1mm,colback=cellbackground, colframe=cellborder]
\prompt{In}{incolor}{88}{\boxspacing}
\begin{Verbatim}[commandchars=\\\{\}]
\PY{c+c1}{\PYZsh{} 预测数据}
\PY{n}{housing\PYZus{}predictions} \PY{o}{=} \PY{n}{forest\PYZus{}reg}\PY{o}{.}\PY{n}{predict}\PY{p}{(}\PY{n}{housing\PYZus{}prepared}\PY{p}{)}
\PY{c+c1}{\PYZsh{} 均方误差(MSE, mean squared error)}
\PY{n}{forest\PYZus{}mse} \PY{o}{=} \PY{n}{mean\PYZus{}squared\PYZus{}error}\PY{p}{(}\PY{n}{housing\PYZus{}labels}\PY{p}{,} \PY{n}{housing\PYZus{}predictions}\PY{p}{)}
\PY{n}{forest\PYZus{}rmse} \PY{o}{=} \PY{n}{np}\PY{o}{.}\PY{n}{sqrt}\PY{p}{(}\PY{n}{forest\PYZus{}mse}\PY{p}{)}
\PY{n}{forest\PYZus{}rmse}
\end{Verbatim}
\end{tcolorbox}

            \begin{tcolorbox}[breakable, size=fbox, boxrule=.5pt, pad at break*=1mm, opacityfill=0]
\prompt{Out}{outcolor}{88}{\boxspacing}
\begin{Verbatim}[commandchars=\\\{\}]
22352.70889226364
\end{Verbatim}
\end{tcolorbox}
        
    \hypertarget{ux4f7fux7528ux4ea4ux53c9ux9a8cux8bc1ux6765ux8bc4ux4f30}{%
\paragraph{3.
使用交叉验证来评估}\label{ux4f7fux7528ux4ea4ux53c9ux9a8cux8bc1ux6765ux8bc4ux4f30}}

    \begin{tcolorbox}[breakable, size=fbox, boxrule=1pt, pad at break*=1mm,colback=cellbackground, colframe=cellborder]
\prompt{In}{incolor}{89}{\boxspacing}
\begin{Verbatim}[commandchars=\\\{\}]
\PY{c+c1}{\PYZsh{} 随机森林的交叉验证}
\PY{k+kn}{from} \PY{n+nn}{sklearn}\PY{n+nn}{.}\PY{n+nn}{model\PYZus{}selection} \PY{k+kn}{import} \PY{n}{cross\PYZus{}val\PYZus{}score}

\PY{n}{forest\PYZus{}scores} \PY{o}{=} \PY{n}{cross\PYZus{}val\PYZus{}score}\PY{p}{(}\PY{n}{forest\PYZus{}reg}\PY{p}{,} \PY{n}{housing\PYZus{}prepared}\PY{p}{,} \PY{n}{housing\PYZus{}labels}\PY{p}{,}
                                \PY{n}{scoring}\PY{o}{=}\PY{l+s+s2}{\PYZdq{}}\PY{l+s+s2}{neg\PYZus{}mean\PYZus{}squared\PYZus{}error}\PY{l+s+s2}{\PYZdq{}}\PY{p}{,} \PY{n}{cv}\PY{o}{=}\PY{l+m+mi}{10}\PY{p}{)}
\PY{n}{forest\PYZus{}rmse\PYZus{}scores} \PY{o}{=} \PY{n}{np}\PY{o}{.}\PY{n}{sqrt}\PY{p}{(}\PY{o}{\PYZhy{}}\PY{n}{forest\PYZus{}scores}\PY{p}{)}
\PY{n}{display\PYZus{}scores}\PY{p}{(}\PY{n}{forest\PYZus{}rmse\PYZus{}scores}\PY{p}{)}
\end{Verbatim}
\end{tcolorbox}

    \begin{Verbatim}[commandchars=\\\{\}]
Scores: [51987.84730523 54832.44874275 51080.86922731 50189.02875471
 53648.34965171 52843.14158323 51640.25146871 51889.54716536
 50299.55037764 52509.92755343]
Mean: 52092.09618300701
Standard deviation: 1370.5339431649338
    \end{Verbatim}

    \hypertarget{ux6a21ux578bux8c03ux53c2ux548cux7f51ux683cux641cux7d22}{%
\subsection{模型调参和网格搜索}\label{ux6a21ux578bux8c03ux53c2ux548cux7f51ux683cux641cux7d22}}

\begin{enumerate}
\def\labelenumi{\arabic{enumi}.}
\tightlist
\item
  手动调整超参数,找到很好的组合很困难
\item
  使用GridSearchCV替你进行搜索,告诉它,进行试验的超参数是什么,和需要尝试的值,它会使用交叉验证评估所有超参数的可能组合
\end{enumerate}

    \begin{tcolorbox}[breakable, size=fbox, boxrule=1pt, pad at break*=1mm,colback=cellbackground, colframe=cellborder]
\prompt{In}{incolor}{90}{\boxspacing}
\begin{Verbatim}[commandchars=\\\{\}]
\PY{k+kn}{from} \PY{n+nn}{sklearn}\PY{n+nn}{.}\PY{n+nn}{model\PYZus{}selection} \PY{k+kn}{import} \PY{n}{GridSearchCV}

\PY{n}{param\PYZus{}grid} \PY{o}{=} \PY{p}{[}
    \PY{p}{\PYZob{}}\PY{l+s+s1}{\PYZsq{}}\PY{l+s+s1}{n\PYZus{}estimators}\PY{l+s+s1}{\PYZsq{}}\PY{p}{:} \PY{p}{[}\PY{l+m+mi}{3}\PY{p}{,} \PY{l+m+mi}{10}\PY{p}{,} \PY{l+m+mi}{30}\PY{p}{]}\PY{p}{,} \PY{l+s+s1}{\PYZsq{}}\PY{l+s+s1}{max\PYZus{}features}\PY{l+s+s1}{\PYZsq{}}\PY{p}{:} \PY{p}{[}\PY{l+m+mi}{2}\PY{p}{,} \PY{l+m+mi}{4}\PY{p}{,} \PY{l+m+mi}{6}\PY{p}{,} \PY{l+m+mi}{8}\PY{p}{]}\PY{p}{\PYZcb{}}\PY{p}{,} \PY{c+c1}{\PYZsh{} 3 * 4 = 12 种组合}
    \PY{p}{\PYZob{}}\PY{l+s+s1}{\PYZsq{}}\PY{l+s+s1}{bootstrap}\PY{l+s+s1}{\PYZsq{}}\PY{p}{:} \PY{p}{[}\PY{k+kc}{False}\PY{p}{]}\PY{p}{,} \PY{l+s+s1}{\PYZsq{}}\PY{l+s+s1}{n\PYZus{}estimators}\PY{l+s+s1}{\PYZsq{}}\PY{p}{:} \PY{p}{[}\PY{l+m+mi}{3}\PY{p}{,} \PY{l+m+mi}{10}\PY{p}{]}\PY{p}{,} \PY{l+s+s1}{\PYZsq{}}\PY{l+s+s1}{max\PYZus{}features}\PY{l+s+s1}{\PYZsq{}}\PY{p}{:} \PY{p}{[}\PY{l+m+mi}{2}\PY{p}{,} \PY{l+m+mi}{3}\PY{p}{,} \PY{l+m+mi}{4}\PY{p}{]}\PY{p}{\PYZcb{}}\PY{p}{,} \PY{c+c1}{\PYZsh{} 2 * 3 = 6 种组合}
  \PY{p}{]}

\PY{n}{forest\PYZus{}reg} \PY{o}{=} \PY{n}{RandomForestRegressor}\PY{p}{(}\PY{n}{random\PYZus{}state}\PY{o}{=}\PY{l+m+mi}{42}\PY{p}{)}  \PY{c+c1}{\PYZsh{} 随机森林回归}
\PY{n}{grid\PYZus{}search} \PY{o}{=} \PY{n}{GridSearchCV}\PY{p}{(}\PY{n}{forest\PYZus{}reg}\PY{p}{,} \PY{n}{param\PYZus{}grid}\PY{p}{,} \PY{n}{cv}\PY{o}{=}\PY{l+m+mi}{5}\PY{p}{,}
                           \PY{n}{scoring}\PY{o}{=}\PY{l+s+s1}{\PYZsq{}}\PY{l+s+s1}{neg\PYZus{}mean\PYZus{}squared\PYZus{}error}\PY{l+s+s1}{\PYZsq{}}\PY{p}{,} \PY{n}{return\PYZus{}train\PYZus{}score}\PY{o}{=}\PY{k+kc}{True}\PY{p}{)}
\PY{n}{grid\PYZus{}search}\PY{o}{.}\PY{n}{fit}\PY{p}{(}\PY{n}{housing\PYZus{}prepared}\PY{p}{,} \PY{n}{housing\PYZus{}labels}\PY{p}{)}
\end{Verbatim}
\end{tcolorbox}

            \begin{tcolorbox}[breakable, size=fbox, boxrule=.5pt, pad at break*=1mm, opacityfill=0]
\prompt{Out}{outcolor}{90}{\boxspacing}
\begin{Verbatim}[commandchars=\\\{\}]
GridSearchCV(cv=5, error\_score=nan,
             estimator=RandomForestRegressor(bootstrap=True, ccp\_alpha=0.0,
                                             criterion='mse', max\_depth=None,
                                             max\_features='auto',
                                             max\_leaf\_nodes=None,
                                             max\_samples=None,
                                             min\_impurity\_decrease=0.0,
                                             min\_impurity\_split=None,
                                             min\_samples\_leaf=1,
                                             min\_samples\_split=2,
                                             min\_weight\_fraction\_leaf=0.0,
                                             n\_estimators=100, n\_jobs=None,
                                             oob\_score=False, random\_state=42,
                                             verbose=0, warm\_start=False),
             iid='deprecated', n\_jobs=None,
             param\_grid=[\{'max\_features': [2, 4, 6, 8],
                          'n\_estimators': [3, 10, 30]\},
                         \{'bootstrap': [False], 'max\_features': [2, 3, 4],
                          'n\_estimators': [3, 10]\}],
             pre\_dispatch='2*n\_jobs', refit=True, return\_train\_score=True,
             scoring='neg\_mean\_squared\_error', verbose=0)
\end{Verbatim}
\end{tcolorbox}
        
    \begin{tcolorbox}[breakable, size=fbox, boxrule=1pt, pad at break*=1mm,colback=cellbackground, colframe=cellborder]
\prompt{In}{incolor}{91}{\boxspacing}
\begin{Verbatim}[commandchars=\\\{\}]
\PY{c+c1}{\PYZsh{} 输出最好的参数}
\PY{n}{grid\PYZus{}search}\PY{o}{.}\PY{n}{best\PYZus{}params\PYZus{}}
\end{Verbatim}
\end{tcolorbox}

            \begin{tcolorbox}[breakable, size=fbox, boxrule=.5pt, pad at break*=1mm, opacityfill=0]
\prompt{Out}{outcolor}{91}{\boxspacing}
\begin{Verbatim}[commandchars=\\\{\}]
\{'max\_features': 6, 'n\_estimators': 30\}
\end{Verbatim}
\end{tcolorbox}
        
    \begin{tcolorbox}[breakable, size=fbox, boxrule=1pt, pad at break*=1mm,colback=cellbackground, colframe=cellborder]
\prompt{In}{incolor}{92}{\boxspacing}
\begin{Verbatim}[commandchars=\\\{\}]
\PY{c+c1}{\PYZsh{} 输出最好的估算器}
\PY{n}{grid\PYZus{}search}\PY{o}{.}\PY{n}{best\PYZus{}estimator\PYZus{}}
\end{Verbatim}
\end{tcolorbox}

            \begin{tcolorbox}[breakable, size=fbox, boxrule=.5pt, pad at break*=1mm, opacityfill=0]
\prompt{Out}{outcolor}{92}{\boxspacing}
\begin{Verbatim}[commandchars=\\\{\}]
RandomForestRegressor(bootstrap=True, ccp\_alpha=0.0, criterion='mse',
                      max\_depth=None, max\_features=6, max\_leaf\_nodes=None,
                      max\_samples=None, min\_impurity\_decrease=0.0,
                      min\_impurity\_split=None, min\_samples\_leaf=1,
                      min\_samples\_split=2, min\_weight\_fraction\_leaf=0.0,
                      n\_estimators=30, n\_jobs=None, oob\_score=False,
                      random\_state=42, verbose=0, warm\_start=False)
\end{Verbatim}
\end{tcolorbox}
        
    \begin{tcolorbox}[breakable, size=fbox, boxrule=1pt, pad at break*=1mm,colback=cellbackground, colframe=cellborder]
\prompt{In}{incolor}{93}{\boxspacing}
\begin{Verbatim}[commandchars=\\\{\}]
\PY{c+c1}{\PYZsh{} 输出18种组合的验证结果}
\PY{n}{cvres} \PY{o}{=} \PY{n}{grid\PYZus{}search}\PY{o}{.}\PY{n}{cv\PYZus{}results\PYZus{}}
\PY{k}{for} \PY{n}{mean\PYZus{}score}\PY{p}{,} \PY{n}{params} \PY{o+ow}{in} \PY{n+nb}{zip}\PY{p}{(}\PY{n}{cvres}\PY{p}{[}\PY{l+s+s2}{\PYZdq{}}\PY{l+s+s2}{mean\PYZus{}test\PYZus{}score}\PY{l+s+s2}{\PYZdq{}}\PY{p}{]}\PY{p}{,} \PY{n}{cvres}\PY{p}{[}\PY{l+s+s2}{\PYZdq{}}\PY{l+s+s2}{params}\PY{l+s+s2}{\PYZdq{}}\PY{p}{]}\PY{p}{)}\PY{p}{:}
    \PY{n+nb}{print}\PY{p}{(}\PY{n}{np}\PY{o}{.}\PY{n}{sqrt}\PY{p}{(}\PY{o}{\PYZhy{}}\PY{n}{mean\PYZus{}score}\PY{p}{)}\PY{p}{,} \PY{n}{params}\PY{p}{)}
\end{Verbatim}
\end{tcolorbox}

    \begin{Verbatim}[commandchars=\\\{\}]
63851.160237887176 \{'max\_features': 2, 'n\_estimators': 3\}
55551.02838972032 \{'max\_features': 2, 'n\_estimators': 10\}
52662.41018371984 \{'max\_features': 2, 'n\_estimators': 30\}
60377.04243720946 \{'max\_features': 4, 'n\_estimators': 3\}
53506.29857019163 \{'max\_features': 4, 'n\_estimators': 10\}
50855.95139995473 \{'max\_features': 4, 'n\_estimators': 30\}
58255.57633989498 \{'max\_features': 6, 'n\_estimators': 3\}
52157.96552289658 \{'max\_features': 6, 'n\_estimators': 10\}
50341.09758695334 \{'max\_features': 6, 'n\_estimators': 30\}
59395.29817637189 \{'max\_features': 8, 'n\_estimators': 3\}
52506.71176229568 \{'max\_features': 8, 'n\_estimators': 10\}
50381.98175578956 \{'max\_features': 8, 'n\_estimators': 30\}
61685.385884779884 \{'bootstrap': False, 'max\_features': 2, 'n\_estimators': 3\}
54108.082179696845 \{'bootstrap': False, 'max\_features': 2, 'n\_estimators': 10\}
60302.23023563375 \{'bootstrap': False, 'max\_features': 3, 'n\_estimators': 3\}
52395.76981294647 \{'bootstrap': False, 'max\_features': 3, 'n\_estimators': 10\}
58492.78323075625 \{'bootstrap': False, 'max\_features': 4, 'n\_estimators': 3\}
52050.21236959879 \{'bootstrap': False, 'max\_features': 4, 'n\_estimators': 10\}
    \end{Verbatim}

    \begin{tcolorbox}[breakable, size=fbox, boxrule=1pt, pad at break*=1mm,colback=cellbackground, colframe=cellborder]
\prompt{In}{incolor}{94}{\boxspacing}
\begin{Verbatim}[commandchars=\\\{\}]
\PY{c+c1}{\PYZsh{} 数据准备步骤也可以当作超参数来处理,网格搜索会自动查找是否添加你不确定的特征,比如是否使用转换器 combinedAttre的超参数add\PYZus{}bedrooms\PYZus{}per\PYZus{}rom}
\PY{c+c1}{\PYZsh{} 也可是使用它自动寻找处理问题的最佳方法,比如异常值,缺失特征,特征选择等}
\end{Verbatim}
\end{tcolorbox}

    \hypertarget{ux968fux673aux641cux7d22}{%
\subsection{随机搜索}\label{ux968fux673aux641cux7d22}}

    \begin{tcolorbox}[breakable, size=fbox, boxrule=1pt, pad at break*=1mm,colback=cellbackground, colframe=cellborder]
\prompt{In}{incolor}{95}{\boxspacing}
\begin{Verbatim}[commandchars=\\\{\}]
\PY{k+kn}{from} \PY{n+nn}{sklearn}\PY{n+nn}{.}\PY{n+nn}{model\PYZus{}selection} \PY{k+kn}{import} \PY{n}{RandomizedSearchCV}
\PY{k+kn}{from} \PY{n+nn}{scipy}\PY{n+nn}{.}\PY{n+nn}{stats} \PY{k+kn}{import} \PY{n}{randint}

\PY{n}{param\PYZus{}distribs} \PY{o}{=} \PY{p}{\PYZob{}}
        \PY{l+s+s1}{\PYZsq{}}\PY{l+s+s1}{n\PYZus{}estimators}\PY{l+s+s1}{\PYZsq{}}\PY{p}{:} \PY{n}{randint}\PY{p}{(}\PY{n}{low}\PY{o}{=}\PY{l+m+mi}{1}\PY{p}{,} \PY{n}{high}\PY{o}{=}\PY{l+m+mi}{200}\PY{p}{)}\PY{p}{,}
        \PY{l+s+s1}{\PYZsq{}}\PY{l+s+s1}{max\PYZus{}features}\PY{l+s+s1}{\PYZsq{}}\PY{p}{:} \PY{n}{randint}\PY{p}{(}\PY{n}{low}\PY{o}{=}\PY{l+m+mi}{1}\PY{p}{,} \PY{n}{high}\PY{o}{=}\PY{l+m+mi}{8}\PY{p}{)}\PY{p}{,}
    \PY{p}{\PYZcb{}}

\PY{n}{forest\PYZus{}reg} \PY{o}{=} \PY{n}{RandomForestRegressor}\PY{p}{(}\PY{n}{random\PYZus{}state}\PY{o}{=}\PY{l+m+mi}{42}\PY{p}{)}
\PY{n}{rnd\PYZus{}search} \PY{o}{=} \PY{n}{RandomizedSearchCV}\PY{p}{(}\PY{n}{forest\PYZus{}reg}\PY{p}{,} \PY{n}{param\PYZus{}distributions}\PY{o}{=}\PY{n}{param\PYZus{}distribs}\PY{p}{,}
                                \PY{n}{n\PYZus{}iter}\PY{o}{=}\PY{l+m+mi}{10}\PY{p}{,} \PY{n}{cv}\PY{o}{=}\PY{l+m+mi}{5}\PY{p}{,} \PY{n}{scoring}\PY{o}{=}\PY{l+s+s1}{\PYZsq{}}\PY{l+s+s1}{neg\PYZus{}mean\PYZus{}squared\PYZus{}error}\PY{l+s+s1}{\PYZsq{}}\PY{p}{,} \PY{n}{random\PYZus{}state}\PY{o}{=}\PY{l+m+mi}{42}\PY{p}{)}
\PY{n}{rnd\PYZus{}search}\PY{o}{.}\PY{n}{fit}\PY{p}{(}\PY{n}{housing\PYZus{}prepared}\PY{p}{,} \PY{n}{housing\PYZus{}labels}\PY{p}{)}
\end{Verbatim}
\end{tcolorbox}

            \begin{tcolorbox}[breakable, size=fbox, boxrule=.5pt, pad at break*=1mm, opacityfill=0]
\prompt{Out}{outcolor}{95}{\boxspacing}
\begin{Verbatim}[commandchars=\\\{\}]
RandomizedSearchCV(cv=5, error\_score=nan,
                   estimator=RandomForestRegressor(bootstrap=True,
                                                   ccp\_alpha=0.0,
                                                   criterion='mse',
                                                   max\_depth=None,
                                                   max\_features='auto',
                                                   max\_leaf\_nodes=None,
                                                   max\_samples=None,
                                                   min\_impurity\_decrease=0.0,
                                                   min\_impurity\_split=None,
                                                   min\_samples\_leaf=1,
                                                   min\_samples\_split=2,
                                                   min\_weight\_fraction\_leaf=0.0,
                                                   n\_estimators=100,
                                                   n\_jobs=None,
oob\_score=Fals{\ldots}
                   iid='deprecated', n\_iter=10, n\_jobs=None,
                   param\_distributions=\{'max\_features':
<scipy.stats.\_distn\_infrastructure.rv\_frozen object at 0x0000023B84876948>,
                                        'n\_estimators':
<scipy.stats.\_distn\_infrastructure.rv\_frozen object at 0x0000023B84886088>\},
                   pre\_dispatch='2*n\_jobs', random\_state=42, refit=True,
                   return\_train\_score=False, scoring='neg\_mean\_squared\_error',
                   verbose=0)
\end{Verbatim}
\end{tcolorbox}
        
    \begin{tcolorbox}[breakable, size=fbox, boxrule=1pt, pad at break*=1mm,colback=cellbackground, colframe=cellborder]
\prompt{In}{incolor}{96}{\boxspacing}
\begin{Verbatim}[commandchars=\\\{\}]
\PY{n}{cvres} \PY{o}{=} \PY{n}{rnd\PYZus{}search}\PY{o}{.}\PY{n}{cv\PYZus{}results\PYZus{}}
\PY{k}{for} \PY{n}{mean\PYZus{}score}\PY{p}{,} \PY{n}{params} \PY{o+ow}{in} \PY{n+nb}{zip}\PY{p}{(}\PY{n}{cvres}\PY{p}{[}\PY{l+s+s2}{\PYZdq{}}\PY{l+s+s2}{mean\PYZus{}test\PYZus{}score}\PY{l+s+s2}{\PYZdq{}}\PY{p}{]}\PY{p}{,} \PY{n}{cvres}\PY{p}{[}\PY{l+s+s2}{\PYZdq{}}\PY{l+s+s2}{params}\PY{l+s+s2}{\PYZdq{}}\PY{p}{]}\PY{p}{)}\PY{p}{:}
    \PY{n+nb}{print}\PY{p}{(}\PY{n}{np}\PY{o}{.}\PY{n}{sqrt}\PY{p}{(}\PY{o}{\PYZhy{}}\PY{n}{mean\PYZus{}score}\PY{p}{)}\PY{p}{,} \PY{n}{params}\PY{p}{)}
\end{Verbatim}
\end{tcolorbox}

    \begin{Verbatim}[commandchars=\\\{\}]
49429.878017658695 \{'max\_features': 7, 'n\_estimators': 180\}
51693.13664471228 \{'max\_features': 5, 'n\_estimators': 15\}
50633.1648206924 \{'max\_features': 3, 'n\_estimators': 72\}
51317.56932852033 \{'max\_features': 5, 'n\_estimators': 21\}
49509.913572343365 \{'max\_features': 7, 'n\_estimators': 122\}
50579.593781990894 \{'max\_features': 3, 'n\_estimators': 75\}
50541.24887362916 \{'max\_features': 3, 'n\_estimators': 88\}
49891.34112450202 \{'max\_features': 5, 'n\_estimators': 100\}
50366.44470237395 \{'max\_features': 3, 'n\_estimators': 150\}
65269.29200804345 \{'max\_features': 5, 'n\_estimators': 2\}
    \end{Verbatim}

    \hypertarget{ux5206ux6790ux6700ux4f73ux6a21ux578b}{%
\subsection{分析最佳模型}\label{ux5206ux6790ux6700ux4f73ux6a21ux578b}}

    \begin{tcolorbox}[breakable, size=fbox, boxrule=1pt, pad at break*=1mm,colback=cellbackground, colframe=cellborder]
\prompt{In}{incolor}{97}{\boxspacing}
\begin{Verbatim}[commandchars=\\\{\}]
\PY{c+c1}{\PYZsh{} 输出最佳估算器里最优的特征值}
\PY{n}{feature\PYZus{}importances} \PY{o}{=} \PY{n}{grid\PYZus{}search}\PY{o}{.}\PY{n}{best\PYZus{}estimator\PYZus{}}\PY{o}{.}\PY{n}{feature\PYZus{}importances\PYZus{}}
\PY{n}{feature\PYZus{}importances}
\end{Verbatim}
\end{tcolorbox}

            \begin{tcolorbox}[breakable, size=fbox, boxrule=.5pt, pad at break*=1mm, opacityfill=0]
\prompt{Out}{outcolor}{97}{\boxspacing}
\begin{Verbatim}[commandchars=\\\{\}]
array([6.62529439e-02, 6.03695060e-02, 4.41513588e-02, 1.86235401e-02,
       1.75334150e-02, 1.78450483e-02, 1.69535923e-02, 2.38691040e-01,
       1.66701200e-01, 5.41161548e-02, 1.08951989e-01, 5.43083202e-02,
       1.46716959e-02, 1.13568649e-01, 1.49393679e-04, 2.67238220e-03,
       4.43977116e-03])
\end{Verbatim}
\end{tcolorbox}
        
    \begin{tcolorbox}[breakable, size=fbox, boxrule=1pt, pad at break*=1mm,colback=cellbackground, colframe=cellborder]
\prompt{In}{incolor}{114}{\boxspacing}
\begin{Verbatim}[commandchars=\\\{\}]
\PY{c+c1}{\PYZsh{} num\PYZus{}attribs}
\PY{n}{extra\PYZus{}attribs} \PY{o}{=} \PY{p}{[}\PY{l+s+s2}{\PYZdq{}}\PY{l+s+s2}{rooms\PYZus{}per\PYZus{}household}\PY{l+s+s2}{\PYZdq{}}\PY{p}{,} \PY{l+s+s2}{\PYZdq{}}\PY{l+s+s2}{population\PYZus{}per\PYZus{}household}\PY{l+s+s2}{\PYZdq{}}\PY{p}{,} \PY{l+s+s2}{\PYZdq{}}\PY{l+s+s2}{bedrooms\PYZus{}per\PYZus{}room}\PY{l+s+s2}{\PYZdq{}}\PY{p}{]}
\PY{c+c1}{\PYZsh{} encoder.classes\PYZus{}:array([\PYZsq{}\PYZlt{}1H OCEAN\PYZsq{}, \PYZsq{}INLAND\PYZsq{}, \PYZsq{}ISLAND\PYZsq{}, \PYZsq{}NEAR BAY\PYZsq{}, \PYZsq{}NEAR OCEAN\PYZsq{}],dtype=object)}
\PY{n}{cat\PYZus{}one\PYZus{}hot\PYZus{}attribs} \PY{o}{=} \PY{n+nb}{list}\PY{p}{(}\PY{n}{encoder}\PY{o}{.}\PY{n}{classes\PYZus{}}\PY{p}{)} 
\PY{n}{attributes} \PY{o}{=} \PY{n}{num\PYZus{}attribs} \PY{o}{+} \PY{n}{extra\PYZus{}attribs} \PY{o}{+} \PY{n}{cat\PYZus{}one\PYZus{}hot\PYZus{}attribs}
\PY{c+c1}{\PYZsh{} 最优特征与列打包排序}
\PY{n+nb}{sorted}\PY{p}{(}\PY{n+nb}{zip}\PY{p}{(}\PY{n}{feature\PYZus{}importances}\PY{p}{,} \PY{n}{attributes}\PY{p}{)}\PY{p}{,} \PY{n}{reverse} \PY{o}{=} \PY{k+kc}{True}\PY{p}{)}
\end{Verbatim}
\end{tcolorbox}

            \begin{tcolorbox}[breakable, size=fbox, boxrule=.5pt, pad at break*=1mm, opacityfill=0]
\prompt{Out}{outcolor}{114}{\boxspacing}
\begin{Verbatim}[commandchars=\\\{\}]
[(0.23869103994492774, 'median\_income'),
 (0.16670119958492133, 'income\_cat'),
 (0.11356864935614602, 'INLAND'),
 (0.10895198866919581, 'population\_per\_household'),
 (0.06625294391455755, 'longitude'),
 (0.06036950601771953, 'latitude'),
 (0.05430832019926962, 'bedrooms\_per\_room'),
 (0.05411615480259234, 'rooms\_per\_household'),
 (0.044151358830570565, 'housing\_median\_age'),
 (0.018623540113329517, 'total\_rooms'),
 (0.017845048302505565, 'population'),
 (0.01753341500142464, 'total\_bedrooms'),
 (0.01695359231755946, 'households'),
 (0.014671695912934811, '<1H OCEAN'),
 (0.004439771155847507, 'NEAR OCEAN'),
 (0.0026723821979612082, 'NEAR BAY'),
 (0.00014939367853693025, 'ISLAND')]
\end{Verbatim}
\end{tcolorbox}
        # zip()
>>>a = [1,2,3]
>>> b = [4,5,6]
>>> c = [4,5,6,7,8]
>>> zipped = zip(a,b)     # 打包为元组的列表
[(1, 4), (2, 5), (3, 6)]
>>> zip(a,c)              # 元素个数与最短的列表一致
[(1, 4), (2, 5), (3, 6)]
>>> zip(*zipped)          # 与 zip 相反,*zipped 可理解为解压,返回二维矩阵式
[(1, 2, 3), (4, 5, 6)]
    \hypertarget{ux901aux8fc7ux6d4bux8bd5ux96c6ux8bc4ux4f30ux7cfbux7edf}{%
\subsection{通过测试集评估系统}\label{ux901aux8fc7ux6d4bux8bd5ux96c6ux8bc4ux4f30ux7cfbux7edf}}

\begin{enumerate}
\def\labelenumi{\arabic{enumi}.}
\tightlist
\item
  从测试集中获取预测器和标签
\item
  运行full\_pipline来转换数据
\item
  在测试集上评估最终模型
\end{enumerate}

    \begin{tcolorbox}[breakable, size=fbox, boxrule=1pt, pad at break*=1mm,colback=cellbackground, colframe=cellborder]
\prompt{In}{incolor}{119}{\boxspacing}
\begin{Verbatim}[commandchars=\\\{\}]
\PY{n}{final\PYZus{}model} \PY{o}{=} \PY{n}{grid\PYZus{}search}\PY{o}{.}\PY{n}{best\PYZus{}estimator\PYZus{}}  \PY{c+c1}{\PYZsh{} 最优估算器}

\PY{n}{X\PYZus{}test} \PY{o}{=} \PY{n}{strat\PYZus{}test\PYZus{}set}\PY{o}{.}\PY{n}{drop}\PY{p}{(}\PY{l+s+s1}{\PYZsq{}}\PY{l+s+s1}{median\PYZus{}house\PYZus{}value}\PY{l+s+s1}{\PYZsq{}}\PY{p}{,} \PY{n}{axis} \PY{o}{=} \PY{l+m+mi}{1}\PY{p}{)}  \PY{c+c1}{\PYZsh{} 删除标签列}
\PY{n}{y\PYZus{}test} \PY{o}{=} \PY{n}{strat\PYZus{}test\PYZus{}set}\PY{p}{[}\PY{l+s+s1}{\PYZsq{}}\PY{l+s+s1}{median\PYZus{}house\PYZus{}value}\PY{l+s+s1}{\PYZsq{}}\PY{p}{]}\PY{o}{.}\PY{n}{copy}\PY{p}{(}\PY{p}{)}  \PY{c+c1}{\PYZsh{} 标签列}

\PY{n}{X\PYZus{}test\PYZus{}prepared} \PY{o}{=} \PY{n}{full\PYZus{}pipeline}\PY{o}{.}\PY{n}{transform}\PY{p}{(}\PY{n}{X\PYZus{}test}\PY{p}{)}  \PY{c+c1}{\PYZsh{} transform估算}

\PY{c+c1}{\PYZsh{} df = pd.DataFrame(X\PYZus{}test\PYZus{}prepared)}
\PY{c+c1}{\PYZsh{} df.head()}

\PY{n}{final\PYZus{}predictions} \PY{o}{=} \PY{n}{final\PYZus{}model}\PY{o}{.}\PY{n}{predict}\PY{p}{(}\PY{n}{X\PYZus{}test\PYZus{}prepared}\PY{p}{)}

\PY{n}{final\PYZus{}mse} \PY{o}{=} \PY{n}{mean\PYZus{}squared\PYZus{}error}\PY{p}{(}\PY{n}{y\PYZus{}test}\PY{p}{,} \PY{n}{final\PYZus{}predictions}\PY{p}{)}
\PY{n}{final\PYZus{}rmse} \PY{o}{=} \PY{n}{np}\PY{o}{.}\PY{n}{sqrt}\PY{p}{(}\PY{n}{final\PYZus{}mse}\PY{p}{)}
\PY{n}{final\PYZus{}rmse}
\end{Verbatim}
\end{tcolorbox}

            \begin{tcolorbox}[breakable, size=fbox, boxrule=.5pt, pad at break*=1mm, opacityfill=0]
\prompt{Out}{outcolor}{119}{\boxspacing}
\begin{Verbatim}[commandchars=\\\{\}]
50527.33631320635
\end{Verbatim}
\end{tcolorbox}
        
    \hypertarget{ux9879ux76eeux542fux52a8ux9636ux6bb5}{%
\subsection{项目启动阶段}\label{ux9879ux76eeux542fux52a8ux9636ux6bb5}}

\begin{enumerate}
\def\labelenumi{\arabic{enumi}.}
\tightlist
\item
  展示解决方案 学习了什么
\item
  什么有用
\item
  什么没有用
\item
  基于什么假设
\item
  以及系统的限制有哪些
\item
  制作漂亮的演示文稿,例如收入中位数是预测房价的首要指标
\end{enumerate}

\hypertarget{ux542fux52a8ux76d1ux63a7ux548cux7ef4ux62a4ux7cfbux7edf}{%
\subsection{启动,监控和维护系统}\label{ux542fux52a8ux76d1ux63a7ux548cux7ef4ux62a4ux7cfbux7edf}}

\begin{enumerate}
\def\labelenumi{\arabic{enumi}.}
\tightlist
\item
  为生产环境做好准备,将生产数据源接入系统
\item
  编写监控代码,定期检查系统的实时性能,性能下降时触发警报,系统崩溃和性能退化
\item
  时间推移,模型会渐渐腐坏,定期使用新数据训练模型
\end{enumerate}

\hypertarget{ux8bc4ux4f30ux7cfbux7edfux6027ux80fd}{%
\subsection{评估系统性能}\label{ux8bc4ux4f30ux7cfbux7edfux6027ux80fd}}

\begin{enumerate}
\def\labelenumi{\arabic{enumi}.}
\tightlist
\item
  需要对系统的预测结果进行抽样评估,需要人工分析,分析师可能是专家,平台工作人员,都需要将人工评估的流水线接入你的系统
\item
  评估输入系统的数据质量
\item
  使用新鲜数据定期训练你的模型,最多6个月
\end{enumerate}

\hypertarget{ux603bux7ed3}{%
\subsection{总结}\label{ux603bux7ed3}}

\begin{verbatim}
本周主要学习 数据准备,构建监控工作,建立人工评估流水线,自动化定期训练模型上,熟悉整个机器学习流程,
\end{verbatim}

\hypertarget{ux5efaux8bae}{%
\subsection{建议}\label{ux5efaux8bae}}

\begin{verbatim}
选择一个数据集,尝试从A到Z的整个过程,从竞赛网站上,选择一个数据集,一个明确目标,以及可以一起分享经验的同伴
https://www.kaggle.com/
\end{verbatim}

    


    % Add a bibliography block to the postdoc
    
    
    
\end{document}
